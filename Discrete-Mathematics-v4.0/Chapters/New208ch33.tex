\chapter{Tougher Counting Problems}\label{ch:Tougher Counting Problems}

\newthought{All of the counting exercises} you've been asked to complete so far 
have not been realistic. In general it won't be true that a counting problem
fits neatly into a section. So we need to work on the bigger picture.

When we start any counting exercise it is true that there is an underlying exercise at the
basic level that we want to consider first. So instead of answering the question immediately
we might first want to decide on what type of exercise we have. So far
we have seen three types which are distinguishable by the answers to 
two questions.
\begin{enumerate}
 \item In forming the objects we want to count, is repetition  allowed?
 \item In forming the objects we want to count, does the order of selection matter?
\end{enumerate}

The three scenarios we have seen so far are described in table~\ref{tbl:basic count probs}.
\begin{margintable}
\centering
\begin{tabular}{cclc}
\toprule
Order & Repetition & Type & Form \\
\midrule
Y   &  Y   &  $r$-strings & $n^r$   \\
Y   &  N   &  $r$-permutations & $P(n,r)$   \\
N   &  N   &  $r$-combinations & ${n\choose r}$  \\
\bottomrule %\addlinespace
\end{tabular} %
\caption{Basic counting problems}\label{tbl:basic count probs}
\end{margintable}

There are two problems to address. First of all, table~\ref{tbl:basic count probs}
 is incomplete.
What about, for example, counting objects where repetition is allowed, but order
doesn't matter. Second of all, there are connections among the types which
make some solutions appear misleading. But as a general rule of thumb, if we
correctly identify the type of problem we are working on, then all we have to do
is use the principles of addition, multiplication, inclusion/exclusion, or exclusion
to decompose our problem into subproblems. The solutions to the subproblems 
often have the same form as the underlying problem. The principles we employed
direct us on how the sub-solutions should be recombined to give the final answer.

\begin{exmp}
 As an example of the second problem, if we ask how many binary strings of length $10$ contain
 exactly three $1$'s, then the underlying problem is an $r$-string problem. But in this case
 the answer is $\dl{{10\choose 3}}$. Of course this is really $\dl{{10\choose 3}1^31^7}$ from
 the binomial theorem. In this case the part of the answer which looks like $n^r$ is suppressed
 since it's trivial. To see the difference we might ask how many ternary strings of length $10$
 contain exactly three $1$'s. Now the answer is $\dl{{10\choose 3} 1^3 2^7}$, since we choose
 the three positions for the $1$'s to go in, and then fill in each of the $7$ remaining positions with a $0$ or a $2$.
\end{exmp}



\section{The Basic Donut Shop Problem}\label{sect:basic donut shop problem}
To begin to address the first problem we introduce
If you get to the donut shop before the cops get there,
you will find that they have a nice variety of donuts. You might want to order several dozen.
They will put your order in a box. You don't particularly care what order the donuts are put into
the box. You do usually want more than one of several types. The number of ways for you to
complete your order is therefore a counting problem where order doesn't matter, and repetition
is allowed.

In order to answer the question of how many ways you can complete your order, we first
recast the problem mathematically. From among $n$ types of objects we want to select 
$r$ objects. If $x_i$ denotes the number of objects of the $i$th type selected, we have
$0\leq x_i$, (since we cannot choose a negative number of chocolate donouts), also
$x_i\in \Z$, (since we cannot select fractional parts of donuts). So, the different ways to order 
are in one-to-one correspondence with the solutions  to
\[
 x_1+x_2+...+x_n=r, \text{ with $x_i \geq 0$, $x_i \in \Z$, for $i-1,2,\dots,n$.}
\]
Next, in order to compute the number of solutions in non-negative integers to $x_1+x_2+...+x_n=r $, 
we model each solution as a string (possibly empty) of $x_1$ 1's followed by a $+$, 
then a string of $x_2$ 1's followed by a $+$, ... then a string of $x_{n-1}$ 1's followed by a $+$, then
a string of $x_n$ 1's. So for example, if $x_1=2, x_2=0, x_3=1, x_4=3$ is a solution to $x_1+x_2+x_3+x_4=6$
the string we get is $11++1+111$. 

Finally, we see that
 the total number of solutions in non-negative integers to 
$x_1+...+x_n=r$, is the number of binary strings of length $r+n-1$ with exactly $r$ $1$'s and
$(n-1)$ $+$'s.
From the remark above, the number of ways to select $r$ donuts from $n$ different types
 is 
\[
 \binom{n+r-1}{r}.
\]


\section{The More Realistic Donut Shop Problem} \label{sect:more realistic donut shop problem}
The basic donut shop problem is not very realistic in two ways. First it is common that some of your order
will be determined by other people. You might for example canvas the people in your office before
you go to see if there is anything you can pick up for them. So whereas you want to order $r$ donuts,
you might have been asked to pick up a certain number of various types.

Now suppose that we know that we want to select
$r$ donuts from among $n$ types so that at least $a_i(a_i\geq 0)$ donuts of type $i$ are selected.
In terms of our equation, we have $x_1+x_2+...+x_n=r$, where $a_i\leq x_i$, and $x_i\in \Z$.
If we set $y_i=x_i-a_i$ for $i=1,...,n$, and $\dl{a=\sum_{i=1}^n a_i}$, then $0\leq y_i$, $y_i\in \Z$ and 
\[
\sum_{i=1}^n y_i =\sum_{i=1}^n (x_i-a_i)=\left[\sum_{i=1}^n x_i\right]-\left[\sum_{i=1}^n a_i\right]=r-a.
\]
So, the number of ways to complete our order is $\displaystyle{\binom{n+(r-a)-1}{(r-a)}}$.
\marginnote{First ask for the donuts your colleagues wanted (total of $a$), then randomly get
the rest ($r-a$).}


Still, we qualified the donut shop problem by supposing that we arrived before the cops did.


\section{The Real Donut Shop Problem}\label{sect:real donut shop problem}
 If we arrive at the donut shop after canvassing our friends,
we want to select $r$ donuts from among $n$ types. The problem is that there are probably only 
a few left of each type. This may place an upper limit on how often we can select a particular type.
So now we wish to count solutions to 
\[
 x_1+x_2+...+x_n=r, \text{ with $a_i\leq x_i\leq b_i$, $x_i \in \Z$.}
\]
 
We proceed by replacing $r$ by $s=r-a$, where $a$ is the sum of lower bounds. 
We also replace $b_i$ by $c_i=b_i-a_i$ for $i=1,..., n$. So we want to find the number of solutions
to $0\leq y_i\leq c_i$, $y_i\in \Z$, and $y_1+y_2+...+y_n=s$. There are several ways to proceed.
We choose inclusion/exclusion. Let us set $\U$ to be all solutions in non-negative integers to $y_1+...+y_n=s$.
Next let $A_i$ denote those solutions in non-negative integers to $y_1+...+y_n=r$, where $c_i<y_i$.
Then we want to compute $|\vl{A_1}\cap \vl{A_2}\cap \vl{A_3}\cap ...\cap \vl{A_n}|$, which we can do
by general inclusion/exclusion, and the ideas from the more realistic donut shop problem.

\begin{exmp}
 Let us count the number of solutions to 
 \[
 x_1+x_2+x_3+x_4=34,
 \]
 where
 $0\leq x_1\leq 4$, $0\leq x_2\leq 5$, $0\leq x_3\leq 8$ and $0\leq x_4\leq 40$.
 
 As discussed above, we
 have $c_1=4, c_2=5, c_3=8$, and $c_4=40$. Hence, we see that 
 \[
 \left\lvert \U\right\rvert=\binom{34+4-1}{34}.
 \]
 Now, $A_i$ will denote the solutions in non-negative
 integers to 
 \[
 x_1+x_2+x_3+x_4=34, \text{ with } x_i>c_i, \text{ for } i=1,2,3,4.
 \]
 Next, realize that $A_4=\emptyset$, so we have 
 \[
 \vl{A_4}=\U \quad\text{ and }\quad 
 \vl{A_1}\cap \vl{A_2}\cap \vl{A_3}\cap \vl{A_4}=\vl{A_1}\cap \vl{A_2}\cap \vl{A_3}.
 \]
 Now, to compute $A_1$, we must first rephrase $x_1>4$ as a non-strict inequality, i.e.~$5\leq x_1$.
 So, it follows that 
 \[
 {|A_1|={29+4-1\choose 29}}.
 \]
 Similarly, we have
 \[
  \lvert A_2\rVert=\binom{28+4-1}{28}, \quad\text{ and }\quad
 \lvert A_3\rvert=\binom{25+4-1}{25}.
 \] 
 Next, we observe that $A_1\cap A_2$ represents the set of  all solutions in non-negative
 integers to
 \[
 x_1+x_2+x_3+x_4=34 \quad\text{with $5\leq x_1$ and $6\leq x_2$}.
 \]
  So, we have 
 \[
 \left\lvert A_1\cap A_2\right\rvert=\binom{23+4-1}{23}.
 \]
  Also, we find that
 \[
  \lvert A_1\cap A_3\rvert=\binom{20+4-1}{20}
  \quad\text{and}\quad \lvert A_2\cap A_3\rvert=\binom{19+4-1}{19}.
 \]
  Finally, we see that
 \[
 \lvert A_1\cap A_2\cap A_3\rvert=
 \binom{14+4-1}{14}.
 \]
 Hence, the final answer is\sidenote{We leave the answer in this form for clarity. The numerical value is not illuminating.}
\begin{align*}
  {34+4-1\choose 34}&-{29+4-1\choose 29}-{28+4-1\choose 28}-{25+4-1\choose 25}\\
  +{23+4-1\choose 23}&+{20+4-1\choose 20}+{19+4-1\choose 19}-{14+4-1\choose 14}.
\end{align*}
\end{exmp}

We can now solve general counting exercises where order is unimportant and repetition is restricted
somewhere between no repetition, and full repetition.


\section{Problems with order and some repetition}
To complete the picture we should be able to also solve counting exercises where order is important
and repetition is partial. This is somewhat easier. It suffices to consider the subcases in
 example~\ref{exmp:quaternary 15}.


\begin{exmp}\label{exmp:quaternary 15}
Let us take as initial problem the number of quaternary strings of length $15$.
There are $4^{15}$ of these. 

Now, if we ask how many contain exactly two $0$'s, the answer is
$\dl{{15\choose 2}3^{13}}$. 

If we ask how many contain exactly two $0$'s and four $1$'s, the answer
is 
\[
\binom{15}{2}\binom{13}{4}2^9.
\]

And, if we ask how many contain exactly two $0$'s, four $1$'s and
five $2$'s, the answer is
\[
\binom{15}{2}\binom{13}{4}\binom{9}{5}\binom{4}{4} = \frac{15!}{2!\cdot4!\cdot5!\cdot4!}.
\]
\end{exmp}
So, in fact many types of counting are related by what we call the multinomial theorem.
\begin{thm}
 When $r$ is a non-negative integer and $x_1,x_2,...,x_n\in \R$, we have
 \[
  (x_1+x_2+...+x_n)^r=\sum_{{e_1+e_2+...+e_n=r\atop 0\leq e_i}} 
  \binom{r}{e_1,e_2,...,e_n} x_1^{e_1} x_2^{e_2}... x_n^{e_n},
 \]
 where $\displaystyle {\binom{r}{e_1,e_2,...e_n}=\frac{r!}{e_1!e_2!...e_n!}}$.
\end{thm}

\section{The six fundamental counting problems}
To recap, when we have a counting exercise, we should first ask whether order is important and then
ask whether repetition is allowed. This will get us into the right ballpark as far as the form of the solution.
We must use basic counting principles to decompose the exercise into sub-problems. Solve the 
sub-problems, and put the pieces back together. Solutions to sub-problems usually take the same
form as the underlying problem, though they may be related to it via the multinomial theorem. 
Table~\ref{tbl:six counting problems} synopsizes our six fundamental cases.
\begin{margintable}
\centering
\begin{tabular}{ccc}
\toprule
Order & Repetition & Form \\
\midrule
Y   &  Y   &  $\dl{n^r}$   \\ \addlinespace
Y   &  N   &  $\dl{P(n,r)}$   \\ \addlinespace
N   &  Y   &  $\dl{{r+n-1\choose r}}$ \\ \addlinespace
N   &  N   &  $\dl{{n\choose r}}$  \\ \addlinespace 
Y   & some & $\dl{{r\choose k_1,k_2,...,k_n}}$ \\ \addlinespace
N   & some & $\dl{{r+n-1\choose r}}$ w/ I-E \\ \addlinespace
\bottomrule
\end{tabular} %
\caption{Six counting problems}\label{tbl:six counting problems}
\end{margintable}



\clearpage
\section{Exercises}

\begin{exer}
How many quaternary strings of length $n$ are there (a quaternary string uses 0's, 1's, 2's, and 3's)?
\end{exer}

\begin{exer}
How many quaternary strings of length less than or equal to 7 are there?
\end{exer}

\begin{exer}
How many solutions in integers are there to $x_1+x_2+x_3+x_4+x_5+x_6+x_7=54$, where $3\leq x_1$, $4\leq x_2$, $5\leq x_3$, and $6\leq x_4,x_5,x_6, x_7$?
\end{exer}

\begin{exer}
\begin{fullwidth}
How many ternary strings of length $n$ start $0101$ and end $212$?
\end{fullwidth}
\end{exer}

\begin{exer} 
A doughnut shop has 8 kinds of doughnuts: chocolate, glazed, sugar, cherry, strawberry, vanilla, caramel, and jalapeno. How many ways are there to order three 
dozen doughnuts, if at least 4 are jalapeno, at least 6 are cherry, and at least 8 are strawberry, but there are no restrictions on the other varieties?
\end{exer}

\begin{exer}

How many strings of twelve lowercase English letters are there\\
\begin{enumerate}[label=(\alph*)]
 \item which start and end with the letter $x$, if letters may be repeated?
 \item which contain the letter $x$ exactly once, if letters can be repeated?
 \item which contain each of the letters $x$ and $y$ both exactly once, if letters can be repeated?
 \item which contain at least one letter from the first half of the alphabet (a through m), where letters may be repeated?
\end{enumerate}

\end{exer}


\begin{exer}
How many bit strings of length 19 either begin $0101$, or have 4th, 5th and 6th digits $101$, or end $1010$?
\end{exer}

\begin{exer}
How many pentary strings of length 15 consist of three 0's, four 1's, three 2's, four 3's and one 4?
\end{exer}

\begin{exer}%9
9
\end{exer}

\begin{exer}
Seven lecturers and fourteen professors are on the faculty of a math department.
\begin{enumerate}[label=(\alph*)]
\item How many ways are there to form a committee with seven members which contains more lecturers
      than professors?
\item How many ways are there to form a committee with seven members where the professors outnumber
      the lecturers on the committee by at least a two-to-one margin?
\item How many ways are there to form a committee consisting of at least five lecturers?
\end{enumerate}
\end{exer}

\begin{exer}
In how many ways can twenty people form a line at a ticket window if Hans and wife Brunhilda are having a spat, and refuse to stand in consecutive places in the line?

\end{exer}

\begin{exer}
Prove that a set with $n\geq 1$ elements has the same number of subsets with an even number of elements, as
subsets with an odd number of elements.
\end{exer}

\section{Problems}

\begin{prob}
A doughnut shop has 8 kinds of doughnuts: chocolate, glazed, sugar, cherry, strawberry, vanilla, caramel, and jalapeno. How many ways are there to order three 
dozen doughnuts, if at most 4 are jalapeno, at most 6 are cherry, and at most 8 are strawberry, but there are no restrictions on the other varieties?
\end{prob}

\begin{prob}
\begin{fullwidth}
How many strings of twelve lowercase English letters are there
\end{fullwidth}
\begin{enumerate}[label=(\alph*)]
 \item which start with the letter $x$, if letters may be repeated?
 \item which contain the letter $x$ at least once, if letters can be repeated?
 \item which contain each of the letters $x$ and $y$ at least once, if letters can be repeated?
 \item which contain at least one vowel, where letters may not be repeated?
\end{enumerate}

\end{prob}

\begin{prob}
How many ternary strings of length 9 have
\begin{enumerate}[label=(\alph*)]
 \item exactly four 1's?
 \item at least three 0's?
 \item at most three 1's?
\end{enumerate}
\end{prob}


\begin{prob}
How many ways are there to seat six people at a circular table where two seatings are considered equivalent if one
can be obtained from the other by rotating the table?
\end{prob}

\begin{prob}
A donut shop sells six types of donut. You buy a scratch{-}off ticket that promises you will win anywhere from one to two dozen donuts. How many different prizes are possible?
\end{prob}

\begin{prob}
How many ternary strings of length $n$ contain no two adjacent identical symbols? Examples: ($n = 8$)
$13123231$ is good, but $13112321$ is bad.
\end{prob}

\begin{prob}
How many ternary strings of length $n$ contain at least two adjacent identical symbols? Examples: ($n=9$)
$131212321$ is bad, but $131123321$ is good.
\end{prob}

\begin{prob}
We have five identical buckets, and six markers numbered $1$ to $6$. How many final distributions of the markers are possible? Examples: one distribution: markers 1, 3 together in one bucket, markers 2, 4,
5 in another bucket, and marker 6 in another bucket. A second possible distribution: markers 1,2,3,4 together in one bucket and markers 5,6 in another bucket. One more example: markers 1,2,3,4, each in a different bucket,
and 5,6 together in one bucket.
\end{prob}

