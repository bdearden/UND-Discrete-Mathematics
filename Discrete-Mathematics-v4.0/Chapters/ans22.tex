   \section*{Chapter 22}
    
\begin{Solution}{22.1a}

\begin{align*}
 233 &= 2\cdot89+55 \\
 89 &= 1\cdot55+34 \\
  55 &= 1\cdot34+21 \\
  34 &= 1\cdot21+13\\
  21  &= 1\cdot13+8 \\
  13  &= 1\cdot8+5 \\
  8  &= 1\cdot5+3 \\
  5  &= 1\cdot3+2 \\
  3  &= 1\cdot2+1 \\
  2  &= 2\cdot1+0 \\
  \end{align*}
 
$\gcd(233,89) = 1$



\end{Solution}

\begin{Solution}{22.1b}


\begin{align*}
 1001 &= 77\cdot13+0 \\
 \end{align*}
 
$\gcd(1001,13) = 13$

\end{Solution}

\begin{Solution}{22.1c}


\begin{align*}
 2457 &= 1\cdot 1458+999 \\
 1458 &= 1\cdot999+ 459 \\
  99 &= 2\cdot459+81 \\
  459 &= 5\cdot81+54 \\
  81  &= 1\cdot54+27 \\
  54  &= 2\cdot27+0 \\
  \end{align*}

$\gcd(2457,1458) = 27$

\end{Solution}

\begin{Solution}{22.1d}


\begin{align*}
 567 &= 1\cdot349+218 \\
 349 &= 1\cdot218+131 \\
  218 &= 1\cdot131+87 \\
  131 &= 1\cdot87+44 \\
  87  &= 1\cdot44+43 \\
  44  &= 1\cdot43+1 \\
  43 &= 43\cdot1 + 0\\
\end{align*}

$\gcd(567,349) = 1$

\end{Solution}

\begin{Solution}{22.2}

\begin{align*} 
  987654321  &= 8\cdot123456789+9 \\
  123456789  &= 13717421\cdot9+0
\end{align*}

$\gcd(987654321,123456789) = 9$

\end{Solution}

\begin{Solution}{22.3}
There are many good answers to this problem. 
\begin{algrthm}
  \hrule\kern5pt\relax
  \begin{algorithmic}%[1]
     \Require{integers $m\geq 0$ and $n>0$}
     \Ensure{integer value of $\gcd(m,n)$} 
     \State $q \gets m$ \Comment{$q$ represents a quotient of a division}
     \State $r \gets n$ \Comment{$r$ represents the remainder of a division}     
     \While{$r > 0$}
       \State $g \gets r$ \Comment{Save current $r$. It will hold the gcd eventually}
       \State $q \gets \lfloor \frac{m}{n}\rfloor$ \Comment{New quotient when $m$ is divided by $n$}
       \State $r \gets m - qn$ \Comment{New remainder when $m$ is divided by $n$}
       \State $m \gets n$ \Comment{Update $m$}
       \State $n \gets r$   \Comment{Update $n$}
     \EndWhile
     \State \textbf{output} $g$ \Comment{The last nonzero remainder}
\end{algorithmic}
  \hrule\kern5pt\relax

\end{algrthm}

\end{Solution}

\begin{Solution}{22.4}
Since $n$ divides both $n$ and $2n$, that means $n$
is a common divisor of $n$ and $2n$. On the other hand, no integer larger
than $n$ can divide $n$. So $n$ is the largest common divisor of $n$ and $2n$.
So $\gcd(n,2n) = n$.
\end{Solution}


