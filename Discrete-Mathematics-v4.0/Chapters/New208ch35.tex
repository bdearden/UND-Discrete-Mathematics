\chapter{Solutions to Recurrence Relations}\label{chpt:solns to recur rels}

\newthought{In chapter~\ref{chpt:counting use recur}, it was} pointed out that 
recursively defined
sequences suffer from one major drawback: In order to compute a
particular term in the sequence, it is necessary to first compute all
the terms of the sequence leading up to the one that is wanted. Imagine
the chore to calculate the $250^{th}$ Fibonacci number, $f_{250}$! For
problems of computation, there is nothing like having a formula like
$a_n= n^2$, into which it is merely necessary to plug the number of
interest.

\section{Solving a recursion by conjecture}
It may be possible to find a formula for a sequence that is
defined recursively. When that can be done, you have the best of both
the formula and recursive worlds.  If we find a formula for the
terms of a recursively defined sequence, we say we have {\bfseries solved}
the recursion.


\begin{exmp}\label{exmp: recur a_n=2a_{n-1}+1}
 Here is an example: The sequence
 $\{a_n\}$ is defined recursively by the initial condition $a_0=2$, and
 the recursive formula $a_n=2a_{n-1}-1$ for $n\geq1$. If
 the first few terms of this sequence are written out, the results are
 \[
 2,3,5,9,17,33,65,129,\cdots,
 \]
  and it shouldn't be too long before the\marginnote{You have to recognize the slightly hidden powers of $2$: $1,2,4,8,16,32,64,\dots$.}
 pattern becomes clear. In fact, it looks like $a_n=2^n+1$ is the formula
 for $a_n$.
 
 To prove that guess is correct, induction would be the best way to
 go. Here are the details. Just to make everything clear, here is what we
 are going to show: If $a_0=2$, and $a_n=2a_{n-1}-1$ for $n\geq1$, then
 $a_n=2^n+1$ for all $n\geq0$. The basis for the inductive proof is the
 case $n=0$. The correct value for $a_0$ is $2$, and the guessed formula
 has value $2$ when $n=0$, so that checks out. Now for the inductive step:
 suppose that the formula for $a_k$ is correct for a particular
 $k\geq0$. That is, assume $a_k=2^k+1$ for some $k\geq0$. Let's show that the formula must
 also be correct for $a_{k+1}$.  That is, we want to show
 $a_{k+1}=2^{k+1}+1$. Well, we know that $a_{k+1}=2a_k-1$, and hence
 $a_{k+1}=2(2^k+1)-1=2^{k+1}+2-1=2^{k+1}+1$, just as was to be proved.
 It can now be concluded that the formula we guessed is correct for all
 $n\geq0$.
\end{exmp}


In example~\ref{exmp: recur a_n=2a_{n-1}+1}, it was possible to guess the correct formula
for $a_n$ after looking at a few terms.  In most cases the formula will
be so complicated that that sort of guessing will be out of the
question. 


\section{Solving a recursion by unfolding}
There is a method that will nearly automatically solve any recurrence
of the form $a_0 = a$ and for, $n\geq 1$, $a_n= ba_{n-1}+c$ (where
$a,b,c$ are constants). The method is called {\bfseries unfolding}. 

\begin{exmp}
 As an example, let's solve $a_0=2$ and, for $n\geq 1$, $a_n=5+2a_{n-1}$.
 The plan is to write down the recurrence relation, and then substitute
 for $a_{n-1}$, then for $a_{n-2}$, and so on, until we reach $a_0$.
 It looks like this
 \begin{align*}
  a_n & = 5+2a_{n-1} \\
  & = 5+2(5+2a_{n-2}) = 5 + 5(2) + 2^2a_{n-2} \\
  & =  5 + 5(2) + 2^2(5+2a_{n-3})= 5+5(2)+5(2^2)+ 2^3a_{n-3}.
 \end{align*}
 
 
 If this substitution is continued, eventually we reach an expression we can compute in closed
 form: \marginnote{In the next to last step we use the formula for adding the terms of a geometric sequence.}
 \begin{align*}
  a_n & =  5+5(2)+5(2^2)+ 5(2^3) + \cdots +5(2^{n-1})+ 2^na_0 \\
  & = 5(1+2+2^2+\cdots+2^{n-1}) + 2^n(2) \\
  & = 5{{2^n-1}\over{2-1}} + 2(2^n) \\
  & =5(2^n-1) + 2(2^n) \\
  & = 7(2^n) -5.
 \end{align*}

\end{exmp}


\clearpage
\section{Exercises}

\begin{exer}
Guess the solution to $a_0=2$, and $a_1=4$, and, for $n\geq2$, $a_n = 4a_{n-1}-3a_{n-2}$
and prove your guess is correct by induction.
\end{exer}

\begin{exer}
Solve by unfolding: $a_0 = 2$, and, for $n\geq 1$, $a_n = 5a_{n-1}$.
\end{exer}

\begin{exer}
Solve by unfolding:  $a_0 = 2$, and, for $n\geq 1$,  $a_n = 5a_{n-1}+ 3$.\\
Hint: This one will involve applying the  geometric sum formula.
\end{exer}

\section{Problems}

\begin{prob}
Guess the solution to $a_0=1$, and $a_1=5$, and, for $n\geq2$, $a_n = a_{n-1}+2a_{n-2}$
and prove your guess is correct by induction.
\end{prob}

\begin{prob}
Solve by unfolding: $a_0 = 2$, and, for $n\geq 1$, $a_n = 7a_{n-1}$.
\end{prob}

\begin{prob}
Solve by unfolding:  $a_0 = 2$, and, for $n\geq 1$,  $a_n = 7a_{n-1}+ 3$.
\end{prob}

