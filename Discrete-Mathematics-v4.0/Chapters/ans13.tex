    \section*{Chapter 13}
\begin{Solution}{13.1}
\marginnote{ A great source of information about numerical sequences is {\itshape The Online Encyclopedia of Integer Sequences} at \url{http://oeis.org}. That site finds 183 sequences that either begin or are otherwise related to $1,2,4,5,7,8$.}
One possible answer: It looks like the list of positive integers in order, skipping the multiples of $3$. So the next few terms will be $10, 11, 13, 14, 16, 17$.

Another answer: The positive integers $n$ (in increasing order) for which $2n+3$ is a prime. The next few terms
would be $10, 13, 14, 17, 19, 20$.
\end{Solution}

\begin{Solution}{13.2}
$a_{100}= 2+6(99) = 596$
\end{Solution}

\begin{Solution}{13.3}
Let $a$ be the initial term and $d$ be the common difference. We have the system $a + 9d = -4$ and $a+15d = 47$. Subtracting the first from the second gives $6d = 51$, so $d = \frac{51}{6} = \frac{17}{2}$, and then $a = 
-4 - 9d = -4 - 9\frac{17}{2} = -\frac{161}{2}$. That means the $11^{th}$ term is $-\frac{161}{2} + 10\left(\frac{17}{2}\right) = \frac{9}{2}$.
\end{Solution}

\begin{Solution}{13.4}
$a_5 = 6(2^4) = 6(16) = 96$
\end{Solution}

\begin{Solution}{13.5}
If $r$ is the common ratio, then $-11 = g_2 = rg_1 = 5r$, and so $r = -\frac{11}{5}$. That means
$g_5 = g_1 r^4 = 5\left (-\frac{11}{5}\right)^4 = \frac{14641}{125}$.
\end{Solution}

\begin{Solution}{13.6} Suppose we have a sequence with initial term $a$ that is arithmetic (with common difference $d$) as well as geometric (with common ration $r$). For the two terms following the initial term, we have $a+d = ar$ and $a+2d = ar^2$. The first equation tells us that $d = a(r-1)$. Subtracting the first equation from the second gives $d = ar^2-ar = ar(r-1)$. So $a(r-1) = ar(r-1)$, and that equation leaves only a few options
for $a$ and $r$. It could be that $a=0$, and that means (1)  $d=0$ and the sequence is $0,0,0,0\ldots$, or (2)  $r-1 = 0$, so $r=1$, and that means $d= 0$, and the sequence is $a,a,a,a,\ldots$, or (3) neither $a=0$ nor $r=1$
in which case $a(r-1) = ar(r-1)$ reduces to $r=1$ which can't be in this case.
Conclusion: the constant sequences $a,a,a,a,\ldots$  are the only sequences that are both arithmetic (initial term $a$ and common difference $0$) and geometric (initial term $a$ and common ratio $1$).
\end{Solution}

\begin{Solution}{13.7}
$\sum_{j=1}^{4} (j^2+1) = 2+5+10+17 = 34$
\end{Solution}

\begin{Solution}{13.8}
$\sum_{k=-2}^{4} (2k-3) = -7 -5 -3 -1+1+3+5 = -7$ 
\end{Solution}

\begin{Solution}{13.9}
The initial term is $2$, the $100^{th}$ term is $2 +(99)(6)$, so the total is $100\left(\frac{2 + 2+(99)(6)}{2}\right)
= 29900$.
\end{Solution}

\begin{Solution}{13.10}
$6 + 12 + 24 + 48 + 96 = 186$
\end{Solution}

\begin{Solution}{13.11}
$1 - \frac{3}{2} +\frac{9}{4} - \frac{27}{8} + \frac{81}{16} = \frac{211}{16}$
\end{Solution}

\begin{Solution}{13.12}
$\displaystyle \sum_{k =1}^n\, \frac{1}{2k}$
\end{Solution}
