\chapter{Properties of Relations}

\newthought{There are several conditions} that can be imposed on a relation $R$ on a set  
$A$ that make it 
useful.  These requirements distinguish those relations which are interesting 
for some reason from
the garden variety junk, which is, let's face it,  what most relations are. 

\section{Reflexive} \marginnote{Reflexive: $(\forall a\in A)[aRa]$}
A relation $R$ on $A$ is {\bfseries reflexive} provided $\forall a\in A, aRa.$
In plain English, a relation is reflexive if every element of its domain is related to
itself. The relation $B(x,y) : x \textbf{ is the brother of } y$ is not 
reflexive since no person is his own brother. On the other hand, the relation
$S(m,n) : m+n \textbf{ is even}$.  is a reflexive relation on the set of integers
since, for any integer $m$, $m+m=2m$ is even.

It is easy to spot a reflexive relation from its digraph:  there is a loop at every vertex.
Also, a reflexive relation can be spotted quickly from its matrix. First, let's agree
that when the matrix of a relation on a set $A$ is written down, the same ordering
of the elements of $A$ is used for both the row and column designators.  For a 
reflexive relation, the entries on the {\bfseries main diagonal} of its matrix
will all be $1$'s. The main diagonal of a square matrix runs from the upper
left corner to the lower right corner.

\section{Irreflexive} \marginnote{Irreflexive: $(\forall a\in A)[a\cancel{R}a]$\\ \qquad\  $\equiv \lnot(\exists a\in A)[aRa]$}
The flip side of the coin from reflexive is irreflexive.
A relation $R$ on $A$ is {\bfseries irreflexive} in case $a{\cancel{R}}a$ for all $a\in A$.
In other words, no element of $A$ is related to itself. The {\itshape brother of} relation
is irreflexive. 
The digraph of an irreflexive relation  contains no loops, and its matrix has all $0$'s 
on the main diagonal.

Actually, that discussion was a little careless. To see why, consider the relation
$S(x,y) : \textbf{ the square of } x \text{ is bigger than or equal to } y$.
Is this relation reflexive? The answer is: we can't tell. The answer depends on the 
domain of the relation, and we haven't been told what that is to be. For example, if the
domain is the set $\N$ of natural numbers, then the relation is reflexive, since
$n^2\geq n$ for all $n\in \N$. However, if the domain is  the set $\R$ of all
real numbers, the relation is not reflexive. In fact, a counterexample to the claim
that $S$ is reflexive on $\R$ is the number $\frac{1}{2}$ since
 $\left(\frac{1}{2}\right)^2 = \frac{1}{4}$, and
$\frac{1}{4} < \frac{1}{}$, so $\frac{1}{2}{\cancel{S}}\frac{1}{2}$. The lesson to be learned
from this example is that the question of whether a relation is reflexive cannot be
answered until the domain has been specified.  The same is true for the irreflexive condition
and the other conditions defined below. Always be sure you know the domain before
trying to determine which properties a relation satisfies. 


\section{Symmetric} \marginnote{Symmetric: $(\forall a,b\in A)[aRb \rightarrow bRa]$}
A relation $R$ on $A$ is {\bfseries symmetric} provided $(a,b)\in R\to (b,a)\in R$. 
Another way to say the
same thing: $R$ is symmetric provided $R=R^{-1}$. In words, $R$ is symmetric provided that
whenever $a$ is related to $b$, then $b$ is related to $a$. 
Any digraph representing a symmetric relation $R$ will have a return edge for every non-loop.
Think of this as saying the graph has no one-way streets.
The matrix $M$ of a symmetric relation satisfies $M=M^T$.
In this case $M$ is symmetric about its main diagonal in the usual geometric sense of symmetry.
The $B(x,y): x \textbf{ is the brother of } y$ relation mentioned before is not
symmetric if the domain is taken to be all people since, for example, $Donny\,B\, Marie$,
but $Marie\,{\cancel{B}}\,Donny$. On the other hand, if we take the domain to be all (human)
males, then $B$ is symmetric. 

\section{Antisymmetric} \marginnote{Antisymmetric:\[ (\forall a,b\in A)[(aRb)\land(bRa) \rightarrow (a=b)] \] }
A relation $R$ on $A$ is {\bfseries antisymmetric} if whenever $(a,b)\in R$ and $(b,a)\in R$,
then $a=b$.  In other words,  the only objects that are each related to the other
are objects that are the same. For example, the usual $\leq$ relation for the 
integers is antisymmetric since if $m\leq n$ and $n\leq m$, then $n=m$.  A digraph 
representing an
 antisymmetric relation will have all streets one-way except loops.
If $M$ is a matrix for $R$, then whenever $a_{i,j}=1$ and $i\neq j$, $a_{j,i}=0$.

\section{Transitive} \marginnote{Transitive: \[ (\forall a,b,c \in A)[(aRb)\land(bRc) \rightarrow (aRc)] \]}
A relation $R$ on $A$ is {\bfseries transitive} if  whenever $(a,b)\in R$ and $(b,c)\in R$, then
$(a,c)\in R$. This can also be expressed by saying $R\circ R\subseteq R$. 
In a digraph for a transitive relation whenever we have a directed path of
 length two
from $a$ to $c$ through $b$, we must also have a direct link from $a$ to $c$. This means
 that any
digraph of a transitive relation has lots of triangles. This includes degenerate triangles
 where
$a, b$ and $c$ are not distinct. A matrix $M$ of a transitive relation satisfies 
$M\odot M\leq M$. 
The relation $\leq$ on $\N$ is transitive, since from $k\leq m$ and $m\leq n$, we can 
conclude $k\leq n$. 

\section{Examples}
\begin{exmp}
 \item Define a relation, $N$ on the set of all living people by the rule
 $a\,N\,b$ if and only if $a, b$ live within one mile of each other.
 This relation is reflexive since every person lives within a mile of himself.
 It is not irreflexive since I live within a mile of myself.
 It is  symmetric since if $a$ lives within a mile of $b$, then $b$ lives within
 a mile of $a$. It is not antisymmetric since Mr{.} and Mrs{.} Smith live within
 a mile of each other, but they are not the same person. It is not transitive:
 to see why, think of the following situation (which surely exists somewhere 
 in the world!): there is a straight road of length $1.5$ miles. Say Al lives
 at one end of the road, Cal lives at the other end, and Sal lives half way between
 Al and Cal. Then Al$\,N\,$Sal and Sal$\,N\,$Cal, but not Al$\,N\,$Cal. 
\end{exmp}
 
\begin{exmp}
 Let $A=\R$ and define $aRb$ iff $a\leq b$, then $R$ is a reflexive, 
 transitive, antisymmetric relation. Because of this example, any relation
 on a set that is reflexive, antisymmetic, and transitive is called an 
 {\bfseries ordering} relation. The subset relation on any collection of sets
 is another ordering relation.
\end{exmp}
 
\begin{exmp} Let $A=\R$ and define $aRb$ iff $a<b$. Then $R$ is irreflexive,
 and transitive.
\end{exmp}
 
\begin{exmp} If $A=\{1,2,3,4,5,6\}$ then 
\begin{align*}
 R=\{&(1,1),(2,2),(3,3),(4,4),(5,5),(6,6),(1,3),(3,1),(1,5), \\
     &(5,1),(2,4),(4,2),(2,6),(6,2),(3,5),(5,3),(4,6),(6,4)\}
\end{align*}
 is reflexive, symmetric, and transitive. In artificial examples such as this one,
 it can be a tedious chore checking that the relation is transitive.
\end{exmp}
 
\begin{exmp} If $A=\{1,2,3,4\}$ and $R=\{(1,1),(1,2),(2,3),(1,3),(3,4),(2,4),(4,1)\}$
  then $R$ is not 
 reflexive, not irreflexive, not symmetric, and not transitive  but it is antisymmetric.
\end{exmp}

\clearpage




\section{Exercises}

\begin{exer}
 Define a relation on $\{1,2,3\}$ which is both symmetric and antisymmetric.
\end{exer}

\begin{exer} 
Define a relation on $\{1,2,3,4\}$ by
 \[
  R=\{(1,2),(2,1),(2,3),(3,2),(3,4),(4,3)\}.
 \]
For each of the five properties of a relation defined in this chapter (reflexive, irreflexive, symmetric, antisymmetric, and transitive) 
either show $R$ satisfies the property, or explain why it does not.
\end{exer}

\begin{exer}
Each matrix below specifies a relation $R$ on $\{1,2,3,4,5,6\}$ with respect to the
 given ordering  $1,2,3,4,5,6$. 
 
 For each of the five properties of a relation defined in this chapter (reflexive, irreflexive, symmetric, antisymmetric, and transitive) 
either show $R$ satisfies the property, or explain why it does not.

\vspace*{0.25cm}
\begin{enumerate*}[label=\alph*), itemjoin=\qquad]
\item 
 $\left[
\begin{matrix}
 1&1&1&0&0&0\\ 
 1&1&1&0&0&0\\ 
 1&1&1&0&0&0\\
 0&0&0&1&1&1\\ 
 0&0&0&1&1&1\\ 
 0&0&0&1&1&1
\end{matrix}
\right]$ 

\item 
$\left[
\begin{matrix}
 1&1&1&1&1&1\\
 0&0&0&0&0&1\\ 
 0&0&0&0&0&1\\ 
 0&0&0&0&0&1\\ 
 0&0&0&0&0&1\\ 
 0&0&0&0&0&1
\end{matrix}
\right]$
\end{enumerate*}

\vspace*{0.25cm}
\begin{enumerate*}[label=\alph*), itemjoin=\qquad]
\item[c)] 
$\left[
\begin{matrix}
 1&0&0&0&0&1\\ 
 0&1&0&0&1&0\\ 
 0&0&1&1&0&0\\ 
 0&0&1&1&0&0\\ 
 0&1&0&0&1&0\\ 
 1&0&0&0&0&0
\end{matrix}
\right]$

\item[d)]
$\left[
\begin{matrix}
 1&0&0&0&0&1\\ 
  0&1&0&0&1&0\\ 
  0&0&1&1&0&0\\ 
  0&0&1&1&0&0\\ 
  0&1&0&0&1&0\\ 
  1&0&0&0&0&1
\end{matrix}
\right]$
\end{enumerate*}
\end{exer}

\begin{exer}
Define the relation {\itshape $C(A,B) : |A|\leq|B|$}, where the
domains for $A$ and $B$ are all subsets of $\Z$. 

For each of the five properties of a relation defined in this chapter (reflexive, irreflexive, symmetric, antisymmetric, and transitive) 
either show $S$ satisfies the property, or explain why it does not.
\end{exer}

\begin{exer} 
Explain why $\emptyset$ is a relation on any set.
\end{exer}

\begin{exer}
Define the relation {\itshape $M(A,B) : |A\cap B| = 1$ (or, in plain English, $A$ and $B$ have exactly one element in common)}, where the
domains for $A$ and $B$ are all subsets of $\Z$. A few examples: 

\begin{itemize}

\item $\{5,10\}\,M\,\{ 1,2,3,4,5,6\}$ is true since the sets $\{5,10\}$
and $\{1,2,3,4,5,6\}$ have exactly one element in common (namely $5$). 

\item $\{ 1,2,3\}\,M\, \{6,7,8,9\}$ is false since  $\{ 1,2,3\}$ and $\{6,7,8,9\}$ have no elements in common.

\item $\{1,2,3,4\}\,M\, \{2,4,6,8\}$ is false since $\{1,2,3,4\}$ and $\{2,4,6,8\}$ have more than one element in common.

\item $\{n | n \in Z \text{ and } n\leq 0\} \, M\, \{n | n \in Z \text{ and } n\geq 0\}$ is true since $\{n | n \in Z \text{ and } n\leq 0\}$
and  $\{n | n \in Z \text{ and } n\geq 0\}$ have exactly one element in common (namely $0$).

\end{itemize}


For each of the five properties of a relation defined in this chapter (reflexive, irreflexive, symmetric, antisymmetric, and transitive) 
either show $M$ satisfies the property, or explain why it does not.
\end{exer}

\clearpage

\section{Problems}

\begin{prob}
Let $R$  be the relation  $\{(1,1)\}$ on the set $A = \{ 1,2\}$. For each of the five properties of a relation defined in this chapter (reflexive, irreflexive, symmetric, antisymmetric, and transitive) 
either show $S$ satisfies the property, or explain why it does not.
\end{prob}

\begin{prob}
Let $R$  be the relation  $\{(1,1), (1,2), (1,3), (2,3)\}$ on the set $A = \{ 1,2,3\}$. For each of the five properties of a relation defined in this chapter (reflexive, irreflexive, symmetric, antisymmetric, and transitive) 
either show $S$ satisfies the property, or explain why it does not.
\end{prob}

\begin{prob}
Let $A$ be the relation on the set $\Z$ of all integer defined by $s\,A\,t$ if and only if $|s|\leq |t|$.  For each of the five properties of a relation defined in this chapter (reflexive, irreflexive, symmetric, antisymmetric, and transitive) either show $S$ satisfies the property, or explain why it does not.
\end{prob}


\begin{prob}
Let $D$ be the relation on the natural numbers defined by the rule $mDn$ if and only if $m$ does not equal $n$. Examples: $5D7$ is true and $4D4$
is false. For each of the five properties of a relation defined in this chapter (reflexive, irreflexive, symmetric, antisymmetric, and transitive) 
either show $S$ satisfies the property, or explain why it does not.
\end{prob}

\begin{prob}
Let $R$ be the relation  $\{(1,2), (2,3), (3,4)\}$ on the set $A=\{1,2,3\}$. The relation $R$ is not transitive on $A$. What is the fewest number of ordered 
pairs that need to be added to $R$ so it becomes a transitive relation on $A$?
\end{prob}

\begin{prob}
Give a counterexample to the claim that a relation $R$ on a set $A$ that is both symmetric and transitive must be reflexive. Hint: There is a very simple example!
\end{prob}

\begin{prob} 
Define the relation {\itshape $M(A,B) : A\cap B = \emptyset$}, where the
domains for $A$ and $B$ are all subsets of $\Z$. 

For each of the five properties of a relation defined in this chapter (reflexive, irreflexive, symmetric, antisymmetric, and transitive) 
either show $M$ satisfies the property, or explain why it does not.
\end{prob}


\begin{prob}
\ %
\begin{enumerate}[label=(\alph*)]
 \item  Let $A=\{1\}$, and consider the empty relation, $\emptyset$,  on $A$. 
 For each of the five properties of a relation defined in this chapter (reflexive, irreflexive, symmetric, antisymmetric, and transitive) 
either show $\emptyset$ satisfies the property, or explain why it does not.

 \item Same question as (a), but now with $A=\emptyset$.
\end{enumerate}
\end{prob}
