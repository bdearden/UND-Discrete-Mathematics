    \section*{Chapter 16}
\begin{Solution}{16.1}
{\bfseries basis:} For $n=1$, the left side is $1\cdot 3 = 3$, and the right side is
$\frac{1\cdot2\cdot9}{6} =3$, so the equality is correct for this case. \\
{\bfseries inductive hypothesis:} Suppose 
\[1\cdot3 + 2\cdot4 + 3\cdot 5 + \cdots + n(n+2) = \frac{n(n+1)(2n+7)}{6}\]
for some $n\geq 1$. Then\\
{\bfseries inductive step:} 
\begin{align*}
1\cdot3 +& 2\cdot4 + 3\cdot 5 + \cdots + n(n+2) + (n+1)(n+3) \\
&= \frac{n(n+1)(2n+7)}{6} + (n+1)(n+3) \qquad\quad\text{using the inductive hypothesis}\\
&= \frac{n(n+1)(2n+7)}{6} + \frac{6(n+1)(n+3)}{6} \\
&= \frac{n(n+1)(2n+7)+ 6(n+1)(n+3)}{6}\qquad\qquad\text{now factor out the common } (n+1)\\
&= \frac{(n+1)[n(2n+7)+ 6(n+3)]}{6}\qquad\qquad \text{and the rest is just algebra}\\
&= \frac{(n+1)(2n^2+7n+ 6n+18)}{6}\\
&= \frac{(n+1)(2n^2+13n+18)}{6}\\
&= \frac{(n+1)(n+2)(2n+9)}{6}\\
&= \frac{(n+1)(n+2)(2(n+1)+7)}{6},
\end{align*}
as we needed to show. $\clubsuit$
\end{Solution}

\begin{Solution}{16.2}
{\bfseries basis:} For $n=1$, $1\cdot 2^1= 2$ and $(1-1)2^{1+1} +2 = 2$, so the equality is correct when $n=1$.\\
 {\bfseries inductive hypothesis:} Suppose 
 \[
 1\cdot 2^1+2\cdot 2^2+3\cdot 2^3+...+n\cdot 2^n=(n-1)2^{n+1}+2, 
\]
for some $n\geq 1$, Then\\
{\bfseries inductive step:} 
\begin{align*}
1\cdot 2^1&+2\cdot 2^2+3\cdot 2^3+...+n\cdot 2^n + (n+1)2^{n+1}\\
&= (n-1)2^{n+1}+2 +(n+1)2^{n+1} \qquad\quad\text{using the inductive hypothesis}\\
&= ((n-1)+(n+1))2^{n+1} + 2 \qquad\qquad \text{and the rest is just algebra}\\
&= (2n)2^{n+1} + 2\\
&= n2^{n+2} + 2\\
&= ((n-1)+1)2^{(n+1)+1} + 2,
\end{align*}
as we needed to show. $\clubsuit$
\end{Solution}

\begin{Solution}{16.3}
{\bfseries basis:} $f_{0} = 0 = 1-1 = f_{2}-1$, so the equation is correct for $n=1$.\\
{\bfseries inductive hypothesis:} Suppose 
\[f_{0}+ f_{1} + f_{2}+\cdots+f_{n} = f_{n+2} - 1\] for some $n\geq 0$. Then\\
{\bfseries inductive step:} 
\begin{align*}
f_{0}&+ f_{1} + f_{2}+\cdots+f_{n} + f_{n+1} \\
&= (f_{n+2} - 1)+f_{n+1}\qquad\quad\text{using the inductive hypothesis}\\
&= (f_{n+1} + f_{n+2} - 1)\\
&= f_{n+3} -1 \qquad\qquad\text{using the recursive definition of the Fibonacci sequence}\\
&= f_{(n+1)+2} - 1,
\end{align*}
as we needed to show.
\end{Solution}


\begin{Solution}{16.4}
{\bfseries basis:} For $n=5$, the inequality is correct: $2^5 = 32 > 25 = 5^2$.\\
{\bfseries inductive hypothesis:} Suppose $2^n > n^2$ for some $n>4$. Then\\
{\bfseries inductive step:} 
\begin{align*}
2^{n+1} & =2 (2^n) >2n^2\qquad\quad\text{using the inductive hypothesis} \\
\text{ now }& 2n^{2}= n^2 + n^2 = n^2 + (n)(n)> n^2 + 3n\qquad\quad\text{since  for } n>4 \text { it is true that } (n)(n) > 3n \\
\text{next }&n^{2}+3n= n^2 +2n + n > n^2 + 2n + 1\qquad\quad\text{since for } n>4  \text { it is true that } n >1\\
\text{and }& n^{2}+ 2n + 1= (n+1)^2,
\end{align*}
Putting the pieces together, we get $2^{n+1}> (n+1)^{2}$ as we needed to show.
\end{Solution}

\begin{Solution}{16.5}
{\bfseries basis:} For $n=0$, $11^n - 6 = 1-6 = -5 = (5)(-1)$, so $11^0-6$ is divisible by $5$.\\
{\bfseries inductive hypothesis:} Suppose $11^n-6$ is divisible by $5$ for some $n\geq 0$. Then\\
{\bfseries inductive step:} 
\[
11^{n+1} - 6 = 11(11^n) - 6 =10(11^n) +(11^n-6) = 2(5)(11^n) + (11^n -6).
\]
In that last expression, the term $2(5)(11^n)$ is certainly divisible by $5$, and the expression  $11^n-6$ is 
divisible by $5$ by the inductive hypothesis. That implies $2(5)(11^n) + (11^n -6)$ is divisible by $5$ as we needed to show.\\
{\bfseries Note:} Here is another way to see that $11^n-6$ is divisible by $5$: for any integer $n\geq 0$, the number
$11^n$ will have units digit $1$, and so $11^n - 6$ will have units digit $5$. Numbers with units digit $5$ are divisible by $5$. This proof does not answer the question posed however, since it is not a proof by induction.
\end{Solution}

\begin{Solution}{16.6}
{\bfseries basis:} With $0$ cuts, we end up with one piece of pizza, namely the whole thing. And, sure enough,
for $n=0$, $\frac{n^2+n+2}{2} = \frac{0^2 +0 + 2}{2} = 1$.\\
{\bfseries inductive hypothesis:} Suppose that  for some $n\geq 0$, $n$ straight lines cut produces a maximum
number of $\frac{n^2+n+2}{2}$ pieces. Then\\
{\bfseries inductive step:} Suppose we add one more cut. Notice that when the new cut crosses an old cut, it will slice one old piece into two new piece. We can't get more that two pieces when one cut crosses another since two straight lines cannot cross each other more than once. So, to get the maximum number of new pieces,
we should make the new cut not parallel to the $n$ previous cuts (and so, with care, be sure to cross all the
previous cuts, and not at a point where previous cuts cross each other). This will give the maximum number
of new pieces equal to $n+1$. Conclusion: the maximum number of pieces with $n+1$ straight cuts is
\[
\frac{n^2+n+2}{2} + (n+1) = \frac{n^2+n+2 +2(n+1)}{2} = \frac{(n+1)^2 +(n+1)+2}{2},
\]
as we needed to show.\\
The list of maximums begins $1,2,4,7,11,16,22,29,37,46,56,67,79, 92$.
\end{Solution}

\begin{Solution}{16.7}
{\bfseries basis:} For $n=0$, we re given $a_0= 0$ and we see $\frac{5^0-1}{4} = 0$, so the basis step is good.\\
{\bfseries inductive hypothesis:} Suppose $a_n = \frac{5^n-1}{4}$ for some $n\geq 0$. Then\\
{\bfseries inductive step:} 
\begin{align*}
a_{n+1} &= 5a_n + 1\qquad\qquad\text{using the recursive definition of the sequence}\\
 &= 5\left(\frac{5^n-1}{4}\right) + 1\qquad\qquad\text{using the inductive hypothesis}\\
 &= \frac{5^{n+1} - 5}{4} + \frac{4}{4}\qquad\qquad \text{and the rest is just algebra}\\
 &=\frac{5^{n+1} - 5+4}{4}\\
 &= \frac{5^{n+1} - 1}{4},
 \end{align*}
 as we needed to show.
\end{Solution}

\begin{Solution}{16.8}
{\bfseries basis:} Since the recursive formula involves the two previous terms, we are going to have to check the closed form formula for the two terms $a_0$ and $a_1$. But they both work ok since $a_0= 1$ and $2(3^0) - 2^0
= 2(1) - 1 = 1$ and $a_1 = 4$ and $2(3^1) - 2^1 = 2(3) - 2 = 4$.\\
{\bfseries inductive hypothesis:} For $n\geq 2$, $a_n$ depends on the two previous terms in the sequence, so it would be wise to use the second form of induction this time. So, let's suppose that $a_k = 2\cdot 3^k - 2^k$ for all $k$ form $0$ to some $n\geq 1$. Then\\
{\bfseries inductive step:} 
\begin{align*}
a_{n+1} &= 5a_{n} -6a_{n-1}\qquad\qquad\text{using the recursive definition, okay since } n+1\geq2\\
&=5(2\cdot3^{n}-2^{n}) - 6(2\cdot3^{n-1} -2^{n-1}) \qquad\qquad\text{using the inductive hypothesis}\\
&= (5\cdot2\cdot3 - 6\cdot2)3^{n-1} -(5\cdot2-6)2^{n-1}\qquad\qquad \text{and the rest is just algebra}\\
&=18\cdot3^{n-1} - 4\cdot2^{n-1}\\
&= 2\cdot3^2\cdot3^{n-1} - 2^2\cdot2^{n-1}\\
&= 2\cdot3^{n+1} - 2^{n+1}.
\end{align*}
as we needed to show.
\end{Solution}
