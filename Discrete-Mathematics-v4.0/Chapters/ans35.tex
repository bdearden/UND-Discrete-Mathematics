   \section*{Chapter 35}
  
    
\begin{Solution}{35.1}

Using the given recursive formula we see the sequence begins $2, 4, 10,  28, 82, 244$, and those values look a
lot like powers of $3$: $1, 3, 8, 27, 81, 243$, so it looks like a reasonable guess is $a_n = 3^n + 1$.

The basis for the induction is the case $n=0$. We are given $a_0 = 2$ while $3^0+1 = 1+1 = 2$,  and
$a_1 = 4$ while $3^1 + 1 = 4$. So $a_n = 3^n+1$ is correct for $n=1,2$
works out okay. For the inductive step, suppose $a_k = 3^k + 1$ for all values of $k\leq n$ for some $n\geq 1$. Then 
\begin{gather*}
a_{n+1} = 4a_{n} - 3a_{n-1} = 4(3^{n}+1) -3(3^{n-1} + 1) \\
\qquad = 4\cdot3^{n} - 3^{n} + 4 - 3 = 3\cdot3^{n} + 1 = 3^{n+1} + 1.
\end{gather*}
$\clubsuit$

\end{Solution}

\begin{Solution}{35.2}

\[ 
a_n = 5a_{n-1} = 5(5a_{n-2}) = 5^2a_{n-2} = 5^2(5a_{n-3}) = 5^3a_{n-3}.
\]
As we continue to unfold, eventually we will reach $a_0$. Notice that the exponent on the $5$ and the subscript on the $a$ always add up to $n$. That makes sense since at each step the exponent goes up $1$ and the subscript goes down $1$, and so the exponent and the subscript always add up to the $n$ they started at in the first step.
That means we eventually reach $a_n = 5^na_0$. Since $a_0 = 2$, we conclude $a_n = 2\cdot 5^n$.

\end{Solution}


\begin{Solution}{35.3}

\begin{gather*}
a_n = 3 + 5a_{n-1} = 3+ 5(3+ 5a_{n-2}) = (3 + 3\cdot5) + 5^2a_{n-2}\\
\qquad = (3 + 3\cdot5) + 5^2a_{n-2} = (3 + 3\cdot5) + 5^2(3+a_{n-3})\\
\qquad = (3+3\cdot 5 +3\cdot 5^2) + 5^3a_{n-3}.\\
\end{gather*}
As we continue to unfold, the first group in parentheses will continue gaining one term at each step and the
last term will have the exponent of the $5$ going up one at a times while the subscript on the $a$ will decease by one at a time. Eventually we will reach the expression
\begin{gather*}
a_n = (3+ 3\cdot5 +3\cdot 5^2+ \cdots+3\cdot5^{n-1}) + 5^na_0\\[3pt]
\qquad = 3(1 + 5 + 5^2\cdots+5^{n-1}) + 5^n\cdot2\\[3pt]
\qquad = 3\frac{5^n-1}{5-1} + 2\cdot5^n\\[3pt]
\qquad = \frac{11\cdot 5^n - 3}{4}.
\end{gather*}

As a check of our work, we can use the recursive formula and the closed form formula to generate six or so terms to see if they produce the same values (or, if we are really ambitious, we can use induction to verify the closed form formula is correct). In any case, the recursive formula and the closed form formula both give

\[
2, 13, 68, 343, 1718, 8593
\]
for the first six terms, and so we can be reasonably confident our work is okay.

\end{Solution}

