\chapter{The Pigeonhole Principle}\label{ch:The Pigeonhole Principle}

\newthought{The pigeonhole principle,} like the sum and product rules, is another one of
those absolutely 
obvious counting facts.
 The statement is simple: If $n+1$ objects are divided into
$n$ piles (some piles can be empty), then at least one pile must have two or
more objects in it. Or, more colorfully, if $n+1$ pigeons land in $n$ pigeonholes,
then at least one pigeonhole has two or more pigeons. What could be more obvious?
The pigeonhole 
principle is used to show that no matter how a certain task is carried
out, some specific result must always happen. 

As a simple example, suppose we have a drawer containing ten 
identical black socks and ten identical white socks. 
How many socks do we need to select to be sure we have a matching pair? The
answer is three. Think of the pigeonholes as the colors black and white, and
as each sock is selected put it in the pigeonhole of its color. After we have
placed the third sock, one of the two pigeonholes must have at least two
socks in it, and we will have a matching pair. Of course, we may have been
lucky and had a pair after picking the second sock, but the pigeonhole principle
\emph{guarantees} that with the third sock we will have a pair.


As another example, suppose license plates are made consisting
of four digits followed by two letters. Are there enough license plates for a state
with seven million cars? No, since there are only $10^4\cdot26^2 = 6760000$ 
possible license plates, and so, by the pigeonhole principle, at least two of
the seven million plates assigned would have to be the same.

\section{General pigeonhole principle}
A slightly fancier version of the pigeonhole principle says that if $N$
objects are distributed in $k$ piles, then there must be a least one pile with
$\left\lceil\frac{N}{k}\right\rceil$ objects in it. 

That formula looks impressive,
 but actually is easy to understand. For example, if there are $52$ people
in a room, we can be absolutely certain that there are at least eight born
on the same day of the week. Think of it this way: with $49$ people, it would be
possible to have seven born on each of the seven days of the week. But when the
\marginnote{Avoidance  principle: how long can we go before our hand is forced?}
$50^{th}$ one is reached, it must boost one day up to an eighth person. That is
really about all there is to it. The general proof of the fancy pigeonhole
principle uses this same sort of reasoning. It is a proof by contradiction, 
and goes as follows:

\begin{thm}[Pigeonhole Principle] If $N$
objects are distributed in $k$ piles, then there must be a least one pile with
$\left\lceil{N\over k}\right\rceil$ objects in it.
\end{thm}
\begin{proof}
Suppose we have $N$ objects distributed in $k$ piles, and suppose that every
pile has fewer than $\left\lceil{N\over k}\right\rceil$ objects in it. 
That means that the piles each contain $\left\lceil{N\over k}\right\rceil-1$ or fewer
objects. We will use the fact that $\left\lceil{N\over k}\right\rceil< {N\over k}+1$
to complete the proof. 
The
total number of object will be at most 
$k\left(\left\lceil{N\over k}\right\rceil-1\right)< k\left(\left({N\over k}+1\right)-1\right)
= N$. That is a contradiction since we know there is a total of $N$ objects in the
$k$ piles.
\end{proof}


\section{Examples}
Even though the pigeonhole principle sounds very simple, clever applications
of it can produce totally unexpected results. 

\begin{exmp}
 Five misanthropes move to a perfectly square deserted island that
 measures two kilometers on a side. Of course, being misanthropes, they want to live 
 as far from each other as possible. Show that, no matter where they build on the
 island, some two will be no more than $\sqrt{2}$ kilometers of each other.
\end{exmp}
\begin{soln}
 Divide the island into four one kilometer by one kilometer
 squares by drawing lines joining the midpoints of opposite sides. Since there
 are five people and four squares, the pigeonhole principle guarantees there
 will be two people living in one of those four squares. But people in one of
 those squares cannot be further apart than the length of the diagonal of the
 square which is, according to Pythagoras, $\sqrt{2}$.\;\qed
\end{soln} 

\begin{exmp}
  For any positive integer $n$, there is a positive multiple of $n$
 made up of a number of $1$'s followed by a number of $0$'s. For example, for $n=1084$,
 we see
 $1084\cdot 1025 = 1111100$.
\end{exmp}
\begin{soln}
 Consider the $n+1$ integers $1$, $11$, $111$, $\cdots$, 
 $11\cdots1$, where the last one consists of $1$ repeated  $n+1$ times. Some two of these
 must be the same modulo $n$, and so $n$ will divide the difference of some two
 of them. But the difference of two of those numbers is of the required type.\;\qed
\end{soln}

\begin{exmp}
 Bill has $20$ days to prepare his tiddledywinks title defense. 
 He has decided to practice at least one hour every day.  But, to avoid burn-out,
 he  will not practice more than a total of $30$ hours. Show there is a  sequence
 of consecutive days during which he practices exactly $9$ hours. 
\end{exmp}
\begin{soln}
 For $j = 1,2,\cdots 20$, let $t_j = $ the total number of hours
 Bill practices up to and including day $j$.
 Since he practices at least one hour every day, and the total number of hours 
 is no more than $30$, we see
 \[
 0<t_1<t_2<\cdots<t_{20}\leq 30.
 \]
 Adding $9$ to each term we get
 \[
 9<t_1+9<t_2+9<\cdots<t_{20}+9\leq 39.
 \]
 
 So we have $40$ integers $t_1,t_2,\cdots, t_{20}, t_1+9,\cdots t_{20}+9$,
 all between $1$ and $39$. By the pigeonhole principle, some two must be
 equal, and the only way that can happen is for $t_i= t_j+9$ for some 
 $i$ and $j$. It follows that $t_i-t_j = 9$, and since the difference
 $t_i-t_j$ is the the total number of hours Bill practiced
 from day $j+1$ to day $i$, that shows there is a sequence of consecutive
 days during which he practiced exactly $9$ hours.\;\qed
\end{soln}

\clearpage
\section{Exercises}

\begin{exer}
Show that in any group of eight people, at least two were born on the 
same day of the week.
\end{exer}

\begin{exer}
Show that in any group of $100$ people, at least $15$ were born on the
same day of the week.
\end{exer}

\begin{exer}
How many cards must be selected from a deck to be sure that at least
six of the selected cards have the same suit?
\end{exer}

\begin{exer}
Show that in any set of $n$ integers, where $n\geq2$, there must be a pair
with a difference that is a multiple of $n-1$.
\end{exer}

\begin{exer}
Al has $75$ days to master discrete mathematics. He decides to study
at least one hour every day, but no more than a total of $125$ hours.
Show there must be a sequence of consecutive days during which
he studies exactly $24$ hours.
\end{exer}

\begin{exer}
Show that in any set of $217$ integers, there must be a pair
with a difference that is a multiple of $216$.
\end{exer}

\section{Problems}

\begin{prob}
Show that in a town with population $18,000$, there must be at least two people with the same three initials. 
\end{prob}

\begin{prob}
What is the smallest town population that will guarantee there will be at least two people with the same three initials?
\end{prob}

\begin{prob}
What is the smallest town population that will guarantee there will be at least five people with the same three initials?
\end{prob}

\begin{prob}
How many cards have to be selected from a $52$ card deck to be sure there will be two cards of the same suit?
\end{prob}

\begin{prob}
How many cards have to be selected from a $52$ card deck to be sure there will be two cards of the same rank?
\end{prob}

\begin{prob}
Five misanthropes buy a six mile by eight mile rectangular plot in the arctic. Show that no matter where they build 
their igloos, there will be at least two people that are no more than five miles apart. (You can assume the ice sheet
they buy is perfectly flat.)
\end{prob}

\begin{prob}
In any list of $n$ integers, there will be a  chunk of  consecutive entries from the list that add up to a multiple of $n$.
For example: in the list $-8, 4, 22, -11, 7$, we have $4+22-11= 15$ is a multiple of $5$. 
\end{prob}

\begin{prob}
Suppose $a_{1}, a_{2}, a_{3}\ldots,a_{99}$ is a permutation of $1,2,\ldots,99$. Show that the product
\[
(a_{1}+1)(a_{2}+2)(a_{3}+3) \dots(a_{99}+99)
\]
is even.
\end{prob}

\begin{prob}
In a rematch, Bill has $30$ days to train for a new defense of his tiddledywinks title. He plans to practice at least one hour
every day, but no more than $45$ hours total. Show there is a sequence of consecutive days during which he practices 
exactly $14$ hours. 
\end{prob}

