   \section*{Chapter 28}
    
\begin{Solution}{28.1}

Number of options for one course $ = 5+4+6 = 15$.

\end{Solution}

\begin{Solution}{28.2}

Number of options for a program of three courses, one from each area $ = 5 \cdot 4 \cdot 6 = 120$.

\end{Solution}

\begin{Solution}{28.3}

$26^2\cdot 10^2 + 26^3\cdot10 + 26^{4}$

\end{Solution}

\begin{Solution}{28.4}

$26^6$

\end{Solution}

\begin{Solution}{28.5}

$26\cdot25\cdot24\cdot23\cdot22\cdot21$

\end{Solution}

\begin{Solution}{28.6a}

$2^{25}$ ($2$ choices for each question, T or F.)

\end{Solution}

\begin{Solution}{28.6b}

$3^{25}$ ($3$ choices for each question, T or F or skip.)

\end{Solution}

\begin{Solution}{28.7}

$1 + 2 + 2^2 + \cdots + 2^9 = \frac{2^{10}-1}{2-1} = 1023$.\\
If you don't want to include the empty string (of length $0$) then the
answer is $1022$.

\end{Solution}

\begin{Solution}{28.8}

The total number of words of length $8$ is $26^8$. The number with no A's is $25^8$.
So the number with at least one A is $26^8 - 25^8$.

\end{Solution}

\begin{Solution}{28.9}

The number of seven letter words with no A's is $25^7$. The number of seven letter words with exactly one A is
$7\cdot 25^6$. The $7$ accounts for the number of options for placing the A, and the $25^6$ accounts for the number of ways of filling in the remaining six spots. So, the number of seven letter words with at most one A is
$25^7+ 7\cdot 25^6$. 

\end{Solution}

\begin{Solution}{28.10}

The number of nine letter words with at least two A's is the total number of nine letter words ($26^9$) minus the number with at most one A ($25^9 + 9\cdot 25^8$). So, using the {\it good = total minus bad} rule the number of nine letter words with at least two A's is $26^9 -(25^9 + 9\cdot 25^8)$

\end{Solution}
