\chapter{Equivalence Relations}

\newthought{Relations capture the essence} of many different mathematical concepts. 
 In this chapter, we will show how to put the idea of {\itshape
are the same kind} in terms of a special type of relation.

  Before considering the formal concept of {\itshape same kind} let's look at a few
simple examples. Consider the question, posed about an ordinary deck of $52$
cards: {\itshape How many different kinds of cards are there?} One possible answer is:
{\itshape There are $52$ kinds of cards}, since all the cards are different. But
another possible answer in certain circumstances is: {\itshape There are four kinds of
cards} (namely clubs, diamonds, hearts, and spades). Another possible answer is:
{\itshape There are two kinds of cards, red and black}. Still another answer is: {\it
There are $13$ kinds of cards: aces, twos, threes, $\cdots$, jacks, queens, and
kings}. Another answer, for the purpose of many card games is: {\itshape There are ten
kinds of cards, aces, twos, threes, up to nines, while tens, jacks, queens, and kings
are all considered to be the same value (usually called 10)}. You can certainly
think of many other ways to split the deck into a number of different kinds.

  Whenever the idea of {\itshape same kind} is used, some properties of the objects
being considered are deemed important and others are ignored. For instance, when
we think of the the deck of cards made of the $13$ different ranks, ace through
king, we are agreeing the the suit of the card is irrelevant. So the jack of
hearts and the jack of clubs are taken to be the same for what ever purposes we
have in mind.

\section{Equvialence relation}
  The mathematical term for {\itshape same kind} is {\bfseries equivalent}. There are
three basic properties always associated with the idea of equivalence.
\begin{enumerate}
 \item \emph{Reflexive}: Every object is equivalent to itself.
 \item \emph{Symmetric}: If object $a$  is equivalent to object $b$, then $b$ is
 also equivalent to $a$.
 \item \emph{Transitive}: If $a$ is equivalent to $b$ and $b$ is equivalent to
 $c$, then $a$ is equivalent to $c$.
\end{enumerate}


  To put the idea of equivalence in the context of a relation, suppose we have
a set $A$ of objects, and a rule for deciding when two objects in $A$ are the
same kind (equivalent) for some purpose. Then we can define a relation $E$ on the
set $A$ by the rule that the pair $(s,t)$ of elements of $A$ is in the relation
$E$ if and only if $s$ and $t$ are the same kind. For example, consider again the
deck of cards, with two cards considered to be the same if they have the same
rank. Then a few of the pairs in the relation $E$ would be $($ace hearts, ace
spades$)$, $($three diamonds, three clubs$)$, $($three clubs, three diamonds$)$,
$($three 
diamonds, three diamonds$)$, $($king diamonds, king clubs$)$, and so on.

  Using the terminology of the previous chapter, this relation $E$, and in
fact any relation that corresponds to notion of equivalence, will be reflexive,
symmetric, and transitive. For that reason, any reflexive, symmetric, transitive
relation on a set $A$ is called an {\bfseries equivalence relation} on $A$.

\section{Equivalence class of a relation}
  Suppose $E$ is an equivalence relation on a set $A$ and that $x$ is one
particular element of $A$. The {\bfseries equivalence class of $x$} is the set of all
the things 
in $A$ that are equivalent to $x$. The symbol used for the equivalence class of $x$ is
$[x]$, so the definition can be written in symbols as $[x]=\{y\in A|y\,E\,x\}$.

For instance, think once more about the deck of
cards 
with the 
equivalence relation {\itshape having the same rank}. The equivalence class of the two
of spades would be the set $[2\spadesuit]=\{2\clubsuit, 2\diamondsuit, 
2\heartsuit, 2\spadesuit\}$. 
That would also be the equivalence class of the two of diamonds. On
the other hand, if the equivalence relation we are using for the deck is {\itshape having
the same suit}, then the equivalence class of the two of spades would be 

$[2\spadesuit]=\{A\spadesuit
, 2\spadesuit , 3\spadesuit , 4\spadesuit , 5\spadesuit , 6\spadesuit,
 7\spadesuit , 8\spadesuit , 9\spadesuit,
10\spadesuit , J\spadesuit , Q\spadesuit , K\spadesuit \}$.

 
 The most important fact about the collection of different equivalence
classes for an equivalence relation on a set $A$ is that they split the set $A$
into separate pieces. In fancier words, they {\bfseries partition} the set $A$. For
example, the equivalence relation of having the same rank splits a deck of cards
into $13$ different equivalence classes. In a sense, when using this equivalence
relation, there are only $13$ different objects, four of each kind.

\section{Examples}
Here are a few more examples of equivalence relations.

\begin{exmp}
Define $R$ on $\N$ by $aRb$ iff $a=b$. In other words, equality is an 
equivalence relation. 
If fact, this example explains the choice of name for such relations. 
\end{exmp}

\begin{exmp} 
Let $A$ be the set of logical propositions and define $R$ on $A$ by
$pRq$ iff $p\equiv q$. 
\end{exmp}

\begin{exmp}  
Let $A$ be the set of people in the world and define $R$ on $A$ 
by $aRb$ iff $a$ and $b$ are the same age in years.
\end{exmp}

\begin{exmp} 
Let $A=\{1,2,3,4,5,6\}$ and $R$ be the relation on $A$ with the 
matrix from exercise 3. part a) of chapter 9.
\end{exmp}

\begin{exmp}\label{ex:10.5} 
Define $P$ on $\Z$ by $a\,P\,b$ iff $a$ and $b$ are
both even, or both odd. We say $a$ and $b$ have the same parity.
\end{exmp}


For the equivalence relation {\itshape has the same rank} on a set of cards
in a $52$ card deck, there are $13$ different equivalence classes. One of the classes
contains all the aces, another contains all the $2$'s, and so on.

\clearpage
\begin{exmp}
 For the equivalence relation from example \ref{ex:10.5}, the equivalence class of $2$ is the
 set of all even integers. 
 \begin{align*}
  [2] = \{ n\,|\, 2\,P\,n\} &=\{ n\,|\, 2 \text{ has the same parity as } n\,\} \\
                            &= \{ n\,|\, n \text{ is even}\,\} = \{ \cdots, -4, -2,0,2,4, \cdots\}
 \end{align*}
 In this example, there are two different equivalence classes, the one comprising
 all the even integers, and the other comprising all the odd integers. As far
 as parity is concerned, $-1232215$ and $171717$ are {\itshape the same}.
\end{exmp}


Suppose $E$ is an equivalence relation on $A$. 
The most important fact about equivalence classes is that every
element of $A$ belongs to exactly one equivalence class. Let's prove that.

\begin{thm}\label{thm:eqvrel->partition}
 Let $E$ be an equivalence relation on a set $A$, and let
 $a\in A$. Then there is exactly one equivalence class to which $a$ belongs.
\end{thm}
\begin{proof}
 Let $E$ be an equivalence relation on a set $A$, and suppose $a\in A$.
 Since $E$ is reflexive, $a\,E\,a$, and so $a\in [a]$ is true. That proves that $a$
 is in at least one equivalence class. To complete the proof, we need to show that
 if $a\in [b]$ then $[b]=[a]$. 
 
 Now, stop and think: Here is what we know: 
\begin{enumerate}
 \item $E$ is an equivalence relation on $A$,
 \item $a\in [b]$, and 
 \item the definition of equivalence class.
\end{enumerate} Using those three pieces of
 information, we need to show the two sets $[a]$ and $[b]$ are equal.
 Now, to show two sets are equal, we show they have the same elements.
 In other words, we want to prove 
\begin{enumerate}
 \item If $c\in [a]$, then $c\in [b]$, and%
\marginnote{For homework, you will complete the proof of this theorem  by doing part (1).} %
   
 \item If $c\in [b]$, then $c\in [a]$.
\end{enumerate}
 Let's give a direct proof of (2).
 
 Suppose $c\in [b]$. Then, according to the definition of $[b]$, $c\,E\,b$.
 The goal is to end up with {\itshape So $c\in [a]$}.  Now, we know $a\in [b]$,
 and that means $a\,E\,b$. Since $E$ is symmetric and $a\,E\,b$, it follows that $b\,E\,a$.
 Now we have $c\,E\,b$ and $b\,E\,a$. Since $E$ is transitive, we can conclude
 $c\,E\,a$, which means $c\in [a]$ as we hoped to show. That proves (2).
\end{proof}

 

\section{Partitions}
\begin{defn}
 A {\bfseries partition} of a set  $A$ is a collection of nonempty, 
 pairwise disjoint subsets of $A$, so that $A$ is
 the union of the subsets in the collection. 
 So for example $\{\{1,2,3\}\,\{4,5,6\}\}$ is a partition
 of $\{1,2,3,4,5,6\}$. 
 The subsets forming a partition are called the {\bfseries parts of the partition}.
\end{defn}


So to express the meaning of theorem \ref{thm:eqvrel->partition} above in different words: The different equivalence 
classes of an equivalence relation on a set partition the set into nonempty disjoint
pieces. More briefly: the equivalence classes of $E$ {\bfseries partition} $A$.

\section{Digraph of an equivalence relation}
The fact that an equivalence relation partitions the underlying set
is reflected in the digraph of an equivalence relation.
If we pick an equivalence class $[a]$ of an equivalence relation
$E$ on a finite set $A$ and we pick $b\in [a]$, then $b\,E\,c$ for all $c\in [a]$. This is true
since $a\,E\,b$ implies $b\,E\,a$ and if $a\,E\,c$, then transitivity fills in $b\,E\,c$. So in any digraph
for $E$ every vertex of $[a]$ is connected to every other vertex in $[a]$ (including itself)
by a directed edge. Also no vertex in $[a]$ is connected to any  vertex in $A-[a]$. 
So the digraph
of $E$ consists of separate components, one for each distinct equivalence
class, where each component contains every possible directed edge.
\section{Matrix representation of an equivalence relation}
In terms of a matrix representation of an equivalence relation $E$ on a finite set $A$ 
of size $n$,
let the distinct equivalence classes have size $k_1, k_2,... k_r$, 
where $k_1+k_2+...+k_r = n$.
Next list the elements of $A$ as 
$a_{1,1},...,a_{k_1,1},a_{1,2},...,a_{k_2,2},.....,a_{1,r},...,a_{k_r,r}$
where the $i$th equivalence class is $\{a_{1,i},...,a_{k_i,i}\}$. 
Then the matrix for $R$ with respect to this
ordering is of the form 
\[
 \left[
  \begin{matrix}
   J_{k_1}&  0 & 0 & ... & 0 \\
   0 & J_{k_2} & 0 & ... & 0 \\ 
   \vdots & \ddots &\ddots & \ddots & \vdots \\
   0 & ... & 0 & J_{k_{r-1}} & 0 \\
   0 & ... & 0 & 0 & J_{k_r}
  \end{matrix}
 \right]
\]
where $J_m$ is the all $1$'s matrix of size $k_m\times k_m$.
Conversely if the digraph of a relation can be drawn to take the above form, 
or if it has a matrix representation
of the above form, then it is an equivalence relation and therefore reflexive, 
symmetric, and transitive.

\clearpage

\section{Exercises} 

\begin{exer}
Let $A = \{0,1,2\}$. Let $R = \{(0,0), (1,1), (2,2), (0,1), (1,0)\}$. 
Is $R$ and equivalence relation on $A$? If it is, what are the equivalence classes?
\end{exer}

\begin{exer}
Let $A = \{0,1,2,3\}$. Let $R = \{(0,0), (1,1), (2,2), (0,1), (1,0)\}$. 
Is $R$ and equivalence relation on $A$? If it is, what are the equivalence classes?
\end{exer}

\begin{exer}
Let $A = \{0,1,2\}$. Let $R = \{(0,0), (1,1), (2,2), (0,1)\}$. 
Is $R$ and equivalence relation on $A$? If it is, what are the equivalence classes?
\end{exer}

\begin{exer}
Let $A = \{0,1,2\}$. Let $R = \{(0,0), (1,1), (2,2), (0,1), (1,0),  (1,2), (2,1) \}$. 
Is $R$ and equivalence relation on $A$? If it is, what are the equivalence classes?
\end{exer}

\begin{exer}
True or False: The relation $R = \{ (1,1), (2,2)\}$ on $A = \{1,2\}$ is both symmetric and antisymmetric.
\end{exer}

\begin{exer}
The relation $S$ is defined on the set $\Z$ of all integers by the rule $m\,S\, n$ if and only if $m^{2} = n^{2}$.
Is $S$ an equivalence relation on $\Z$? If it is, what are the equivalence classes of $S$?
\end{exer}

\begin{exer}
Let $L$ be the collection of all straight lines in the plane. Four examples of elements in $L$: $x + y = 0$, 2x-y - 5$, x = 7, y = 0$.
A relation $C$ on $L$ is defined by the rule $l_{1}\, C \, l_{2}$ provided the lines $l_{1}$ and $l_{2}$ have at least one point in common.
(The letter $C$ should remind us of {\itshape cross}, and, loosely speaking, two lines are related if they cross each other. We will have to 
agree that a line crosses itself.) Is $C$ an equivalence relation on $L$? If it is, what are the equivalence classes of $C$?
\end{exer}

\begin{exer}
Let $R$ be a relation on a non-empty set $A$ that is both symmetric, transitive. And, suppose that
for each  $a\in A$,  $aRb$ for at least one  $b\in A$. Prove that $R$ is reflexive, hence, an equivalence relation.
\end{exer}

\begin{exer} Let $E$ be an equivalence relation on a set $A$, and let $a, b \in A$. Prove that
either  $[a]\cap [b] = \emptyset$ or else $[a] = [b]$.
\end{exer}


\begin{exer} Let $A=\{1,2,3,4,5,6,7,8\}$. 
Form a partition of $A$ using $\{1,2,4\}, \{3,5,7\}$, and $\{6,8\}$.
These are the equivalence classes for an equivalence relation, $E$, on $A$.
\begin{tasks}(2)
     \task Draw a digraph of $E$.
     \task Determine a $0$-$1$ matrix of $E$. 
\end{tasks}
\end{exer}

\begin{exer} Determine if each matrix represents an equivalence relation on $\{a,b,c,d,e,f,g,h\}$.
If the matrix represents an equivalence relation find the equivalence classes.  The natural order
of the elements,  $[a,b,c,d,e,f,g,h]$, defines the matrices.

\vspace*{0.25cm}
\begin{enumerate*}[label=(\alph*), itemjoin=\qquad]
\item 
 $\left[
\begin{matrix}
 1&0&1&0&1&0&1&0\\ 
 0&1&0&1&0&1&0&0\\ 
 1&0&1&0&1&0&1&0\\
 0&1&0&1&0&1&0&0\\ 
 1&0&1&0&1&0&1&0\\ 
 0&1&0&1&0&1&0&0\\ 
 1&0&1&0&1&0&1&0\\
 0&0&0&0&0&0&0&1
\end{matrix}
\right]$ 

\item 
$\left[
\begin{matrix}
 1&1&0&0&0&0&1&1\\
 1&1&0&0&0&0&1&1\\ 
 0&0&1&1&0&0&0&0\\ 
 0&0&1&1&0&0&0&0\\
 0&0&0&0&1&1&0&0\\ 
 0&0&0&0&1&1&0&0\\ 
 1&1&0&0&0&0&1&1\\
 1&1&0&0&0&0&1&1 
\end{matrix}
\right]$
\end{enumerate*}
\end{exer}

\begin{exer} 
Complete the proof of theorem~\ref{thm:eqvrel->partition} on page~\pageref{thm:eqvrel->partition} by proving part (1).
\end{exer}

\clearpage

\section{Problems}

\begin{prob} 
Let $A$ be the set of people alive on earth. 
For each relation defined below, determine if
it is an equivalence relation on $A$. If it is, describe the equivalence classes.
If it is not, determine which properties of an equivalence relation fail.
\begin{tasks}
  \task $a\,H\,b \iff$  $a$ and $b$ are the same height.
  \task $a\,G\,b \iff$  $a$ and $b$ have a common grandparent.
  \task $a\,L\,b \iff$  $a$ and $b$ have the same last name.
  \task $a\,N\,b \iff$ $a$ and $b$ have a name (first name  or last name) in common.
  \task $a\,W\,b \iff$ $a$ and $b$ were born less than a day apart.
\end{tasks}
\end{prob}

\begin{prob}
Let $L$ be the collection of all straight lines in the plane. Four examples of elements in $L$: $x + y = 0$, 2x-y - 5$, x = 7, y = 0$.
A relation $P$ on $L$ is defined by the rule $l_{1}\, P \, l_{2}$ provided the lines $l_{1}$ and $l_{2}$ are parallel.
Is $P$ an equivalence relation on $L$? If it is, what are the equivalence classes of $P$?
\end{prob}

\begin{prob}
\ %
  \begin{tasks}
     \task Given an example of an equivalence relation on $\N$ for which there are exactly
     two equivalence classes.
     \task Given an example of an equivalence relation on $\N$ for which every equivalence class
     has cardinality two.
  \end{tasks}
\end{prob}

\begin{prob} 
Consider the relation \textbf{$B(x,y) : x$ is the brother of $y$} on the set, $M$,
of living human males. Is  $M$ reflexive? Is $M$ symmetric? Is $M$ transitive?
(To be precise, {\itshape brothers} will mean two different males with the same two
parents. Don't consider half{-}brothers for this problem.) 
\end{prob}

\begin{prob}
Let $A = \{ a,b,c,d,e,f,g\}$. There are many different equivalence relations on $A$.
  \begin{tasks}
    \task Of all the equivalence relations on $A$, which have the smallest number of ordered pairs?
    \task Of all the equivalence relations on $A$, which have the  largest number of ordered pairs?
  \end{tasks}
\end{prob}

\begin{prob}
The relation $R = \{ (a,a), (a,b)\}$ is not an equivalence relation on the set $A = \{ a,b,c\}$. What is the fewest number of ordered pairs that need to be added to $R$ so the result is an equivalence relation on $A$?
\end{prob}

\begin{prob}
 Prove or give a counterexample: Suppose $R$ is an equivalence relation on the lower case letters of the alphabet. True or False: All the equivalence classes of $R$ have the same cardinal number. 
\end{prob}

\begin{prob}
Let $A$ be the set of all ordered pairs of positive integers.
So some members of $A$ are $(3,6),  (7,7), (11,4), (1,2981)$. A relation on $A$ is defined by the rule 
$(a,b) R (c,d)$ if and only if $ad = bc$. For example $(3,5) R (6,10)$ is true since $(3)(10)=(5)(6)$.\\[3pt]
  \begin{tasks}
     \task Explain why $R$ is an equivalence relation on $A$.
     \task List four ordered pairs in the equivalence class of $(2,3)$.
  \end{tasks}
\end{prob}

\begin{prob}   
Let $A=\{1,2,3,4,5,6\}$. 
Form a partition of $A$ using $\{1, 2\}, \{3, 4, 5\}$, and $\{6\}$.
These are the equivalence classes for an equivalence relation, $E$, on $A$.
 Draw the  {\bf digraph} of $E$.
 \end{prob}

\begin{prob}
Let $A = \{ 1,2,3\}$. The relation $E = \{ (1,1),(2,2),(3,3),(2,3),(3,2)\}$ is an
equivalence relation on $A$.  $F=\{ (1,1),(2,2),(3,3),(1,2),(2,1)\}$ 
is another equivalence relation on $A$. Compute the composition $F\circ E$.
Is $F\circ E$ and equivalence relation on $A$?
\end{prob}



