\chapter{The Integers}

\newthought{{\bfseries Number theory} is concerned} with the integers and their properties. 
In this chapter  the rules of the arithmetic of integers are reviewed. The 
surprising fact is that all the dozens of rules and tricks you know for working 
with integers (and for doing algebra, which is just arithmetic with symbols) 
are consequences of just a few basic facts. The list of facts given in sections~\ref{sect:integer operations} and \ref{sect: Z order properties}
is actually longer than necessary; several of these rules can be derived
from the others. 


\section{Integer operations}\label{sect:integer operations}
The set of  {\bfseries integers}, $\{ \cdots ,  -2,-1,0,1,2,\cdots\}$,  is denoted by 
the symbol $\Z$.
The two familiar arithmetic operations for the integers, addition and
multiplication, obey  several basic rules. First, notice that addition and
multiplication are {\bfseries binary operations}. In other words, these two
operations combine a pair of integers to produce a value. It is not possible to add
(or multiply) three numbers at a time. We can figure out the sum of three numbers,
but it takes two steps: we select two of the numbers, and add them up, and then
add the third to the preliminary total. Never are more than two numbers added
together at any time. A list of the seven fundamental facts about
addition and multiplication of integers follows.

\clearpage
\begin{enumerate}
 \item The integers are {\bfseries closed} with respect to addition and multiplication.
 
 That means that when two integers are added or multiplied, the result is another
 integer. In symbols, we have 
 \[
 \forall a,b\in \Z,\ ab\in \Z \text{ and } a+b\in \Z.
 \]
 
 
 \item Addition and multiplication of integers are {\bfseries commutative} 
 operations.
 
 That means that the {\it order} in which the two numbers are combined has no
 effect on the final total. Symbolically, we have
 \[
 \forall a,b\in \Z,\ a+b=b+a \text{ and }  ab=ba.
 \]
 
 
 \item Addition and multiplication of integers are {\bfseries associative} 
 operations.
 In other words, when we compute the sum (or product) of three integers, it does
 not matter whether we combine the first two and then add the third to the total, 
 or add the first to the total of the last two. The final total will be the same
 in either case. Expressed in symbols, we have
 \[
 \forall a,b,c\in \Z,\ a(bc)=(ab)c \text{ and } a+(b+c)=(a+b)+c.
 \]
 
 
 \item There is an {\bfseries additive identity} denoted by $0$. 
 It has the property that when it is added to any number the result is
 that number right back again. In symbols, we see that
 \[
 0+a=a=a+0 \text{ for all } a\in \Z.
 \]
 
 
 \item Every integer has an {\bfseries additive inverse}: $\forall n\in \Z, \exists m\in \Z$ 
 so that $n+m=0=m+n$. As usual, $m$ is denoted by $-n$. So, we write $n+(-n)=(-n)+n = 0$.
 
 
 \item $1$ is a {\bfseries multiplicative identity}. That is, we have $1a=a=a1$ for all $a\in \Z$.
 
 And finally, there is a rule which establishes a connection between the operations of addition and multiplication.
 
 
\clearpage 
 \item Multiplication {\bfseries distributes} over addition. Again, we symbolically write
 \[
 \forall a,b,c\in \Z,\ a(b+c)=ab+ac.
 \]
\end{enumerate}


The seven facts in section~\ref{sect:integer operations}, together with a few concerning ordering stated in section~\ref{sect: Z order properties},  tell all there is to 
know 
about arithmetic. Every other fact
can be proved from these. For example, here is a proof of the cancellation
law for addition using the facts listed above.

\begin{thm}[Integer cancellation law]
 For integers $a,b,c$, if $a+c = b+c$ then $a=b$.
\end{thm}
\begin{proof}
 Suppose $a+c=b+c$. Add $-c$ to both sides of  that equation
 (applying fact 5 above) to get
 $(a+c) + (-c) = (b+c) +(-c)$. Using the associative rule, that equation
 can be rewritten as $a+ (c+(-c)) = b + (c+(-c))$, and that becomes
 $a+0=b+0$. By property 4 above, that means $a=b$.
\end{proof}

\begin{thm}
For any integer $a$, $a0=0$.
\end{thm}
\begin{proof}
Here are the steps in the proof. You supply the justifications
for the steps.
\begin{align*}
 a0& = a(0+0) \\
 a0 & = a0 + a0 \\
 a0 + (-(a0)) & = (a0 + a0) + (-(a0)) \\
 a0 + (-(a0)) & = a0 + (a0 + (-(a0))) \\
 0 & = a0 + 0 \\
 0 & = a0
\end{align*}
\end{proof}

Your justification for each step should be stated as using one, or more, of the fundamental facts 
as applied to the specific circumstance in each line.


\clearpage
\section{Order properties}\label{sect: Z order properties}
The integers also have an order relation, {\it $a$ is less than or
equal to $b$}: $a\leq b$. This relation satisfies three fundamental
order properties: $\leq$ is a reflexive, antisymmetric, and transitive relation on $\Z$.
 
 The notation $b\geq a$ means the same as $a\leq b$. Also $a<b$ (and $b>a$)
 are shorthand ways to say $a\leq b$ and $a\not= b$.
 
 The {\bfseries trichotomy law} holds: 
 for $a\in \Z$ exactly one of $a>0, a=0,$ or  $a<0$ is true.

The ordering of the integers is related to the arithmetic by several rules:
\begin{enumerate}

 \item If $a<b$, then $a+c<b+c$ for all $c\in \Z$.
 
 \item If $a<b$ and $c>0$, then $ac<bc$.
 
 \item If $a<b$ and $c<0$, then $bc<ac$.
\end{enumerate}

And, finally, the rule that justifies proofs by induction:

The Well Ordering Principle for $\Z$:
 The set of positive integers is {\bfseries well-ordered}: 
  every nonempty subset of positive integers has a least element. 


\clearpage
\section{Exercises}

\begin{exer} 
Prove that if $a>0$ and $b>0$, then $ab>0$.
\end{exer}

\begin{exer}\label{zero property}
Prove that if $ab=0$, then $a=0$ or $b=0$.  Hint: Try an indirect proof with four cases.
Case 1: Show that if $a>0$ and $b>0$, then $ab\not=0$. Case 2: Show that if $a>0$ and $b<0$,
then $ab\not=0$. There are two  more similar cases. (This fact is called the {\it zero property}.)
\end{exer}

\begin{exer} 
Prove the cancellation law for multiplication: 
For integers $a,b,c$, with $c\not=0$, if $ac=bc$, then $a=b$. (Hint: Use exercise \ref{zero property})
\end{exer}



\section{Problems}

\begin{prob}
Prove that if $a>0$ and $b<0$, then $ab <0$.
\end{prob}

\begin{prob}
Prove that if $n$ is an integer, then $n^{2}\geq 0$.
\end{prob}

\begin{prob}
Prove that is $m^{2} = n^{2}$, then $m=n$ or $m = -n$.\\
 (Hint from algebra: $a^{2}-b^{2} = (a+b)(a-b)$.) 
\end{prob}




