\chapter{The Two Fundamental Counting Principles}

\newthought{The next few chapters will deal} with the topic of {\bf combinatorics:} the art of
counting. By counting we mean determining the number of different ways of 
arranging objects in certain patterns or the number of ways of carrying out 
a sequence of tasks. For example, suppose we want to count the number of ways
of making a bit string of length two. Such a problem is small enough that the 
possible arrangements can be counted by {\itshape brute force}. In other words,
we can simply make a list of all the possibilities: $00, 01, 10, 11$. So the answer is
four. If the problem were to determine the number of bit strings of length fifty,
the brute force method loses a lot of its appeal. For problems where brute force
counting is not a reasonable alternative, there are a few principles we can apply
to aid in the counting. In fact, there are just two basic principles on which all
counting ultimately rests.

Throughout this chapter, all sets mentioned will be finite sets, and if $A$ is a set, $|A|$
will denote the number of elements in $A$.

\section{The sum rule}
The {\bfseries sum rule} says that if the sets $A$ and $B$ are disjoint, then
\[
\lvert A\cup B\rvert = \lvert A\rvert+\lvert B\rvert.
\]

\clearpage
\begin{exmp}
 For example, if $A=\{a,b,c\}$ and $B=\{j,k,l,m,n\}$, then $\lvert A\rvert=3, \lvert B\rvert=5$, and, sure
 enough, 
 \[
 \lvert A\cup B\rvert = \lvert \{a,b,c,j,k,l,m,n\}\rvert = 8 = 3+5.
 \]
\end{exmp}

Care must be used when\marginnote{\dbend}
applying the sum principle that the sets are disjoint. If $A=\{a,b,c\}$ and $B=\{b,c,d\}$,
then $\lvert A\cup B \rvert = 4$, and not $6$.

\begin{exmp}
 As another example of the sum principle, if we have a collection of $3$ dogs and $5$ cats,
 then we can select one of the animals in $8$ ways.
\end{exmp}

\subsection{Counting two independent tasks}
The sum principle is often expressed in different language: If we can do task $1$ in $m$ ways
and task $2$ in $n$ ways, and the tasks are {\itshape independent} (meaning that both
tasks cannot be done at the same time), then there are $m+n$ ways to do one of the two tasks.
The independence of the tasks is the analog of the disjointness of the sets in the set version
of the sum rule.

A serious type of error\marginnote{\dbend}
 is trying to use the sum rule for tasks that are
not independent. For instance, suppose we want to know \emph{in how many different ways
we can select  either a deuce or a six from an ordinary deck of $52$ cards.} We
could let the first task be the process of selecting a deuce from the deck. That
task can be done in $4$ ways since there are $4$ deuces in the deck. For the
second task, we will take the operation of selecting a six from the deck. Again,
there are $4$ ways to accomplish that task. Now these tasks are independent since
we cannot simultaneously pick a deuce and a six from the deck. So, according to
the sum rule, there are $4+4=8$ ways of selecting one card from a deck, and having
that card be either a deuce or a six.\

Now consider the similar sounding question: \emph{In how many ways can we select
either a deuce or a diamond from a deck of $52$ cards?} We could let the first task
again be the operation of selecting a deuce from the deck, with $4$ ways to carry
out that task. And we could let the second task be the operation of selecting a
diamond from the deck, with $13$ ways to accomplish that. But in this case, the
answer to the question is not $4+13 = 17$, since these tasks are not
independent. It is possible to select a card that is both a deuce and a
diamond. So the sum rule cannot be used. What is the correct answer? Well, there
are $13$ diamonds, and there are $3$ deuces besides the two of diamonds, and so
there are actually $16$ cards in the deck that are either a deuce or a diamond. That
means there are $16$ ways to select a card from a deck and have it turn out to be
either a deuce or a diamond.

\subsection{Extended sum rule}
The sum rule can be extended to the case of more than two sets (or more than two tasks):
If $A_1, A_2, A_3, \cdots,A_n$ is a collection of {\itshape pairwise disjoint} sets,
then $|A_1\cup A_2\cup A_3\cup\cdots\cup A_n| = |A_1|+|A_2|+|A_3|+\cdots+|A_n|$.
Or, in terms of tasks: If task $1$ can be done in $k_1$ ways, and task $2$ in $k_2$,
and task $3$ in $k_3$ ways, and so on, until task $n$ can be done in $k_n$ ways,
and if the tasks are all independent\sidenote{They must be \textbf{pairwise} independent!}, 
then we can do one task in $k_1+k_2+k_3+\cdots + k_n$
ways. 

\begin{exmp}
 For example, if we own three cars, two bikes, a motorcycle, four pairs of roller skates, and
 two scooters, then we can select one of these modes of transportation in 
 $3+2+1+4+2= 12$ ways.
\end{exmp}

\subsection{Sum rule and the logical \textit{or}}
The sum rule is related to the logical connective {\itshape or}. That is reasonable since the
sum rule counts the number of elements in  the set $A\cup B = \{\,x\,|\,x\in A\hbox{ or } x\in B\,\}$.
In terms of tasks, the sum rule counts the number of ways to do either task $1$ or task $2$.
Generally speaking, when the word {\itshape or} occurs in a counting problem, the sum rule
is the tool to use.\marginnote{But, verify independence!}
\ms
 \section{The product rule}
The logical connective {\itshape and} is related to the second fundamental counting principle:
the {\bfseries product rule}. The product rule says: 
\[
\lvert A\times B \rvert = \lvert A \rvert \cdot \lvert B \rvert.
\] 
An explanation of this
is that $A\times B$ consists of all ordered pairs $(a,b)$ where $a\in A$ and $b\in B$. 
There are $\lvert A \rvert$ choices for $a$ and then $\lvert B \rvert$ choices for $b$. 

\subsection{Counting two sequential tasks: logical \textit{and}}
In terms of tasks, the product rule says that if task $1$ can be done in $m$ ways
 and 
task $2$ can be done in $n$ ways after task $1$ has been done, then there are
$mn$ ways to do both tasks, the first
then the second. Here the relation with the logical connective {\itshape and} is also obvious. We
need to do task $1$ and task $2$. Generally speaking, the appearance of {\itshape and} in a 
counting problem suggests the product rule will come into play.

\subsection{Extended product rule}
As with the sum rule, the product rule can be used for situations with more than two sets
or more than two tasks. In terms of sets, the product rule reads
$\lvert A_1\times A_2\times \cdots A_n \rvert = \lvert A_1\rvert\cdot\lvert A_2 \rvert \cdots \lvert A_n \rvert$. In terms of tasks, it reads,
if task $1$ can be done in $k_1$ ways, and for each of those ways, task $2$ can be done in 
$k_2$ ways, and for each of those ways, task $3$ can be done in $k_3$ ways, and so on, until
for each of those ways, task $n$ can be done in $k_n$ ways, then we can do
task $1$ followed by task $2$ followed by task $3$, etc, followed by task $n$ in
$k_1k_2k_3\cdots k_n$ ways. That sounds worse than it really is.

\begin{exmp}
 How many bit strings are there of length five?
 \begin{soln}
 We can think of task $1$ as filling in the first (right hand) position, task $2$ as filling in the
 second position, and so on.
 We can argue that 
 we have two ways to do task $1$,
 and then two ways to do task $2$, 
 and then two  ways to do task $3$, 
 and then two ways to do task $4$, 
 and then two ways to do task $5$.
 So, by the product rule, \marginnote{The same reasoning shows that, in general, there are $2^n$ bit strings of length $n$.}
 there are $2\cdot 2\cdot 2\cdot 2\cdot 2=2^5=32$ ways to do all five tasks, and so there
 are $32$  bit strings of length five.
 \end{soln}
\end{exmp}


\begin{exmp}
Suppose we are buying a car with five choices
for the exterior color and three choices for the interior color. Then there is a
total of $3\cdot5=15$ possible color combinations that we can choose from. The
first task is to select an exterior color, and there are $5$ ways to do that. The
second task is to select an interior color, and there are $3$ ways to do that. So
the product rule says there are $15$ ways total to do both tasks. Notice that
there is no requirement of independence of tasks when using the product
rule. However, also notice that the number of ways of doing the second task \textbf{must}
be the same no matter what choice is made for doing the first task.
\end{exmp}

\begin{exmp}\label{exmp:two-digit 1-9}
For another,
slightly more complicated, example of the product rule in action, suppose we
wanted to make a two-digit number using the digits $1$, $2$, $3$, $4$, $5$, $6$,
$7$, $8$, and $9$. How many different such two-digit numbers could we form? Let's
make the first task filling in the left digit, and the second task filling in the
right digit. There are $9$ ways to do the first task. And, no matter how we do
the first task, there are $9$ ways to do the second task as well. So, by the
product rule, there are $9\cdot9=81$ possible such two-digit numbers. 
\end{exmp}

\begin{exmp}\label{exmp:two-digit no dups}
 Now, let's
 change the problem in example~\ref{exmp:two-digit 1-9} a little bit. 
 Suppose we wanted two-digit numbers made up of
 those same nine digits, but we do not want to use a digit more than once in any of
 the numbers. In other words, $37$ and $91$ are OK, but we do not want to count
 $44$ as a possibility. We can still make the first task filling in the left digit,
 and the second task filling in the 
 right digit. And, as before, there are $9$ ways to do the first task. But now,
 once the first task has been done, there are only $8$ ways to do the second task,
 \marginnote{No matter in what way the first task  was done, there are always $8$ ways to
 to the second task in sequence. What if you chose to pick the second digit first?}
 since the digit used in the first task is no longer available for doing the
 second task. For instance, if the digit $3$ was selected in the first task,
 then for the second task, we will have to choose from the eight digits $1$, $2$,
 $4$, $5$, $6$, 
 $7$, $8$, and $9$. So, according to the product rule, there are $9\cdot8=72$ ways
 of building such a number. 
\end{exmp}


\begin{exmp}\label{exmp:two-digits no dup 2}
 Just for fun, here is another way to see the answer in example~\ref{exmp:two-digit no dups} is
 $72$. We saw above that there are $81$ ways to make a two-digit number when we
 allow repeated digits. But there are $9$ two digit numbers that do have
 repeated digits (namely $11$, $22$, $\cdots$, $99$). That means there must be
 $81-9 =72$  two-digit numbers without repeated digits.
\end{exmp}

\subsection{Counting by subtraction: $Good \,=\, Total\,-\,Bad$}
The trick we used in example~\ref{exmp:two-digits no dup 2}  looks like a new counting principle, but it is really the sum rule being
applied in a tricky way. Here's the idea. Call the set of all the two-digit numbers
(not using $0$) $T$, call the set with no repeated digits $N$, and call the set
with repeated digits $R$. By the sum rule, $|T|=|N|+|R|$, so $|N| = |T|-|R|$.
This is a very common trick. 

Generally, suppose we are interested in counting some arrangements,
let's call them the {\itshape Good} arrangements. But it is not easy for some reason
to count the {\itshape Good} arrangements directly. So, instead, we count the {\itshape Total}
number of arrangements, and subtract the number of {\itshape Bad} arrangements:
\[
Good \,=\, Total\,-\,Bad.
\]
Let's have another example of this trick.
\begin{exmp}\label{exmp:words >0 vowels}
 By a word of length five, we will mean any string
 of five letters from the $26$ letter alphabet. How many words contain
 at least one vowel. The vowels are: \textsf{a,e,i,o,u}.
 
 By the product rule, there is a total of $26^5$ possible words of
 length five. The bad words are made up of only the $21$ non-vowels.
 So, by the sum rule, the number of good words is $26^5-21^5$.
\end{exmp}


\section{Using both the sum and product rules}
As in  example~\ref{exmp:words >0 vowels}, 
most interesting counting problems involve a combination of both the sum and
product rules.

\begin{exmp}
Suppose we wanted to count the number of different possible bit strings of length
five that start with either three $0$'s or with two $1$'s. Recall that a bit
string is a list of $0$'s and $1$'s, and the length of the bit string is the total
number of $0$'s and $1$'s in the list. So, here are some bit strings that satisfy
the stated conditions: $00001$, $11111$, $11011$, and $00010$. On the other hand,
the bit strings $00110$ and $10101$ do not meet the required condition. 

To do this
problem, let's first count the number of {\itshape good} bit strings that start with
three $0$'s. In this case, we can think of the construction of such a bit string as
doing five tasks, one after the other, filling in the leftmost bit, then the next
one, then the third, the next, and finally the last bit. There is only one way to
do the first three tasks, since we need to fill in $0$'s in the first three
positions. But there are two ways to do the last two tasks, and so, according to
the product rule there are $1\cdot1\cdot1\cdot2\cdot2 = 4$ bit strings of length
five starting with three $0$'s. Using the same reasoning, there are
$1\cdot1\cdot2\cdot2\cdot2=8$ bit strings of length five starting with two
$1$'s. Now, a bit string cannot both start with three $0$'s and also with two
$1$'s, (in other words, starting with three $0$'s and starting with two $1$'s are
independent). And so, according to the sum rule, there will be a total of $4+8=12$
bit strings of length five starting with either three $0$'s or two $1$'s.
\end{exmp}

\begin{exmp}
 How many words of six letters (repeats OK) contain exactly
 one vowel?
 \begin{soln}
  Let's break the construction of a good word down into a number of tasks. \begin{enumerate}[label=Task \arabic*:, itemindent=1cm]
    \item Select a spot for the vowel: $6$ choices.
     
     \item Select a vowel for that spot: $5$ choices.
     
     \item Fill first empty spot with a non-vowel: $21$ choices
     
     \item Fill next empty spot with a non-vowel: $21$ choices
     
     \item Fill next empty spot with a non-vowel: $21$ choices
     
     \item Fill next empty spot with a non-vowel: $21$ choices
     
     \item Fill last empty spot with a non-vowel: $21$ choices
   \end{enumerate}
   By the product rule, the number of good words is $6\cdot5\cdot21^5$.
 \end{soln}
 
\end{exmp}

\begin{exmp}\label{exmp:license plates}
Count the number of strings on license plates which either 
consist of three capital English letters, followed by three digits, or consist of two digits
followed by four capital English letters.
\begin{soln}
Let $A$ be the set of strings which consist of three capital English letters
followed by three digits, and $B$ be the set of strings which consist of two digits followed
by four capital English letters. By the product rule $\lvert A \rvert=26^3\cdot 10^3$ since
there are 26 capital English letters and 10 digits. Also by the product rule
$\lvert B \rvert=10^2\cdot 26^4$. Since $A\cap B=\emptyset$, by the sum rule
the answer is $26^3\cdot 10^3 + 10^2\cdot 26^4$.
\end{soln}
\end{exmp}


\section{Answer form $\longleftrightarrow$ solution method}
In the previous examples we might continue on with the arithmetic. For instance, in
the last, example~\ref{exmp:license plates},
using the distributive law on our answer to factor out common terms
we see $\lvert A\cup B \rvert=10^2\cdot 26^3(10+26)$
is an equivalent answer.
This, in turn, simplifies to $\vert A\cup B \rvert=10^2\cdot 26^3\cdot 36$, and that gives 
\[
\lvert A\cup B \rvert=100\cdot17576\cdot 36=63,273,600.
\]

Of all of these answers the most valuable is probably 
$26^3\cdot 10^3 + 10^2\cdot 26^4$, since {\bfseries the form
of the answer is indicative of the manner of solution.} 
We can readily observe
that the sum rule was applied to two disjoint subcases. For each subcase
the product rule was applied to compute the intermediate answer. As a general
rule, answers to counting problems should be left in this uncomputed form.

The next most useful solution is the last one. When we have an answer of this
form we can use it to consider whether or not our answer makes sense intuitively.
For example if we knew that $A$ and $B$ both were subsets of a set of cardinality 
$450$ and we computed that $\lvert A\cup B \rvert>450$, this would indicate that we made
an error, either in the logic of our counting, or in arithmetic, or both.

\clearpage
\section{Exercises}

\begin{exer}
To meet the science requirement a student must take
 one of the following courses: a choice of $5$ biology courses, $4$
physics courses, or $6$ chemistry courses. In how many ways can the
one course be selected?
\end{exer}

\begin{exer}
Using the data of problem 1, a student has decided to take one biology,
one physics, and one chemistry course. How many different such selections
are possible?
\end{exer}

\begin{exer}
A serial code is formed in one of three ways: (1) two letters followed by two digits, or
(2) three letters followed by one digit, or (3) four letters. How many different codes are there?
 (Unless otherwise
indicated, {\itshape letters} will means
 upper case letters chosen from the usual
$26$-letter alphabet and {\itshape digits} are selected from 
$\{0,1,2,3,4,5,6,7,8,9\}$.) 
\end{exer}

\begin{exer}
How many words of length
six are there if letters may be repeated? (Examples: BBBXBB, ABATBC are OK).
\end{exer}

\begin{exer}
How many words of length
six are there if letters may not be repeated? (Examples: BBBBXB, ABATJC are bad
but ABXHYR is OK).
\end{exer}

\begin{exer}
A true/false test contains 25 questions. 
\begin{enumerate}[label=(\alph*)]
 \item How many ways can a student complete the test if every question 
 must be answered?
 
 \item How many ways can a student complete the test if questions 
 can be left unanswered?
\end{enumerate}
\end{exer}

\begin{exer}
How many binary strings of length less than or equal to nine are there?
\end{exer}

\begin{exer}
How many eight-letter words contain at least one $A$?
\end{exer}

\begin{exer}
How many seven-letter words contain at most one $A$?
\end{exer}

\begin{exer}
How many nine-letter words contain at least two $A$'s?
\end{exer}

\section{Problems}

\begin{prob}
My piggy bank contains $20$ pennies, $4$ nickels, $7$ dimes, and $2$ quarters. In how many ways can I select one coin?
\end{prob}

\begin{prob}
My piggy bank contains $20$ pennies, $4$ nickels, $7$ dimes, and $2$ quarters. In how many ways can I select four coins, one of each value?
\end{prob}

\begin{prob}
A multiple choice test contains 10 questions. There are four possible answers
for each question.
\begin{enumerate}[label=(\alph*)]
 \item How many ways can a student complete the test if every question 
 must be answered?
 
 \item How many ways can a student complete the test if questions 
 can be left unanswered?
\end{enumerate}
\end{prob}

\begin{prob}
Computer ID's are length seven strings made up of any combination of seven different letters and digits. How many different ID's are there? 
\end{prob}

\begin{prob}
Computer ID's are length seven strings made up of any combination of seven letters and digits, with repeats allowed. How many different ID's are there? 
\end{prob}

\begin{prob}A code word is either a sequence of three letters followed
by two digits or two letters followed by three digits. (Unless otherwise
indicated, {\itshape letters} will means
 upper case letters chosen from the usual
$26$-letter alphabet and {\itshape digits} are selected from 
$\{0,1,2,3,4,5,6,7,8,9\}$.) How many different code words are possible?
\end{prob}

\begin{prob}
Code words consist of five letters followed by five digits. How many code word contain at least one $X$?
\end{prob}

\begin{prob}
Code words consist of five letters followed by five digits. How many code word contain exactly one $X$?
\end{prob}


\begin{prob}
Code words consist of five letters followed by five digits. How many code word contain exactly two $X$'s?
\end{prob}


\begin{prob}
How many bit strings of length ten begin and end with $1$'s?
\end{prob}

\begin{prob}
How many bit strings of length t least two but no more than ten begin and end with $1$'s?
\end{prob}

\begin{prob}
There are five roads from $A$ to $B$, three roads from $B$ to $C$, and five roads from $C$ to $D$. How many different routes are there 
from $A$ to $B$ to $C$ to $D$?
\end{prob}

\begin{prob}
There are five roads from $A$ to $B$, three roads from $B$ to $C$, and five roads from $C$ to $D$. How many different round trip routes are there 
from $A$ to $D$ and back to $A$?
\end{prob}

\begin{prob}
 There are five roads from $A$ to $B$, three roads from $B$ to $C$, and five roads from $C$ to $D$. How many different round trip routes are there 
from $A$ to $D$ and back to $A$ if you cannot travel over any road more than once?
\end{prob}
