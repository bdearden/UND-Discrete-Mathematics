\chapter{GCD's and the Euclidean Algorithm}


\newthought{The {\bfseries greatest common divisor} of $a$ and $b$, not both $0$,} is the largest
integer which divides both $a$ and $b$. For example, the greatest common divisor
of $21$ and $35$ is $7$.
We write $\gcd(a,b)$, as shorthand for the greatest common divisor of $a$ and $b$. So 
$\gcd(35,21) = 7$. 

There are several ways to find the $\gcd$ of two integers, $a$ and $b$ (not both $0$).

First, we could simply list all the positive divisors of $a$ and $b$ and pick the largest
number that appears in both lists. Notice that $1$ will appear in both lists.
For the example above the positive divisors of $35$ are $1$, $5$, $7$, and $35$. For $21$
the positive divisors are $1$, $3$, $7$, and $21$. The largest number appearing in both lists
is $7$, so $\gcd(35,21) = 7$. 

Another way to say the same thing: If we let $D_a$ denote the set of positive divisors
of $a$, then $\gcd(a,b) = $ the largest number in $D_a\cap D_b$.

The reason $\gcd(0,0)$ is not defined is that every positive integer divides $0$, and so there
is no largest integer that divides $0$. From now on, when we use the symbol $\gcd(a,b)$,
we will tacitly assume $a$ and $b$ are not both $0$. The integers $a$ and $b$ can be
negative.
For example if $a=-34$ and $b=14$, then the set of positive divisors of $-34$
is $\{1,2,17,34\}$ and the set of positive divisors of $14$ is $\{1,2,7,14\}$. The set of positive common divisors
of $14$ and $-34$ is the set $\{1,2,17,34\}\cap \{1,2,7,14\}=\{1,2\}$. The largest number in this
 set is $2=\gcd(-34,14)$.

Obviously then $\gcd(a,b)=\gcd(-a,b)$ since $a$ and $-a$ have the same set of positive divisors.
So when computing the $\gcd(a,b)$ we may as well replace $a$ and $b$ by their absolute values
if one or both happen to be negative.

Here are a few easy facts about $\gcd$'s: 
\begin{enumerate}
 \item If $a\not = 0$, then  $\gcd(a,a) = a$.
 
 \item  $\gcd(a,1) = 1$.
 
 \item  $\gcd(a,b) = \gcd(b,a)$. 
 (The order $a$ and $b$ are given is not important,\\
 but it is traditional to list them with $a\geq b$.)
 
 \item If $a\not = 0$ and $a|b$, then $\gcd(a,b)=|a|$. 
 
 \item If $a\neq 0$, $\gcd(a,0)=|a|$.
\end{enumerate}


If $\gcd(a,b)= 1$, we say that $a$ and $b$ are {\bfseries relatively prime}\label{def:rel prime}. When $a$ and $b$ are relatively prime, they
have no common prime divisor. For example $12$ and $35$ are relatively prime. 

\section{Euclidean algorithm}
It's pretty clear that computing $\gcd(a,b)$ by listing all the positive visors of $a$ and all the positive divisors of $b$, and selecting the largest integers that appears in both lists is not very efficient.  
There is a better way of 
computing $\gcd(a,b)$.

\begin{thm}
If $a$ and $b$ are integers (not both $0$)  and $a=sb+t$ for integers $s$ and $t$, 
then $\gcd(a,b)=\gcd(b,t)$.
\end{thm}
\begin{proof}
To prove the theorem, we will show that the list of positive integers that 
divide both $a$ and $b$
is identical to the list of positive integers that divide both $b$ and $t=a-sb$. So, suppose 
$d|a$ and $d|b$. Then $d|(a-sb)$ so $d|t$. Hence $d$ divides both $b$ and $t$. On the other
hand, suppose $d|b$ and $d|t$. Then $d|(sb+t)$, so that $d|a$. Hence $d$ divides both
$a$ and $b$. It follows that $\gcd(a,b)=\gcd(b,t)$.
\end{proof}


Euclid is given the credit for discovering this fact, and its use for computing $\gcd$'s is
called the {\bfseries Euclidean algorithm} in his honor. The idea is to
use the theorem repeatedly until a pair of numbers is reached for which the $\gcd$
is obvious.
Here is an example of
the Euclidean algorithm in action.

\begin{exmp}
Since $14=1\cdot 10 + 4$, $\gcd(14,10)=\gcd(10,4)$. In turn $10=2\cdot 4+2$ so
$\gcd(10,4)=\gcd(4,2)$. Since $4=2\cdot 2$, $\gcd(4,2)=\gcd(2,0)=2$. So $\gcd(10,14)=2$.

The same example, presented a little more compactly, and without explicitly 
writing out the divisions, looks like
$$
\gcd(14,10) = \gcd(10,4) = \gcd(4,2) = \gcd(2,0) = 2
$$
At each step, the second number is replaced by the remainder when the 
first number is divided by the second, and the second moves into the first spot.
The process is repeated until the second number is a $0$ (which must happen eventually
since the second number never will be negative, and it goes down by at least $1$ with each
repetition of the process). The $\gcd$ is then the number in the first spot when the second
spot is $0$ in the last step of the algorithm.
\end{exmp}

Now, a more exciting example.
\begin{exmp}
Find the greatest common divisor of $540$ and $252$. We may present the computations compactly,
without writing\sidenote{Do the divisions yourself to verify the results.} out the divisions. 
We have
\[\gcd(540,252)=\gcd(252,36)=\gcd(36,0) = 36.
\]
\end{exmp}


\section{Efficiency of the Euclidean algorithm}
Using the Euclidean algorithm to find $\gcd$'s is extremely efficient. Using a calculator
with a ten digit display, you can find the $\gcd$ of two ten digit integers in a matter of a few
minutes at most using the Euclidean algorithm. On the other hand, doing the same problem
by first finding the positive divisors of the two ten digit integers would be a tedious
project lasting several days. Some modern cryptographic systems rely on the computation
of the gcd's of integers of hundreds of digits. Finding the positive divisors of such
large integers, even with a computer, is, at present, a hopeless task. But a computer
 implementation of
the  Euclidean  algorithm will produce the gcd of integers of hundreds of digits in the blink
of an eye. 



\section{The Euclidean algorithm in quotient/remainder form}
The Euclidean algorithm can also be written out as a sequence of divisions: 
\begin{align*}
 a &= q_1\cdot b + r_1, ~~0< r_1 < b \\
 b &= q_2\cdot r_1 + r_2, ~~0< r_2 < r_1 \\
 r_1 &= q_3\cdot r_2 + r_3, ~~0< r_3 < r_2 \\
 \vdots & = \vdots  \\
 r_k &= q_{k+2}\cdot r_{k+1} + r_{k+2}, ~~0< r_{k+2} < r_{k+1} \\
 \vdots & = \vdots  \\
 r_{n-2} &= q_n\cdot r_{n-1} + r_n, ~~0< r_n < r_{n-1} \\
 r_{n-1} &= q_{n+1}\cdot r_n + 0
\end{align*}
The sequence of integer remainders $b>r_1>...>r_k>...\geq 0$ must eventually reach $0$. 
Let's say $r_n\not=0$, but $r_{n+1}=0$, so that $r_{n-1}=q_{n+1}\cdot r_n$. 
That is, in the sequence of remainders, $r_n$ is the last non-zero term. 
Then, just as in the examples above we see that the $\gcd$ of $a$ and $b$ is the last nonzero
remainder:
\begin{align*}
 \gcd(a,b) = \gcd(b,r_1) = \gcd(r_1,r_2) =\cdots&= \gcd(r_{n-1},r_n) \\
 & = \gcd(r_n,r_{n+1})
 = \gcd(r_n,0) = r_n.
\end{align*}
Let's find $\gcd(317,118)$ using this version of the Euclidean algorithm.  Here are the
steps:
\begin{align*}
 317 &= 2\cdot118+81 \\
 118 &= 1\cdot81+37\\
  81 &= 2\cdot37+7\\
  37 &= 5\cdot7+2 \\
  7  &= 3\cdot2+1 \\
  2  &= 2\cdot1+0
\end{align*}
Since the last non{-}zero remainder is $1$, we conclude that
$\gcd(317,118)=1$. So, in the terminology introduced above, we would say that
$317$ and $118$ are relatively prime.




\clearpage

\section{Exercises}

\begin{exer}
Use the Euclidean algorithm to compute $\gcd(a,b)$ in each case.\\
\no a) $a=233,\,b=89$\hskip .3in b) $a=1001,\, b=13$\hskip .3in c) $a=2457,\,b=1458$\hskip .3in d) $a=567,\,b=349$
\end{exer}

\begin{exer}
Compute $\gcd(987654321,123456789)$.
\end{exer}

\begin{exer}
 Write a step-by-step  algorithm 
that implements the Euclidean algorithm for finding $\gcd$'s.
\end{exer}

\begin{exer}
If $n$ is a positive integer, what is $\gcd(n, 2n)$?
\end{exer}

\section{Problems}

\begin{prob}

Use the Euclidean algorithm to compute $\gcd(a,b)$ in each case.\\

\no a) $a=216,\,b=111$\hskip .3in b) $a=1001,\, b=11$\hskip .3in c) $a=663,\,b=5168$\hskip .3in d) $a=1357,\,b=2468$
\end{prob}

\begin{prob}
Compute $\gcd(733103,91637)$.
\end{prob}

\begin{prob}
If $p$ is a prime, and $n$ is any integer, what are the
possible values of $\gcd(p,n)$?
\end{prob}

\begin{prob}
Prove or give a counterexample: If $p$ and $q$ are two different primes, then $\gcd(2p,2q) = 2$.
\end{prob}

\begin{prob}
If $p$ is a prime, and $m$ is a  positive integer, determine $\gcd(p, p^{m})$.
\end{prob}

\begin{prob}
If $p$ is a prime, and $m\leq n$ are positive integers, determine $\gcd(p^{m}, p^{n})$.
 \end{prob}
 
 \begin{prob}
 Show that if $n$ is a positive integer, then $\gcd(n, n+1) = 1$.
 \end{prob}

