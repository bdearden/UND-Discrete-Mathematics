    \section*{Chapter 20}
    
\begin{Solution}{20.1}
\underbar{Proof}:\\
Suppose $a>0$ and $b>0$. Since $a>0$, multiplying both sides of $b>0$ by $a$ gives $ab>a0$. We know $a0 = 0$. It follows that $ab>0$. $\clubsuit$\\[3pt]


\end{Solution}

\begin{Solution}{20.2}

\underbar{Proof}:\\
Suppose neither $a$ nor $b$ is $0$. Consider four cases:
\begin{enumerate}

\item {\it $a>0$ and $b>0$}: In this case, we proved $ab>0$ in Exercise 1. In particular then, $ab\not=0$ in this case.

\item {\it $a>0$ and $b<0$}:  Since $a>0$, multiplying both sides of $b<0$ by $a$ gives $ab<a0$. We know $a0 = 0$. It follows that $ab<0$. So $ab\not = 0$ in this case.

\item {\it $a<0$ and $b>0$}: Since $a<0$, multiplying both sides of $b>0$ by $a$ gives $ab<a0$. We know $a0 = 0$. It follows that $ab<0$. So $ab\not = 0$ in this case.

\item {\it $a<0$ and $b<0$}: Since $a<0$, multiplying both sides of $b<0$ by $a$ gives $ab>a0$. We know $a0 = 0$. It follows that $ab>0$. In particular then, $ab\not=0$ in this case.
\end{enumerate}
So, in any case, if neither $a$ nor $b$ is $0$, then $ab\not =0$. $\clubsuit$\\[5pt]


\end{Solution}

\begin{Solution}{20.3}
\underbar{Proof}:\\
Suppose $c\not= 0$ and $ac = bc$. We can rewrite $ac = bc$ as $ac - bc = 0$, and then use the distributive property to write that as $(a-b)c = 0$. Applying the result of Exercise 2, we conclude either $a-b = 0$ or $c=0$.
Since $c\not =0$, it must be that $a-b = 0$. Adding $b$ to each side of that equation shows $a=b$. $\clubsuit$

\end{Solution}


