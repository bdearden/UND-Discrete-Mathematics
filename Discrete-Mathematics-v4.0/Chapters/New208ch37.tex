\chapter{Solving Nonhomogeneous Recurrences}\label{ch:Solving Nonhomogeneous Recurrences}

\newthought{When a linear recurrence relation} with constant coefficients for a
sequence $\{s_n\}$ looks like
$$s_n= c_1s_{n-1}+c_2s_{n-2}+\cdots+c_ks_{n-k} + f(n),$$
where $f(n)$ is some (nonzero) function of $n$, then the recurrence
relation  is said to be {\bfseries nonhomogeneous}. For example, $s_n =
2s_{n-1}+n^2+1$ is a nonhomogeneous recurrence. Here $f(n) = n^2+1.$
The methods used in the last chapter are not adequate to deal with
nonhomogeneous problems. But it wasn't all a waste since those methods
do provide one step in the solution of nonhomogeneous problems.


\section{Steps to solve nonhomogeneous recurrence relations}
% The 
%steps used to solve nonhomogeneous linear recurrence relations with 
%constant coefficients are:
\begin{enumerate}[label=Step (\arabic*):]
 \item Replace the $f(n)$ by $0$ to create a
 homogeneous recurrence relation,
 $$s_n= c_1s_{n-1}+c_2s_{n-2}+\cdots+c_ks_{n-k}.$$
 Now solve this and write down the general solution\sidenote{We learned
 to do this in chapter~\ref{chpt:char roots}.}. For example, in the case of no repeated roots,  
 the general solution will look something
 like:
 $$s_n=a_1r_1^n+a_2r_2^n+\cdots+a_kr_k^n,$$
 where the constants $a_1,a_2,\cdots,a_k$ are to be determined.
 
\clearpage
 \item Next, find one {\bfseries particular solution} to the
 original nonhomogeneous recursion.  In other words, one
 specific sequence that obeys the recursive formula (ignoring the 
 initial conditions).  A method for finding a particular
 solution that works in many cases is to guess! Actually, it is to make
 an educated guess. Reasonable guesses depend on the form of $f(n)$. 
 There is an algorithm that will produce the correct guess,
 but it is so complicated it isn't worth learning for the few simple
 examples we will be doing. Instead, rely on the following guidelines
 to guess the form of a  particular solution.
 
 Roughly, the plan is the guess a particular solution that is the most general
 function of the same type as $f(n)$. Specifically, table~\ref{tabl:particular solns}
 shows reasonable guesses.
 \begin{margintable}
 \centering
 \begin{tabular}{cl}
  \toprule
  $f(n)$ 
  & Particular Solution Guess \\
  \midrule
  $c$ (a constant)      & $A$ (constant) \\
  $n$                   & $An+B$ \\
  $n^2$                 & $An^2+Bn+C$ \\
  $n^3$                 & $An^3+Bn^2+Cn+D$ \\
  $2^n$                 & $A2^n$ \\
  $r^n$ ($r$ constant)  & $Ar^n$ \\
  \bottomrule\addlinespace
 \end{tabular}\caption{Particular solution patterns}\label{tabl:particular solns}
 \end{margintable}
 
 
 These guesses can be {\itshape mixed-and-matched}. For example, if 
 \[
  f(n)= 3n^2+5^n,
 \] then a reasonable candidate particular solution
 would be 
 \[
  An^2+Bn+C+D5^n. 
 \]
 
 Once a guess has been made for the form of a particular solution, that
 guess is plugged into the recurrence relation, and the coefficients
 $A,B,\cdots$ are determined. In this way a specific particular solution
 will be found.
 
 It will sometimes happen that when the equations are set up to determine
 the coefficients of the particular solution, an inconsistent system
 will appear. In such a case, as with repeated characteristic roots, the
 trick is (more-or-less) to multiply the guess for the particular solution by $n$, and
 try again.
 
 \item Once a particular solution has been found, add the particular 
 solution of step (2) to the general solution of the
 homogeneous recurrence found in step (1).  If we denote a particular
 solution by $h(n)$, then the total general solution looks like
 \[
  s_n=a_1r_1^n+a_2r_2^n+\cdots+a_kr_k^n+h(n).
 \]
 
 \item Invoke the initial conditions to
 determine the values of the coefficients
  $a_1,a_2,\cdots,a_k$
 just as we did for the
 homogeneous problems in chapter~\ref{chpt:char roots}.  
 
\end{enumerate}





The major oversight made solving
a nonhomogeneous recurrence relation is trying to determine the coefficients
 $a_1,a_2,\cdots,a_k$ before
the particular solution is added to the general solution. This mistake
will usually lead to inconsistent information about the coefficients,
and no solution to the recurrence will be found.


\section{Examples}
\begin{exmp}
 Let's solve the Tower of Hanoi recurrence using this
 method.
 
 The recurrence is $H_0=0$, and, for $n\geq 1$, $H_n = 2H_{n-1}+1$.
 We know the closed form formula for $H_n$ is $2^n-1$ already, but let's
 work it out using the method outlined above.
 \begin{enumerate}[label=Step (\arabic*):]
   \item Find the general solution of related homogeneous recursion 
   (indicated by the superscript $(h)$):
   $H_n^{(h)} = 2H_{n-1}^{(h)}$. That will be $H_n^{(h)} = A2^n$.
   
   \item Guess the particular solution (indicated by superscript $(p)$): 
   $H_n^{(p)} = B$, a constant. Plugging that guess into the
   recurrence gives  $B = 2B+1$, and so we see $B=-1$.
   
   \item Hence, the general solution to the Tower of Hanoi recurrence is
   \[
    H_n = H_n^{(h)}+H_n^{(p)} = A2^n -1.
   \]
   
   \item Now, use the initial condition to determine $A$: When $n=0$, we
   want
   $0 = A2^0-1$ which means $A=1$. Thus, we find the expected result:
   \[
    H_n = 2^n-1, \text{ for $n\geq 0$}.
   \]
 \end{enumerate}
\end{exmp}

\clearpage
\begin{exmp}
 Here is a more complicated example worked out in 
 detail to exhibit the
 method. Let's solve the recurrence
\begin{align*}
  s_1 &=2,\quad s_2=5 \quad\text{ and}, \\
  s_n &=s_{n-1}+6s_{n-2}+3n-1, \quad\text{ for $n\geq3.$}
\end{align*}
\begin{enumerate}[label=Step (\arabic*):]
  \item Find the general solution of $s_n=s_{n-1}+6s_{n-2}.$ After
  finding the characteristic equation, and the characteristic roots, the
  general solution turns out to be $s_n=a_13^n+a_2(-2)^n$.
  
  \item To find a particular solution let's guess that there is a
  solution $h(n)$ that looks like $h(n)=an+b$, where $a$ and $b$ are to be
  determined. To find values of $a$ and $b$ that work, we substitute this
  guess for a solution into the original recurrence relation. In this
  case, the result of plugging in the guess $(s_n=h(n)=an+b)$ gives us:
  \[
   an+b=a(n-1)+b+6(a(n-2)+b)+3n-1.
  \]
  which can be rearranged to
  \[
   (6a+3)n+(-13a+6b-1)=0.
  \]
  If this equation is to be correct for all $n$, then, in particular, it
  must be correct when $n=0$ and when $n=1$, and that tells us that
  \begin{align*}
       -13a+6b-1&=0 \quad\text{ and}, \\
   6a+3-13a+6b-1&=0.
  \end{align*} 
  Solving this pair of equations we find $a=-{1\over2}$ and
  $b=-{{11}\over{12}}.$ And, sure enough, if you plug this alleged
  solution into the original recurrence, you will see it checks.
  
  \item Write down the general solution to the original nonhomogeneous
  problem by adding the particular solution of step (2) to the general
  solution from step~(1) getting:
  \[
   s_n=a_13^n+a_2(-2)^n+\left(-\frac{1}{2}\right)n+\left(-{\frac{11}{12}}\right).
  \]
  
  
  \item Now $a_1,a_2$ can be calculated: For $n=1$, the first initial condition
  gives 
  \begin{align*}
    2 &=a_13^1+a_2(-2)^1+ \left(-{1\over2}\right)1+\left(-{{11}\over{12}}\right), \\
    \intertext{and for $n=2$, we get} 
    5 &=a_13^2+a_2(-2)^2+ \left(-{1\over2}\right)2+\left(-{{11}\over{12}}\right).
  \end{align*}
   Solving these two equations for $a_1$ and $a_2$, we find that $a_1=
  {{11}\over{12}}$ and $a_2=-{1\over3}$.\par
  So the solution to the recurrence is
  \[
   s_n={\frac{11}{12}}3^n-{\frac{1}{3}}(-2)^n+\left(-\frac{1}{2}\right)n+\left(-{\frac{11}{12}}\right).
  \]
\end{enumerate}
 

\end{exmp}
\clearpage
\section{Exercises}
{\itshape Use the general solutions for the
related homogeneous problems of chapter~\ref{chpt:char roots} to help solve
the following nonhomogeneous recurrence relations with initial conditions.}

\begin{exer}
 $a_0=3, a_1=6$ and $a_n=a_{n-1}+6a_{n-2}+1,$ for $n\geq 2$.
\end{exer}

\begin{exer}
$a_2=5, a_3=13$ and $a_n=7a_{n-1}-10a_{n-2}+n,$ for $n\geq 4$.
\end{exer}

\begin{exer}
$a_1=3, a_2=5$ and $a_n=4a_{n-1}-4a_{n-2}+2^n$, for $n\geq 3$.
\end{exer}

\begin{exer}
$a_0=1, a_1=6$ and $a_n=6a_{n-1}-9a_{n-2}+n,$ for $n\geq 2$.
\end{exer}

\begin{exer}
$a_0=2, a_1=5, a_2 = 15$, and 
$a_n=6a_{n-1}-11a_{n-2}+6a_{n-3}+2n+1$, for $n\geq 3$.
\end{exer}

\section{Problems}

\begin{prob}
Solve $a_0=1$, and $a_n=2a_{n-1} + 1,$ for $n\geq 1$. 
\end{prob}

\begin{prob}
Solve $a_0=2, a_1=5$ and $a_n=a_{n-1}+6a_{n-2} + 2,$ for $n\geq 2$.
\end{prob}


\begin{prob}
$a_0=3,a_1=7,$ and $a_n=6a_{n-1}-5a_{n-2} + n,$ for $n\geq 2$.

\end{prob}


\begin{prob}
$a_2=5, a_3=13,$ and $a_n=3a_{n-1}+10a_{n-2} + n+ 2,$ for $n\geq 4$.
\end{prob}


\begin{prob}
$a_1=3, a_2=5,$ and $a_n=8a_{n-1}-16a_{n-2} n^{2},$ for $n\geq 3$.
\end{prob}


\begin{prob}
$a_1=2, a_2=8$, and $a_n=4a_{n-2} + 2^{n}$, for $n\geq 3$.
\end{prob}


\begin{prob}
$a_0=0, a_1=1, a_2 = 2$, and $a_n=-a_{n-1}+4a_{n-2}+4a_{n-3}+ 2n$, for $n\geq 3$.
\end{prob}


\begin{prob}
$a_0=0$, $a_1=1$, and
for $n\geq 2$, $a_n=2a_{n-1}+a_{n-2}+1$.

\end{prob}






