\chapter{Logical Equivalence} 
\newthought{It is clear that} the propositions {\itshape It is sunny and it is warm} and
{\itshape It is warm and it is sunny} mean the same thing. More generally, for any
propositions $p$, $q$, we see that $p\land q$ and $q\land p$ have the same meaning. To say it a little differently,
for any choice of truth values for $p$ and $q$, the propositions  $p\land q$ and 
$q\land p$ have the
same truth value. One more time: $p\land q$ and $q\land p$ have identical truth tables.

\section{Logical Equvalence}
Two propositions with identical truth values are called {\bfseries logically equivalent}.
The expression $p\equiv q$ means $p,q$ are logically equivalent. 

Some logical equivalences are not as transparent as the example above. With a little thought
it should be clear that {\itshape I am not taking math or I am not taking physics} means the same as
{\itshape It's not the case that I taking math and physics}. In symbols, $(\lnot m)\lor(\lnot p)$ means the\marginnote[-1.5cm]{To be convinced these two proposition really have the same content,
look at the truth table, (\ref{tbl: De Morgan's Law}), for the two propositions, and notice that the final truth values are identical.}
same as $\lnot(m\land p)$.

\begin{exmp}[De Morgan] Prove that
$\lnot(p\land q)\equiv (\lnot p \lor \lnot q)$ using a truth table. We construct the truth table \ref{tbl: De Morgan's Law} in the order
or precedence: $\lnot$ before $\land$ or $\lor$, but the expresion in parentheses has highest precedence.
%We construct the table using additional columns for compound parts of the two expressions.

\begin{margintable}[-1.0cm]
\begin{tabular}{@{ }c@{ }@{ }c | c@{ }@{}c@{}@{ }c@{ }@{ }c@{ }@{ }c@{ }@{}c@{ } | c@{ }@{ }c@{ }@{ }c@{ }@{ }c@{ }@{ }c@{ }@{ }c@{ }@{ }c}
p & q & $\lnot$ & ( & p & $\land$ & q & ) &  & $\lnot$ & p & $\lor$ & $\lnot$ & q & \\
\hline 
T & T & \textcolor{red}{F} &  & T & T & T &  &  & F & T & \textcolor{red}{F} & F & T & \\
T & F & \textcolor{red}{T} &  & T & F & F &  &  & F & T & \textcolor{red}{T} & T & F & \\
F & T & \textcolor{red}{T} &  & F & F & T &  &  & T & F & \textcolor{red}{T} & F & T & \\
F & F & \textcolor{red}{T} &  & F & F & F &  &  & T & F & \textcolor{red}{T} & T & F & \\
\end{tabular}\label{tbl: De Morgan's Law}
\caption{De Morgan's Law}
\end{margintable}
\end{exmp} 

It is probably a little harder to believe
 $(p\to q)\equiv (\lnot p\lor q)$, but checking a truth table shows they are in fact equivalent.
Saying {\itshape If it is Monday, then I am tired} is identical to saying
{\itshape It isn't Monday or I am tired.} Complete table \ref{tbl: p->q=not pvq} to demonstrate their
equivalence.
\begin{margintable}
\begin{tabular}{@{ }c@{ }@{ }c | c@{ }@{ }c@{ }@{ }c@{ }@{ }c@{ }@{ }c | c@{ }@{ }c@{ }@{ }c@{ }@{ }c@{ }@{ }c@{ }@{ }c}
p & q &  & p & $\rightarrow$ & q &  &  & $\sim$ & p & $\lor$ & q & \\
\hline 
T & T &  & T &  & T &  &  &  & T &  & T & \\
T & F &  & T &  & F &  &  &  & T &  & F & \\
F & T &  & F &  & T &  &  &  & F &  & T & \\
F & F &  & F &  & F &  &  &  & F &  & F & \\
\end{tabular}
\caption{Prove $p\to q \equiv \lnot p\lor q$}
\label{tbl: p->q=not pvq}
\end{margintable}


\section{Tautologies and Contradictions}
A proposition , $\mathbb{T}$, which is always true is called a {\bfseries tautology}. A {\bfseries contradiction} is a 
proposition, $\mathbb{F}$,
which is always false.  The prototype example of a tautology is $p\lor \lnot p$, and for a contradiction,
$p\land \lnot p$.  Notice that since $p\iff q$ is $T$ exactly when $p$ and $q$ have the same
truth value,  two propositions $p$ and $q$ will be logically equivalent provided
 $p\iff q$ is a tautology. 

\section{Related \textbf{If \textellipsis, then \textellipsis} propositions} 
There are three propositions related to the basic \textbf{If \textellipsis, then \textellipsis} implication: $p\to q$. First  
$\lnot q\to \lnot p$  is called the {\bfseries contrapositive}
of the implication. The {\bfseries converse} of the implication is the
proposition $q\to p$. Finally, the {\bfseries inverse} of the implication is  $\lnot p\to \lnot q$. 
Using a truth table, it is easy to check that an implication and its contrapositive are
logically equivalent, as are the converse and the inverse. A common slip is to think
the implication and its converse are logically equivalent. Checking a truth table shows that
isn't so. The implication {\itshape If an integer ends with a $2$, then it is even}
is $T$, but its converse, {\itshape If an integer is even, then it ends with a $2$},
is certainly $F$. 

\section{Fundamental equivalences}
Table \ref{tbl:fund_eqvs} contains the most often used equivalences.
 These are well worth learning by sight and by name. 
\begin{table}\label{tbl:fund_eqvs}
\begin{tabular}{ c  c }
\hline
\textbf{Equivalence}  & \textbf{Name} \\
\hline
$\lnot (\lnot p)\equiv p$ & \hbox{Double Negation} \\
\hline
$p\land \mathbb{T} \equiv p$ & {\lower 5pt\hbox{Identity laws}} \\
$p\lor \mathbb{F} \equiv p$ &  \\
\hline
$p\lor \mathbb{T} \equiv \mathbb{T}$ &  {\lower 5pt\hbox{Domination laws}}\\
$p\land \mathbb{F} \equiv \mathbb{F}$ & \\
\hline
$p\lor p \equiv p$ &  {\lower 5pt\hbox{Idempotent laws}}\\
$p\land p\equiv p$ & \\
\hline
$p\lor q \equiv q\lor p$ &  {\lower 5pt\hbox{Commutative laws}}\\
$p\land q \equiv q\land p$ & \\
\hline
$(p\lor q)\lor r \equiv p\lor (q\lor r)$ &  {\lower 5pt\hbox{Associative laws}}\\
$(p\land q)\land r\equiv p\land (q\land r)$ & \\
\hline
$p\lor (q\land r)\equiv (p\lor q)\land (p\lor r)$ &  
{\lower 5pt\hbox{Distributive laws}} \\
$p\land (q\lor r) \equiv (p\land q) \lor (p\land r)$ & \\
\hline
$\lnot (p\land q) \equiv (\lnot p\lor \lnot q)$ 
& {\lower 5pt \hbox{De Morgan's laws}} \\
$\lnot (p\lor q) \equiv (\lnot p \land \lnot q)$ & \\
\hline
$p\lor \lnot p \equiv \mathbb{T}$ &\hbox{ Law of Excluded Middle} \\
$p\land \lnot p \equiv \mathbb{F}$ & \hbox{Law of Contradiction} \\
\hline
$p \to q \equiv \lnot p \lor q$ & \text{Disjunctive form} \\ 
\hline
$p \to q \equiv \lnot q \to \lnot p$ & \text{Implication $\equiv$ Contrapositive}\\
$\lnot p \to \lnot q \equiv q \to p$ & \text{Inverse $\equiv$ Converse} \\
\hline
\end{tabular}
\caption{Logical Equivalences}
\end{table}

\section{Disjunctive normal form}%
Five basic connectives have been given: $\lnot, \land, \lor, \to, \iff$, but that is really just
for convenience. It is possible to eliminate some of them using logical equivalences.
For example, $p\iff q \equiv (p\to q)\land (q\to p)$ so there really is no need to
explicitly use the biconditional. Likewise, $p\to q \equiv \lnot p\lor q$, so the use
of the implication can also be avoided. Finally, $p\land q \equiv \lnot(\lnot p \lor\lnot q)$
so that there really is no need ever to use the connective $\land$. Every proposition
made up of the five basic connectives can be rewritten using only $\lnot$ and $\lor$ (probably
with a great loss of  clarity however). 

The most often used standardization, or normalization, of logical propositions is the
\textbf{disjunctive normal form (DNF)}, using only $\lnot$ (negation), $\land$ (conjunction), 
and $\lor$ (disjunction). A propositional form is considered to be in DNF if and only if it is a disjunction of one or more conjunctions of one or more {\it literals} (a {\itshape literal} is a letter or a letter preceded by the negation symbol). For example, the following are
all in disjunctive normal form:
\begin{itemize}
\item $p \land q$
\item $p$
\item $(a \land q) \lor r$
\item $(p \land \lnot q \land \lnot r) \lor (\lnot s \land t \land u)$
\end{itemize}
While, these are \textbf{not} in DNF:
\sidenote{
Use the fundamental equivalences to find DNF versions of each.}
\begin{itemize}
\item  $\lnot(p \lor q)$ \em this is \textbf{not} the disjunction of literals.
\item $p \land (q \land (r \lor s))$ \em an \text{or} is embedded in a conjunction.
\end{itemize}

\section{Proving equivalences}
It is always possible the verify a logical equivalence via a truth table. But it also possible to
verify equivalences by stringing together previously known equivalences. Here are
two examples of this process.

\begin{exmp}
\label{exmp:formalproof1}
Show $\lnot(p\lor(\lnot p\land q))\equiv \lnot p\land\lnot q$. %
\sidenote{ %
The plan is to start with the expression $\lnot(p\lor(\lnot p\land q))$, work
through a sequence of equivalences ending up with $\lnot p\land\lnot q$. It's pretty
much like proving identities in algebra or trigonometry.}
\begin{proof}
\begin{align*}
\lnot(p\lor(\lnot p\land q))
&\equiv \lnot p \land \lnot(\lnot p\land q)         &\text{De Morgan's Law}\\
&\equiv \lnot p \land (\lnot(\lnot p)\lor\lnot q)   &\text{De Morgan's Law}\\
&\equiv \lnot p\land(p\lor\lnot q)                  &\text{Double Negation Law}\\
&\equiv (\lnot p\land p)\lor(\lnot p\land \lnot q)  &\text{Distributive Law}\\
&\equiv ( p\land \lnot p)\lor(\lnot p\land \lnot q) &\text{Commutative Law}\\
&\equiv \mathbb{F} \lor (\lnot p \land \lnot q)     &\text{Law of Contradiction}\\
&\equiv  (\lnot p \land \lnot q)\lor\mathbb{F}      &\text{Commutative Law}\\
&\equiv \lnot p\land\lnot q                         &\text{Identity Law}
\end{align*}

\end{proof}
\end{exmp}

\clearpage
\begin{exmp}
\label{exmp:formalproof2}
Show $(p\land q)\to(p\lor q)\equiv\mathbb{T}$.

\begin{proof}
\begin{align*}
(p\land q)\to(p\lor q)
&\equiv\lnot(p\land q)\lor(p\lor q)         &\text{Disjunctive form}\\
&\equiv (\lnot p\lor\lnot q)\lor(p\lor q)   &\text{De Morgan's Law}\\
&\equiv (p\lor \lnot p)\lor(q \lor \lnot q) &\text{Associative and Commutative Laws}\\
&\equiv \mathbb{T}\lor\mathbb{T}            &\text{Commutative Law and Excluded Middle}\\
&\equiv \mathbb{T}                          &\text{Domination Law}
\end{align*}
\end{proof}

\end{exmp}


\clearpage

\section{Exercises}

\begin{exer}
 Use truth tables to verify each of the following equivalences:
\begin{tasks}(2)
	\task $(p\lor q)\lor r \equiv p\lor (q\lor r)$
	\task $\lnot p\land(p\lor q)\equiv \lnot(q \to p)$
	\task $p\lor (q\land r)\equiv (p\lor q)\land (p\lor r)$
	\task 
	\end{tasks}
\end{exer}

\begin{exer}
Show that the statements are not logically equivalent. 
\begin{enumerate}[label= \alph*)]
\item $p\land (q\to r)\not\equiv (p \land q)\to r$

\item $p\to q\not\equiv q\to p$

\item $p\to q\not\equiv \lnot p\to \lnot q$
\end{enumerate}
\end{exer}

\begin{exer}
Use truth tables to show that the following are tautologies.
\begin{enumerate}[label= \alph*)]
\item $[p\land (p\to q)]\to q$

\item $[(p\to q)\land (q\to r)]\to (p\to r)$

\item $[(p\lor q)\to r]\to [(p\to r) \land (q\to r)]$
\end{enumerate}
\end{exer}

\begin{exer} Consider the implication {\itshape If it is Saturday, then I will mow the lawn}.
\label{exer:conincon} 
\begin{enumerate}[label= \alph*)]
\item Write the converse of the implication.
\item Write the inverse of the implication.
\item Write the contrapositive of the implication.
\end{enumerate}
\end{exer}

\begin{exer} Consider the three answers for exercise \ref{exer:conincon}.
One of the three is logically equivalent to the implication. Which one?
Two of the three are not logically equivalent to the implication, but are
logically equivalent to each other. Which two?
\end{exer}

\begin{exer}
The statements below are not tautologies. In each case, find an assignment of truth values to the literals, (that is, a letter or a letter preceded
by the negation symbol), so the statement is false.
\begin{enumerate}[label= \alph*)]
\item $[(p\land q)\to r]\longleftrightarrow [(p\to r)\land (q\to r)]$

\item $[(p\land q)\lor r]\to [p\land (q\lor r)]$
\end{enumerate}
\end{exer}

\begin{exer}
Give proofs of the following equivalences using the Fundamental Logical Equivalences, following the pattern
of examples \ref{exmp:formalproof1} and \ref{exmp:formalproof2}.
\begin{enumerate}[label= \alph*)]
\item $(\lnot p \land (p \lor q))\to q \equiv \mathbb{T}$.

\item $(p\land\lnot r)\to\lnot q \equiv p\to(q\to r)$.

\item $p\lor(p\land q)\equiv p$. (This is a tough one.)\label{setablaw}
\end{enumerate}
\end{exer}

\section{Problems}

\begin{prob}
Use a truth table to show $p\to q\equiv \lnot p \lor q$.
\end{prob}

\begin{prob}
Use a truth table to show $(p\land q)\to r \equiv p\to (q\to r)$.
\end{prob}

\begin{prob}
Use a truth table to show $p\to q\equiv \lnot q\to \lnot p$.
\end{prob}

\begin{prob} Consider the implication {\itshape If I work, then I get paid. }
\label{prob:conincon} 
\begin{enumerate}[label= \alph*)]
\item Write the converse of the implication.
\item Write the inverse of the implication.
\item Write the contrapositive of the implication.
\item Write the inverse of the contrapositive of the implication.
\end{enumerate}
\end{prob}

\begin{prob} Consider the four answers for exercise \ref{prob:conincon}.
Which of the are logically equivalent to the implication?
Which of the four are not logically equivalent to the implication, (but are
logically equivalent to each other)?
\end{prob}

\begin{prob}
Use a truth table to show $[p\to (q\to r)]\not\equiv [(p\to q)\to r]$.
\end{prob}

\begin{prob}
 Use a truth table to show that $(p\land q)\to p$ is a  tautology.
\end{prob}

\begin{prob}
Give a proof of $\lnot p\to(p\to q)\equiv \mathbb{T}$ using the Fundamental Logical Equivalences, following the pattern of examples \ref{exmp:formalproof1} and \ref{exmp:formalproof2}.
\end{prob}




