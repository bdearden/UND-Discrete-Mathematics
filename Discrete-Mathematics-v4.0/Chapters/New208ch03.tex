\chapter{Predicates and Quantifiers}


\newthought{%
The sentence $x^2-2=0$ is not} a proposition. It cannot be assigned a truth
value unless some more information is supplied about the variable $x$.
Such a statement is called a  {\bfseries predicate} or a {\bfseries propositional
function}.  

\section{Predicates}
Instead of using a single letter to denote
a predicate, a symbol such as $S(x)$ will be used to indicate the dependence of the sentence on a 
variable. Here are two more examples of
predicates. 
\begin{enumerate}
\item $A(c) : $ {\itshape Al drives a $c$}, and 
\item $B(x,y) :$ {\itshape $x$ is the brother of $y$}.\marginnote{The second example is an instance of a {\bfseries two-place predicate}.}
\end{enumerate}
%\marginnote{The second example is an instance of a {\bfseries two-place predicate}.}
With a given predicate, there is an associated set of objects which can be
used in place of the variables. For example, in the predicate $S(x) : x^2-2=0$, 
it is understood that the $x$ can be replaced by a number. Replacing $x$ by, say, the
word {\itshape blue} does not yield a meaningful sentence. For the predicate
$A(c)$ above, $c$ can be replaced %
\marginnote{Usually the domain of discourse is left for the
reader to guess, but if the domain of discourse is something other than an obvious choice,
the writer will mention the domain to be used.}%
 by, say, makes of cars (or maybe types of nails!).
 For $B(x,y)$, the $x$ can be replaced by any human male, and the $y$ by any human. 
The collection of possible replacements for a variable in a predicate is called the 
{\bfseries domain of discourse} for that variable.

\section{Instantiation and Quantification}
A predicate is not a proposition, but it can be converted into a proposition. There
are three ways to modify a predicate to change it into a proposition. Let's use
$S(x) : x^2-2=0$ as an example.

The first way to change $S(x)$ to make it into a proposition is to assign a 
specific value from the variable's domain of discourse to the variable.
For example, setting $x=3$, gives the (false) proposition $S(3) : 3^2 - 2=0$.
On the other hand, setting $x=\sqrt{2}$ gives the (true) proposition 
$S(\sqrt{2}): (\sqrt{2})^2-2 = 0$. The process of setting a variable equal to a specific
object in its domain of discourse is called {\bfseries instantiation}. Looking at the
two-place predicate $B(x,y) : $ $x$ is the brother of $y$, we can instantiate
both variables to get the (true) proposition {\itshape $B($Donny, Marie$) : 
$ Donny is the brother of Marie}. Notice that the sentence
{\itshape $B($Donny, y$) : $ Donny is the brother of $y$} has not been
converted into a proposition since it cannot be assigned a truth value without some
information about $y$. But it has been converted from a two-place predicate to a
one-place predicate.

A second way to convert a predicate  to a proposition is to precede the predicate
with the phrase {\itshape There is an $x$ such that}.
For example, {\itshape There is an $x$ such that } $S(x)$ would become
 {\itshape There is an $x$ such that } $x^2-2=0$. This proposition is true 
if there is at least one choice of $x$ in its domain of discourse for which
the predicate becomes a true statement. The phrase   {\itshape There is an $x$ such that }
is denoted in symbols by $\exists x$, so the proposition above would be written
as $\exists x\, S(x)$ or $\exists x\, (x^2-2=0)$. When trying to determine the truth
value of the proposition $\exists x\, P(x)$, it is important to keep the domain
of discourse for the variable in mind. For example, if the domain for $x$ in  
 $\exists x\, (x^2-2=0)$ is all integers, the proposition is false. But if its
domain is all real numbers, the proposition is true. The phrase  
{\itshape There is an $x$ such that } (or, in symbols, $\exists x$) is called
 {\bfseries existential quantification}\sidenote{In English it can also be read as
{\itshape There exists $x$} or {\itshape For some $x$}.}.

The third and final way to convert a predicate into a proposition is by
{\bfseries universal quantification}\sidenote{The phrase {\itshape For all $x$} is
also rendered in English as {\itshape For each $x$} or
{\itshape For every $x$}.} . The universal quantification of a predicate, $P(x)$,
is obtained  by preceding the predicate with the phrase {\itshape For all $x$}, producing
the proposition {\itshape For all $x, P(x)$}, or, in symbols, $\forall x\, P(x)$. This
proposition is true provided the predicate becomes a true proposition for every object
in the variable's domain of discourse. Again, it is important to know the domain
of discourse for the variable since the domain will have an effect on the truth 
value of the quantified proposition in general. 

For multi-placed predicates, these three conversions can be mixed and matched. For example,
using the obvious domains for the predicate {\itshape $B(x,y) : $ $x$ is the 
brother of $y$} here are some conversions into propositions:

\begin{enumerate}
\item $B(Donny, Marie)$ has both variables instantiated. The proposition is true.

\item $\exists y\, B(Donny, y)$ is also a true proposition. It says {\itshape Donny} is
somebody's brother. The first variable was instantiated, the second was existentially
quantified.

\item $\forall y\, B(Donny,y)$ says everyone has Donny for a brother, and that is false.

\item $\forall x\, \exists y\, B(x,y)$ says every male is somebody's brother, and that is false.

\item $\exists y\,\forall x\, B(x,y)$ says there is a person for whom every male is a brother, and that is false.

\item $\forall x\, B(x,x)$ says every male is his own brother, and that is false.
\end{enumerate}


\section{Translating to symbolic form}
Translation between ordinary language and symbolic language can get a little tricky when
quantified statements are involved. Here are a few more examples.

\begin{exmp}\label{exmp:Porche}
Let $P(x)$ be the predicate \textbf{$x$ owns a Porsche}, and
let $S(x)$ be the predicate \textbf{$x$ speeds}. The domain of discourse for the variable
in each predicate will be the collection of all drivers. The proposition $\exists x P(x)$ says
\textbf{Someone owns a Porsche}.  It could also be translated as \textbf{There
is a person $x$ such that $x$ owns a Porsche}, but that sounds too stilted for ordinary
conversation. A smooth translation is better. The proposition $\forall x (P(x)\to S(x))$
says \textbf{All Porsche owners speed.} 

Translating in the other direction, 
the proposition \textbf{No speeder owns a Porsche} could be expressed
as $\forall x (S(x) \to \lnot P(x))$.
\end{exmp}

\begin{exmp}\label{exmp:Al knows}
Here's a more complicated example: translate the proposition \textbf{Al 
knows only Bill} into symbolic form. Let's use $K(x,y)$ for the predicate \textbf{$x$ knows
$y$}. The translation would be $K(Al,Bill)\land \forall x\,(K(Al,x)\to (x=Bill))$.
\end{exmp}

\begin{exmp}\label{exmp:sum of evens}
For one last example, let's translate \textbf{The sum of two even integers is even}
into symbolic form. Let $E(x)$ be the predicate \textbf{$x$ is even}. As with many statements
in ordinary language, the proposition is phrased in a shorthand code that the reader is 
expected to unravel. As
given, the statement doesn't seem to have any quantifiers, but they are implied. 
Before converting it to symbolic form, it might help to expand it to its more long winded version:
\textbf{For every choice of two integers, if they are both
even, then their sum is even}. Expressed this way, the translation to symbolic 
form is duck soup:
$\forall x\,\forall y\, ((E(x)\land E(y))\to E(x+y))$.
\end{exmp}

\section{Quantification and basic laws of logic}
Notice  that if the domain of discourse consists of finitely many entries $a_1, ..., a_n$,
then $\forall x\, p(x)\equiv p(a_1)\wedge p(a_2) \wedge ... \wedge p(a_n)$. 
So the quantifier $\forall$ can be expressed in terms of the logical connective $\wedge$. 
The existential quantifier and $\vee$ are similarly linked: 
$\exists x\, p(x)\equiv p(a_1)\lor p(a_2) \lor ... \lor p(a_n)$.

From the associative and commutative laws of logic  we see that we can rearrange any system of propositions which
are linked only by $\wedge$'s or linked only by $\vee$'s.\sidenote{For instance, consider  examples~\ref{exmp:Porche} -- \ref{exmp:sum of evens} with finite domains of discourse.} Consequently any more generally 
quantified proposition of the form $\forall x\forall y\, p(x,y)$ is logically equivalent 
to $\forall y \forall x\, p(x,y)$.
Similarly for statements which contain only existential quantifiers. But the distributive 
laws come into play when $\wedge$'s and $\vee$'s are mixed. So care must be taken with
predicates  which contain both existential and universal quantifiers, as the following example shows.

\begin{exmp}
Let $p(x,y): \mathbf{x+y=0}$ and let the domain of discourse be all real numbers for 
both $x$ and $y$. The proposition $\forall y\,\exists x\, p(x,y)$ is true, since, for any given $y$, by setting (instantiating) $x=-y$ we convert $\mathbf{x+y=0}$ to the true statement
$\mathbf{(-y)+y=0}$\sidenote{$(\forall y \in \mathbb{R})[(-y)+y=0]$ is a tautology.}. However the proposition $\exists x\, \forall y\, p(x,y)$ is false. If we set (instantiate) $y=1$, then
$\mathbf{x+y=0}$ implies that $x=-1$. When we set $y=0$, we get $x=0$. Since $0\neq -1$ there is no $x$
which will work for all $y$, since it would have to work for the specific values of $y=0$ and $y=1$.
\end{exmp}

\section{Negating quantified statements}
To form the negation of  quantified statements, we apply De Morgan's laws. This can be seen in 
case of a finite domain of discourse as follows:
\begin{align*}
\lnot (\forall x\, p(x))&\equiv\lnot (p(a_1)\wedge p(a_2) \wedge ... \wedge p(a_n))\\
&\equiv \lnot p(a_1)\lor\lnot p(a_2) \lor \dots \lor\lnot p(a_n)\\
&\equiv \exists x\, \lnot p(x)
\end{align*}
In the same way, we have $\lnot (\exists x\, p(x)) \equiv (\forall x\, \lnot p(x))$.
\sidenote{Use De Morgan's laws to find a similar expression for 
$\lnot(\exists x p(x))$.}


\clearpage
\section{Exercises}
\begin{exer}
Let $p(x): 2x\geq 4$, for integers $x$. Determine the truth values of the following propositions. 
\begin{tasks}(2)
	\task $p(2)$
	\task $p(-3)$
	\task $\forall x\, ((x\leq 10)\to p(x))$
	\task $\exists x\, \lnot p(x)$
\end{tasks}
\end{exer}

\begin{exer}\label{ex:quant->eng}
Let $p(x,y)$ be {\itshape $x$ has read $y$}, where the domain of discourse for 
$x$ is all students
in this class, and the domain of discourse for $y$ is all novels. Express the
following propositions in English. 
\begin{tasks}(2)
	\task $\forall x\, p(x$, War and Peace)
	\task $\exists x\, \lnot p(x$, The Great Gatsby)
	\task $\exists x\, \forall y\, p(x,y)$
	\task $\forall y\, \exists x\, p(x,y)$
\end{tasks}

\end{exer}

\begin{exer}\label{ex:eng->quant}
Let $F(x,y)$ be the statement {\itshape $x$ can fool $y$}, where the domain of discourse
for both $x$ and $y$ is  all people. Use quantifiers to express each of the 
following statements. 
\begin{tasks}(1)
	\task I can fool everyone.
	\task George can't fool anybody.
	\task No one can fool himself.
	\task There is someone who can fool everybody. 
	\task There is someone everyone can fool.
	\task Ralph can fool two different people.
\end{tasks}

\end{exer}

\begin{exer}
Negate each of the statements from exercise \ref{ex:quant->eng} in English. 

\end{exer}

\begin{exer}

Negate each statement from exercise \ref{ex:eng->quant} in logical symbols.  Of course, the easy
answer would be to simply put $\neg$ in front of each statement. But use the principle
given at the end of this chapter to move the negation across the quantifiers.

\end{exer}

\begin{exer}

Express symbolically: {\itshape The product of an even integer and an 
odd integer is even}.
\end{exer}

\begin{exer}
Express in words the meaning of 
\[
\exists x \, P(x) \land \forall x \,  \forall y \,\left( (P(x)\land P(y)) \to (x=y)\right).
\]

\end{exer}


\clearpage
\section{Problems}

\begin{prob}\label{bensprob}
Let $h$ be {\itshape Ben is healthy.}, $w$: {\itshape Ben is wealthy.}, and $s$: {\itshape Ben is wise.}\\
Express the following in English:
  \begin{tasks}
     \task $h\land w$
     \task $w \lor s$
     \task $h \to (w \land s)$
     \task $(h \to w) \land s$
   \end{tasks}
\end{prob}

\begin{prob}\label{movieprob}
Let $S(x,y)$ be the predicate {\itshape $x$ has seem $y$} where the domain of discourse for $x$
is all students in this class and the domain of discourse for $y$ is all movies. Express the following in logical symbols using quantifiers.
  \begin{tasks}
     \task Every student in this class has seen {\itshape Gone With The Wind}.
     \task No student in this class has seen {\itshape Jaws}.
     \task {\itshape The Shape of Water} has been seen by someone in this class.
     \task Some students have seen every movie.
     \task For each movie, there is at least one student in the class who has seen that movie.
   \end{tasks}
\end{prob}

\begin{prob}
Negate the propositions in~\ref{bensprob} in English.
\end{prob}

\begin{prob}
Negate the propositions in~\ref{movieprob} in symbols. Note: An easy way to do this is to simply write $\lnot$
in front of the answers in ~\ref{movieprob}. Don't do that! Give the negation with no quantifiers coming after a negation symbol.
\end{prob}

\begin{prob}
Negate the propositions in~\ref{movieprob} in English. Note: An easy way to do this is to simply write {\itshape
It is not the case that ....} 
in front of each proposition. Don't do that! Give the negation as a reasonably natural English sentence.
\end{prob}

