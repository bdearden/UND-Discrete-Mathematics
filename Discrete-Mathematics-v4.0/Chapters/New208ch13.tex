\chapter{Sequences and Summation}


\newthought{A {\bfseries sequence} is a
list} of numbers in a specific order. For example, the positive integers 
$1,2,3,\cdots$ is a
sequence, as is the list $4$, $3$, $3$, $5$, $4$, $4$, $3$, $5$, $5$, $4$ of the number of letters
in the English words of the ten digits in order {\itshape zero}, {\itshape one}, $\cdots$,
{\itshape nine}. 
Actually, the first is an example of an infinite sequence, the second is
a finite sequence. The first sequence goes on forever; there is no last number. The
second sequence eventually comes to a stop. In fact the second sequence has only
ten items. A {\bfseries term} of a sequence is one of the numbers that
appears in the sequence. The first term is the first number in the list,
the second term is the second number in the list, and so on.

\section{Specifying sequences}
  A more general way to think of a sequence is as a function from some subset
of $\mathbb{Z}$ having a least member (in most cases
either
 $\{\,0,\,1,\,2,\,\cdots\,\}$ or
$\{\,1,\,2,\,\cdots\,\}$) with codomain some {\itshape arbitrary} set. %
\marginnote{Computer science texts use the former and elementary math application texts use the later. Mathematicians use any such well-ordered domain set.}%
 In most mathematics
courses the codomain will be a set of numbers, but that isn't necessary. For
example, consider the finite sequence of initial letters of the words in the
previous paragraph: $a,s,i,a,l,o,n,\cdots,a,s,o$. If the letter $L$ is used to
denote the function that forms this sequence, then $L(1) = a$, $L(2) = s$, and so on.

\subsection{Defining a Sequence With a Formula}
  The examples of sequences given so far were described in words,
but there are other ways to tell what objects appear in the sequence.
One way is with a formula. For example, let $s(n)=n^2$, for
$n=1,2,3,\cdots$. As the values $1,2,3$ and so on are plugged into
$s(n)$ in succession, the infinite sequence $1,4,9,16,25,36,\cdots$ is
built up. It is traditional to write $s_n$ (or $t_n$, etc)
instead of $s(n)$ when describing the terms of a sequence, so the
formula above would usually be seen as $s_n=n^2$. Read that as
{\itshape $s$ sub $n$ equals $n^2$}. When written this
way, the $n$ in the $s_n$ is called a {\itshape subscript} or {\itshape index}. 
The subscript of $s_{173}$ is $173$.

\begin{exmp}
 What is the $50^{th}$ term of the sequence defined by the formula 
 $\displaystyle s_j = \frac{j+1}{j+2}$, where $j=1,2,3,\ldots$? We
 see that
 \[
 s_{50} = \frac{51}{52}.
\]
\end{exmp}

\begin{exmp}
 What is the $50^{th}$ term of the sequence defined by the formula 
 $\displaystyle t_k = \frac{k+1}{k+2}$, where $k=0,1,2,3,\ldots$? 
 Since the indicies start at $0$, the $50$th term will be $t_{49}$:
 \[
 t_{49} = \frac{50}{51}.
\]
\end{exmp}

\subsection{Defining a Sequence by Suggestion}
A sequence can also be specified by listing an initial portion
of the sequence, and trust the reader to successfully perform the
mind reading trick of guessing how the sequence is to continue
based on the pattern suggested by those initial terms. For example,\
consider the sequence $7,10,13,16,19,22,\cdots$. The symbol $\cdots$
means {\itshape and so on}. In other words, you {\it should} be able to figure
out the way the sequence will continue. This method of specifying
a sequence is dangerous of course. For instance, the number of terms 
sufficient for  one person to spot the pattern might not be 
enough for another person. Also, maybe there are several different
{\itshape obvious} ways to continue the pattern

\begin{exmp}
 What is the next term in the sequence $1,3,5,7\cdots$?
 One possible answer is $9$, since it looks like we are listing the positive
 odd integers in increasing order. But another possible answer is $8$: 
 maybe we are listing each positive integer with an {\bfseries e} in its name.
 You can probably think of other ways to continue the sequence. 
 \end{exmp}
 
 In fact,
 for any finite list of initial terms, there are always infinitely
 many more or less natural ways to continue the sequence. A reason can
 always be provided for absolutely any number to be the next in the 
 sequence. However, there will typically be only one or two {\itshape obvious}
 simple choices for continuing a sequence after five or six terms.   
  
 
 \section{Arithmetic sequences}
 The simple  pattern suggested by the initial terms $7,10,13,16,19,22,\cdots$ 
 is that the sequence begins with a $7$, and each term is produced
 by adding $3$ to the previous term. This is an
 important type of sequence. The general form is $s_1=a$ ($a$ is just some
 specific number), and, from the second term on, each new term
 is produced by adding $d$ to the previous term (where $d$ is some
 fixed number). In the last example, $a=7$ and $d=3$. A sequence of this
 form is called an {\bfseries arithmetic sequence}. The number $d$ is called
 the {\bfseries common difference}, which makes sense since $d$ is the
 difference of any two consecutive terms of the sequence.   It is
 possible to write down a formula for $s_n$ in this case. After all, to
 compute $s_n$ we start with the number $a$, and begin adding $d$'s to
 it. Adding one $d$ gives $s_2=a+d$, adding two $d$'s gives $s_3=a+2d$, and so
 on. For $s_n$ we will add $n-1$ $d$'s to the $a$, and so we see
 $s_n=a+(n-1)d$. In the numerical example above, the $5^{th}$ term of the
 sequence ought to be $s_5=7+4\cdot3=19$, and sure enough it is. The
 $407^{th}$ term of the sequence is $s_{407}=7+406\cdot3= 1225 $.
 
 \begin{exmp}
  The $1^{st}$ term of an arithmetic sequence is $11$
  and the the $8^{th}$ term is $81$. What is a formula for the $n^{th}$
  term?
  
  We know $a_1 = 11$ and $a_8 = 81$. Since 
  $a_8 = a_1 + 7d$, where $d$ is the common difference, we get the
  equation $81 = 11 + 7d$. So $d=10$. We can now write down a 
  formula for the terms of this sequence: $a_n = 11 + (n-1)10 = 1 + 10n$.
  Checking, we see this formula does give the required values for $a_1$
  and $a_8$.
 \end{exmp}

\section{Geometric sequences}
  For an arithmetic sequence we added the same quantity to get from
one term of the sequence to the next. If instead of adding we multiply
each term by the same thing to produce the next term the result
is called a {\bfseries geometric sequence}. 
 
\begin{exmp}
  Let $s_1=2$, and suppose we multiply by $3$ to get from
 one term to the next.
 The sequence we build now looks like
 $2,6,18,54,162,\cdots$, each term being $3$ times as large as the previous term.
\end{exmp}
 

In general, if $s_1=a$, and, for  
$n\geq1$, each new term is $r$ times the preceeding term, then the formula for the $n^{th}$ term of
the sequence is $s_n=ar^{n-1}$, which is reasoned out just as for
the formula for the arithmetic sequence above. The quantity $r$ in the
geometric sequence is called the {\bfseries common ratio} since it is the ratio
of any term in the sequence to its predecessor (assuming $r\not=0$ at any
rate).

\section{Summation notation}
   A sequence of numbers is an ordered list of numbers. A {\bf
summation} (or just {\bfseries sum}) is a sequence of numbers added up. A sum with
$n$ terms (that is, 
with $n$ numbers added up) will be denoted by $S_n$ typically. Thus if
we were dealing with sequence $1,3,5,7,\cdots,2n-1,\cdots$,
then $S_3 = 1+3+5$, and $S_n = 1+3+5+\cdots+(2n-1)$. For the arithmetic
sequence $a, a+d, a+2d, a+3d,\cdots$, we see $S_n = a +(a+d) +
(a+2d)+\cdots+(a+(n-1)d)$.


  It gets a little awkward writing out such extended sums and so a
compact way to indicate a sum, called {\bfseries summation notation},
is introduced. For the sum of the first $3$ odd positive integers above
we would write $\displaystyle{\sum_{j=1}^3(2j-1)}$. The Greek letter
sigma $(\Sigma)$ is supposed to be reminiscent of the word summation. The $j$ is
called 
the {\bfseries index of summation} and the number on the bottom of the
$\Sigma$
specifies the starting value of $j$ while the number above the $\Sigma$
gives the ending value of $j$. The idea is that we replace $j$ in
turn by $1$, $2$ and $3$, in each case computing the value of the
expression following the $\Sigma$, and then add up the terms produced.
In this example, when $j=1$, $2j-1=1$, when $j=2$, $2j-1=3$ and finally,
when $j=3$, $2j-1=5$. We've reached the stopping value, so we have
$\displaystyle{\sum_{j=1}^3(2j-1)}=1+3+5=9.$ 

Notice that the index of
summation takes only integer values. If it starts at $6$, then next it
is replaced by $7$, and so on. If it starts at $-11$, then next it is
replaced by $-10$, and then by $-9$, and so on.

The symbol used for the index of summation does not have to be $j$. Other
traditional choices for the index of summation are $i$, $k$, $m$ and $n$. So for
example, 
\[
\sum_{j=0}^4 (j^2+2) = 2+3+6+11+18,
\]
and
\[
\sum_{i=0}^4 (i^2+2) = 2+3+6+11+18,
\]
and
\[
\sum_{m=0}^4 (m^2+2) = 2+3+6+11+18,
\]
and so on. Even though a different index letter is used, the formulas produce the
same sequence of numbers to be added up in each case, so the sums are the
same.

Also, the starting and ending points can for the index can be changed
without changing the value of the sum provided
care is taken to change the formula appropriately. Notice that
\[
\sum_{k=1}^3 (3k-1)= \sum_{k=0}^2 (3k+2)
\]
In fact, if the terms are written out, we see
\[
\sum_{k=1}^3 (3k-1)= 2+5+8
\]
and 
\[
\sum_{k=0}^2 (3k+2) = 2+5+8
\]

\begin{exmp} We see that
\[
\sum_{m=-1}^5 2^m = 2^{-1} + 2^0 + 2^1 + 2^2 + 2^3 + 2^4 +2^5 = \frac{127}{2}.
\]
\end{exmp}

\begin{exmp} We find that
\[
 \sum_{n=3}^6 2 = 2+2+2+2 = 8.
\]
\end{exmp}

\section{Formulas for arithmetic and geometric summations}
  There are two important formulas for finding  sums 
that are worth remembering. The first is the sum of the first $n$ terms of
an arithmetic sequence.
\[
S_{n} = a +(a+d)+(a+2d) +\cdots +(a+(n-1)d).
\]
  Here is a clever trick that can be used to find a simple
formula for the quantity $S_n$: the list of numbers is added up twice,
once from left to right, the second time from right to left. When the
terms are paired up, it is clear the sum is $2S_{n}= n[a+(a+(n-1)d)]$.
A diagram will make the idea clearer:
\begin{align*}
     &a         &+ (a+d)~~~~~~~~~~~        &+ (a+2d)       &+ \cdots &+ ~(a+(n-1)d) \\
 +  (&a+(n-1)d) &+ (a+(n-2)d)   &+ (a+(n-3)d)   &+ \cdots &+ ~~a \\
 \hline
   (2&a+(n-1)d) &+ (2a+ (n-1)d) &+ (2a+ (n-1)d) &+\cdots  &+ (2a+ (n-1)d)
\end{align*}

%$$\vbox{\offinterlineskip
%\halign { $#$ & $#$ & $#$ & $#$ & $#$ & $#$ \cr
%~ & a & + (a+d) & + (a+2d) &+ \cdots & + (a+(n-1)d) \cr
%+ & (a+(n-1)d) & + (a+(n-2)d) & + (a+(n-3)d) &+\cdots & + a\cr
%\noalign{\hrule}
%~& (2a+(n-1)d) & +(2a+ (n-1)d) & +(2a+ (n-1)d) &+\cdots & +(2a+ (n-1)d)\cr}}$$

The bottom row contains $n$ identical terms, each equal to $2a+(n-1)d$,
and so 
 $2S_{n}= n\left[2a+(n-1)d)\right]$. 
Dividing by $2$ gives the important formula, for $n=1,2,3,\ldots$, %
\marginnote{An easy way to remember the formula is to think of the quantity in the
parentheses as the average of the first and last terms to be added, and
the coefficient, $n$, as the number of terms to be added.}
\begin{equation}\label{eqn:arith series formula}
S_{n}= n\left(\frac{2a+(n-1)d}{2}\right)=
  n\left(\frac{a +(a+(n-1)d)}{2}\right).
\end{equation}

\begin{exmp}
The first $20$ terms of the arithmetic sequence $5, 9, 13, \cdots$ is
found to be
\[
S_{20} = 20\left({5+ 81}\over2\right) = 860.
\]
\end{exmp}

 
For a geometric sequence,
 a little algebra  produces a formula for
the sum of the first $n$ terms of the sequence. 
The resulting formula for $S_{n} =
a+ar+ar^2+\cdots+ar^{n-1}$, is 
\[
S_{n} = {{a-ar^{n}}\over{1-r}}= a\left({{1-r^{n}}\over{1-r}}\right),
\text{ if $r\not=1$.}
\]

\begin{exmp} The sum of the first ten
terms of the geometric sequence $2,{2\over3},{2\over9},\cdots$
would be %
\marginnote{Notice that the numerator in this case is the difference of the first
term we have to add in and the term {\itshape immediately following} the last
term we have to add in.}
\[
S_{10}={{2-2\left(1\over3\right)^{10}}\over{1-\left(1\over3\right)}}
\]
 The expression for $S_{10}$ can be simplified as
\[
S_{10}={{2-2\left(1\over3\right)^{10}}\over{1-\left(1\over3\right)}}=2\left({{1-\left(1\over3\right)^{10}}\over{1-\left(1\over3\right)}}\right)=
2\left({{1-\left(1\over3\right)^{10}}\over{2\over3}}\right)=
3\left(1-{1\over{3^{10}}}\right)= 3-{1\over{3^9}}
\]
\end{exmp}


Here is the algebra that shows the geometric sum formula is correct.

Let $S_n = a+ar+ar^2+\cdots+ar^{n-1}$. Multiply both sides of that
equation by $r$ to get
\[
rS_n = r(a+ar+ar^2+\cdots+ar^{n-1}) = ar +ar^2+ar^3+\cdots +ar^{n-1}+ar^n
\]
Now subtract, and observe that most terms will cancel:
\begin{align*}
 S_n-rS_n & = (a+ar+ar^2+\cdots+ar^{n-1})-(ar +ar^2+ar^3+\cdots +ar^{n-1}+ar^n) \\
 &= a + (ar+ar^2+\cdots+ar^{n-1}) -(ar +ar^2+ar^3+\cdots +ar^{n-1})-ar^n \\
 &= a - ar^n
\end{align*}
So $S_n(1-r) = a-ar^n$. Assuming $r\not=1$, we can divide both sides of that
equation by $1-r$, producing the promised formula\sidenote{Find a formula for $S_n$ when $r=1$.}:
\begin{equation}\label{eqn:geom series formula}
S_n= {{a-ar^{n}}\over{1-r}}= a\left({{1-r^{n}}\over{1-r}}\right), \text{ if $r\not=1$.}
\end{equation}

\clearpage

\section{Exercises}

\begin{exer} 
Guess the next term in the sequence $1,2,4,5,7,8,\cdots$. What's
another possible answer?
\end{exer}

\begin{exer} 
What is the $100^{th}$ term of the arithmetic sequence with initial
term $2$ and common difference $6$?
\end{exer}

\begin{exer}
The $10^{th}$ term of an arithmetic sequence is $-4$
and the $16^{th}$ term is $47$. What is the $11^{th}$ term?
\end{exer}

\begin{exer} 
 What is the $6^{th}$ term of the geometric sequence with initial
term $6$ and common ratio $2$?
\end{exer}

\begin{exer}
The first two terms of a geometric sequence are $g_1=5$ and $g_2=-11$. What is the $g_5$?
\end{exer}

\begin{exer}
Which sequences are both a geometric sequence also an arithmetic sequence?
\end{exer}

\begin{exer} 
 Evaluate $\sum_{j=1}^{4} (j^2+1)$.
\end{exer}

\begin{exer}
  Evaluate $\sum_{k=-2}^{4} (2k-3)$.
\end{exer}

\begin{exer}
 What is the sum of the first $100$ terms  of the arithmetic sequence with initial
term $2$ and common difference $6$?
\end{exer}

\begin{exer} 
 What is the sum of the first five terms of the geometric sequence with initial
term $6$ and common ratio $2$?
\end{exer}

\begin{exer} 
 Evaluate $\sum_{i=0}^{4} \left(-{3\over2}\right)^i$.
 \end{exer}

\begin{exer}
 Express in summation notation: $\displaystyle {1\over2}+{1\over4}+{1\over6}+
\cdots+{1\over{2n}}$, the sum of the reciprocals of the first $n$ even
positive integers.
 \end{exer}
 
 \clearpage
 
\section{Problems}

\begin{prob} 
Guess the next term in the sequence $1, 3, 5, 7, 8, 9\cdots$. What's
another possible answer?
\end{prob}

\begin{prob}
Guess the next term in the sequence $1, 2, 2, 3, 2, 4, 2, 4, 3\cdots$.
\end{prob}

\begin{prob}
A sequence begins $1, 3, 9, 15$. Could it be an arithmetic sequence?
Could it be a geometric sequence?
\end{prob}

\begin{prob}
What is the $20^{th}$ term of the arithmetic sequence with initial
term $4$ and common difference $5$?
\end{prob}

\begin{prob}
The $8^{th}$ term of an arithmetic sequence is $20$
and the $12^{th}$ term is $40$. What is the $25^{th}$ term?
\end{prob}

\begin{prob} 
 What is the $7^{th}$ term of the geometric sequence with initial
term $3$ and common ratio $4$?
\end{prob}

\begin{prob}
Two terms of a geometric sequence are $g_3=2$ and $g_5=72$. There two possible
values for $g_4$. What are those two values?
\end{prob}

\begin{prob}
A geometric sequence has initial term $3$, and common ration $7$. Determine the smallest value of $n$
so that the $n^{th}$ term of the sequence is more that one million.
\end{prob}

\begin{prob} 
 Evaluate $\sum_{j=1}^{4} (j+1)^{2}$.
\end{prob}

\begin{prob}
 Evaluate $\sum_{k=-2}^{4} (2k+3)$.
\end{prob}

\begin{prob}
 What is the sum of the first $100$ terms  of the arithmetic sequence with initial
term $2$ and common difference $6$?
\end{prob}

\begin{prob} 
 What is the sum of the first four terms of the geometric sequence with initial
term $3$ and common ratio $-2$?
\end{prob}

\begin{prob}
What is the sum of the first four thousand terms of the geometric sequence with initial
term $3$ and common ratio $-1$?
\end{prob}

\begin{prob}
You have two parents, and four grandparents, and eight great grandparents, for a total fourteen 
ancestors three generations back. How many ancestors do you have $50$ generations back?
(A generation is generally taken to be about $30$ years, so $50$ generations is about $1500$
years. That would take us back to about the time the decimal system was invented in India. 
How can you explain the obviously impossible answer to this problem?)
\end{prob}

\begin{prob} 
 Evaluate $\sum_{i=0}^{4} \left({3\over2}\right)^i$.
 \end{prob}

\begin{prob}
 Express in summation notation: $\displaystyle {1\over1}+{1\over3}+{1\over5}+
\cdots+{1\over{2n-1}}$, the sum of the reciprocals of the first $n$ odd
positive integers.
 \end{prob}
 


