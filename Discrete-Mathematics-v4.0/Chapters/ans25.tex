   \section*{Chapter 25}
    
\begin{Solution}{25.1}

Since $\gcd(21,48) = 3$ and $3$ does not divide $8$, there are no integer solutions to this equation. 

\end{Solution}

\begin{Solution}{25.2}

Since $\gcd(21,48) = 3$ and $3$ does  divides $9$, there are integer solutions to this equation.
To find one solution, we could use the Extended Euclidean Algorithm, but in this case the numbers are small
enough that we can find a solution by inspection (in other words, {\it we can guess an answer}). First, let's reduce the equation by dividing each side by $3 = \gcd(21,48)$ to get the equation $7x+16y = 3$. We can quickly see a solution: $x=5,\,y = -2$  since $7(5) + 16(-2) = 35 - 32 = 3$. Multiplying both sides of $7(5)+ 16(-2) = 3$ by $3$ gives
$21(5) + 48(-2) = 9$. Now that we have one solution to $21x+48y = 8$ we can write down all solutions:

\begin{align*}
x & = 5 + \frac{48}{\gcd(21,48)}k = 5 + 16k\\
y &= -2 -\frac{21)}{\gcd(21,48)}k = -2 - 7k
\end{align*}
where $k$ is any integer. Just to check our work, let's test the solution when $k = 10$ (so $x= 165$ and $y = -72$).

\[
(21)(165) + (48)(-72) = 3465- 3456 = 9
\]
That looks good.

\end{Solution}

\begin{Solution}{25.3}
Since $gcd(33,12)=3$ and $3$ does not divide $7$, there are no solutions.
\end{Solution}

\begin{Solution}{25.4}
We need one solution to get the ball rolling.
We could use the continued fraction algorithm to write $3 = gcd(33,12)$
as a linear combination of $33$ and $12$, and we would probably have to
do that if the numbers were larger. But with these small numbers we can
do the work in our head: $3 = (33)(-1) + (12)(3)$.

Multiply that equation by $2$ on each side to get $(33)(-2)+ (12)(6) = 6$.

So now we have one solution to the given equation: $x = -2$ and $y=6$.

Using the formulas that produce all solutions once one is known we get all solutions
are given by

\[
 x = -2 + \frac{12}{3}k =-2+4k\qquad \text{ and }\qquad y = 6 - \frac{33}{3}k=6-11k
 \]
where $k$ is any integer.

For example, when $k=5$ we get the solution $x = 18$, $y = -49$.

\end{Solution}

\begin{Solution}{25.5}
First, let's find all solutions to $59x + 37y = 4270$ by applying the Extended Euclidean Algorithm.

\begin{table}
\renewcommand{\arraystretch}{1.25}
\begin{tabular}{|*{8}{>{\raggedleft\arraybackslash}p{0.996cm}|}}
  \hline
 59&37&22&15&7&1&0 \\
 \hline
  &&1&1&1&2&7 \\
 \hline
 0&1&-1&2&-3&8&-59 \\
 \hline
 1&0&1&-1&2&-5&37 \\
 \hline
\end{tabular}
\end{table}
The table shows $\gcd(59,37) = 1 = 59(-5) + 37(8)$.  Multiplying by $4270$, we see one solution to
$59x+37y = 4270$ is given by $x = -5(4270)$ and $y = 8(4270)$. That means all solutions to that 
equations are given by
\[
x = -5(4270)+ 37k \qquad \text{and} \qquad y = 8(4270) - 59k\qquad \text{where $k$ is any integer}.
\] 
We need to find values of $k$ for which both $x$ and $y$ are $0$ or more. In other words, we want to solve

\[
-5(4270)+37k\geq0 \qquad\text{and}\qquad 8(4270)-59k\geq0.
\]
That reduces to $\displaystyle \frac{5(4270)}{37}\leq k \leq \frac{8(4270)}{59}$, or
$577.02\ldots\leq k\leq 578.98\ldots$. The only integer option is $k = 578$.
We conclude the number of vases sold is $x = -5(4270) +37(578) = 36$ and the 
number of ash trays is $y = 8(4270)-59(578) = 58$.


\end{Solution}
