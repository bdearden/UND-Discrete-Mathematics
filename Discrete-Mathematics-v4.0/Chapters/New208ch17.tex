\chapter{Algorithms}


\newthought{An {\bfseries algorithm} is a recipe} to solve a problem. For example, here is an algorithm
that solves the problem of finding the distance traveled by a car given the time it has traveled,
$t$, and its average speed, $s$: \textsf{multiply $t$ and $s$}. 

\section{Properties of an algorithm}
Over time, the requirements of what exactly constitutes an algorithm have matured. A really
precise definition would be filled with all sorts of technical jargon, but the ideas are 
commonsensible
enough that an informal description will suffice for our purposes. So, suppose we have in mind a 
certain class of problems (such as determine the distance traveled given time traveled and 
average speed). The properties of an
algorithm to solve examples of that class of problems are:
\begin{enumerate}
 \item {\bfseries Input}: The algorithm is provided with data.
 
 \item {\bfseries Output}: The algorithm produces a solution.
 
 \item {\bfseries Definiteness}: The instructions that make up the algorithm are precisely
 described. They are not open to interpretation.
 
 \item {\bfseries Finiteness}: The output is produced in a finite number of steps.
 
 \item {\bfseries Generality}: The algorithm produces correct output for any set of
 input values.
\end{enumerate}

The algorithm for finding distance traveled given time traveled and average speed obviously 
meets all five requirements of an algorithm. Notice that, in this example, we have assumed 
the user of the algorithm understands what it means to multiply
two numbers. If we cannot make that assumption, then we would need to add a number of 
additional steps to the algorithm to solve the problem of multiplying two numbers together.
Of course, that would make the algorithm significantly longer. When describing algorithms,
we'll assume the user knows the usual algorithms for solving common problems such
as addition, subtraction, multiplication, and division  of numbers, and knows how to determine
if one number is larger than another, and so on.

\section{Non-algorithms}\label{sec:non-algorithms}
Just as important as an example of what an algorithm is, is an example of what is not an algorithm.
For example, we might describe the method by which most people look up a number in a phone book.
You open the book and look to see if the listing you're looking for is on that page or not. If it is
you find the number using the fact that the listings are alphabetized and you're done. If the number
you're looking for is not on the page, you use the fact that the listings are alphabetized to either
flip back several pages, or forward several pages. This page is checked to see if the listing is on it.
If it is not we repeat the process. One problem in this case is that this description is not definite.
The phrase {\itshape flip back several pages} is too vague, it violates the definiteness 
requirement. Another problem is that someone could flip
back and forth between two pages and never find the number, and so violate the finiteness 
requirement. So this method is not an algorithm.

\section{Linear search algorithm}
Continuing the example from section \ref{sec:non-algorithms} of looking up a phone number, one algorithm
for completing this is to look at the first entry in the book. If it's the number you're looking for you're
done. Else move to the next entry. It's either the number you're looking for or you move to the
next entry. This is an example of a {\bfseries linear search algorithm}. It's not too bad
for finding Adam Aaronson's number if he is in the book, but it is terrible if you're trying to reach 
Zebulon Zyzniewski.

\section{Binary search algorithm}
Another algorithm to complete that task of finding a phone number is the {\bfseries binary search algorithm}.
We open the phone book to the middle entry. If it's the number we're looking for, we're done.
Else we know the number is listed in the first half of the book, or the last half of the book since
 the entries are alphabetized. We then pick the middle entry
of the appropriate  half, and repeat the halving process on that half, 
until eventually the name is located.
There are a few details to fix up to make this a genuine algorithm. For example, what is the
middle entry if there are an even number of items listed? Also, what happens if the name
we are looking for isn't in the phone book? But it is clear with a little effort we can add a 
few lines to the instructions to make this process into an algorithm.


\section{Presenting algorithms}
It is traditional to present algorithms %in a telegraphic format or 
in a \emph{pseudocode} form
similar to a program for a computer. 
For instance, the linear search name lookup algorithm given above could be written 
%in telegraphic form as:
%\begin{algrthm}
%\hrule\kern5pt\relax
% \begin{enumerate}[label=\bfseries step \arabic*:,leftmargin=*]
%  \setlength{\itemsep}{0cm}%
%  \setlength{\parskip}{0cm}%
%  \item[\bfseries input:] {\itshape name} to be found in a given {\itshape list}
%  \item {\bfseries set} $n=1$
%  \item {\bfseries if} list is shorter than $n$, {\bfseries output} {\itshape name not found} 
%  and {\bfseries stop}
%  \item {\bfseries if} name in spot $n$ is the one we want, {\bfseries then} {\bfseries output}
%   phone number in spot $n$ and {\bfseries stop}
%  \item {\bfseries replace}  $n$ by $n+1$ and  {\bfseries go to} step 2.
% \end{enumerate}\hrule\kern5pt\relax
%\caption{Linear search (telegraphic)}\label{alg:lin search telegraphic}
%\end{algrthm}
%
%The same algorithm\sidenote{There are many other ways to implement a linear search.} in pseudocode format might look like:
in pseudocode form, as shown in Algorithm~\ref{alg:lin srch repeat} on page~\pageref{alg:lin srch repeat}. Another version using a \textbf{for} loop is displayed in Agorithm~\ref{alg:lin srch for loop}.
\begin{algrthm}
\hrule\kern5pt\relax
\begin{algorithmic}[1]
\Require{\textit{(name,phonelist):} \textit{name} to be found in a given \textit{phonelist}}
\Ensure{ \textit{phonelist(namespot)} $=$ phone number of \textit{name} in \textit{phonelist}} 
\State $\text{\itshape namespot}\gets 1$ \Comment{{\color{blue}set \textit{namespot} to $1$, position of first \textit{name} in \textit{list}}}
\Repeat \Comment{{\color{blue}execute the block of code between \textbf{repeat} and \textbf{until}}}
  \If{\textit{list(namespot)} is \textit{name}} %  
        \Comment{{\color{blue}if \textbf{True} execute code to \textbf{end if}}}
    \State \textbf{output} \textit{phonelist(namespot)}
    \State \textbf{stop} \Comment{{\color{blue}We're Done!}}
  \EndIf \Comment{{\color{blue}execute the lines between \textbf{if} and \textbf{end if}}}
  \State $namespot \gets namespot+1$ \Comment{{\color{blue}increment $namespot$ by $1$}}
\Until{\textit{length(list)} < $namespot$} \Comment{{\color{blue}if \textbf{False}, jump back to \textbf{repeat}}}
\State \textbf{output} \textit{name not found}
    \State \textbf{stop}
\end{algorithmic}
\kern2pt\hrule\kern5pt\relax
\caption{Linear search (repeat/until). {(A $\vartriangleright$ \color{blue} indicates a comment follows.)}}\label{alg:lin srch repeat}
\end{algrthm}

\begin{algrthm}
\hrule\kern5pt\relax
\begin{algorithmic}[1]
 \Require{\textit{(name,phonelist)}} \Ensure{phonelist(namespot)}
 \Ensure{ \textit{phonelist(namespot)} $=$ phone number of \textit{name} in \textit{phonelist}} 
 \For{ $namespot \in \{1,2,\dots,length(phonelist)\}$ }
   \If{\textit{list(namespot)} is \textit{name}}
     \State \textbf{output} \textit{phonelist(namespot)}
     \State \textbf{stop}
   \EndIf
 \EndFor
 \State \textbf{output} \textit{name not found}
 \State \textbf{stop}
\end{algorithmic}
\kern2pt\hrule\kern5pt\relax
\caption{Linear search (for loop)}\label{alg:lin srch for loop}
\end{algrthm}

\clearpage
\section{Examples}

\begin{exmp}
Here is an algorithm for determining $\lfloor{m/n}\rfloor$ for 
positive integers $m,n$.
\begin{algrthm}
  \hrule\kern5pt\relax
  \begin{algorithmic}[1]
     \Require{positive integers $m$ and $n$}
     \Ensure{integer value of $\lfloor m/n \rfloor$} 
     \State $k \gets 0$ \Comment{$k$ will eventually hold our answer}
     \While{$m\geq 0$}
       \State $m \gets m-n$ \Comment{We're doing division by repeated subtraction}
       \State $k \gets k+1$ \Comment{$k$ counts the number of subtractions}
     \EndWhile
     \State \textbf{output} $k-1$ \Comment{We counted one too many subtractions!(How?)}
  \end{algorithmic}
  \hrule\kern5pt\relax
  \caption{Calculate $\lfloor m/n \rfloor$}
\end{algrthm}
%\begin{marginalgrthm}
%\kern2pt\hrule\kern5pt\relax
%\begin{algorithmic}[1]
% \Require{positive integers $m$ and $n$}
% \Ensure{value of $\lfloor{m/n}\rfloor$}
% \State $k\gets 0$
% \Repeat
%   \State {\bfseries replace} $m$ by $m-n$
%   \State $k \gets k+1$ 
% \Until{$m<0$}
% \State \textbf{output} $k$
% \State \textbf{stop}
% \end{algorithmic}
% \kern2pt\hrule\kern5pt\relax
% \caption{telegraphic form}
%\end{marginalgrthm}

\noindent Here are the sequence of steps this algorithm would carry out with input $m = 23$ and $n=7$:
\qquad[initial status. $m=23$, $n=7$, $k=\text{(undefined)}$]
\begin{enumerate}\itemsep0pt
 \item[instr 1:] \emph{Set $k$ to be $0$} \qquad[status: $m=23$, $n=7$, $k=0$]
 \item[instr 2:] is $m\geq 0$? \emph{Yes, $(23\geq0)$ is true. 
    Do next instruction (i.e.~instr 3).}
 \item[instr 3:] \emph{$m$ reset to be $m-n=23-7=16$.} \quad[status: $m=16$, $n=7$, $k=0$]
 \item[instr 4:] \emph{$k$ reset to be $k+1=0+1=1$.} \qquad[status: $m=16$, $n=7$, $k=1$]
 \item[instr 5:] \emph{jump back to the matching \textbf{while} (i.e.~instr 2).}
 \item[instr 2:] is $m \geq 0$? \emph{Yes, $(16\geq0)$ is true. Do next instruction.} 
 \item[instr 3:] \emph{$m$ reset to be $m-n=16-7=9$.} \qquad[status: $m=9$, $n=7$, $k=1$]
 \item[instr 4:] \emph{$k$ reset to be $k+1=2+1=3$.} \qquad[status: $m=9$, $n=7$, $k=2$]
 \item[instr 5:]\emph{ jump back to the matching \textbf{while}.} 
 \item[instr 2:] is $m \geq 0$? \emph{Yes, $(9\geq0)$ is true. Do next instruction. } 
 \item[instr 3:] \emph{$m$ reset to be $m-n=9-7=2$.} \qquad[status: $m=2$, $n=7$, $k=2$]
 \item[instr 4:] \emph{$k$ reset to be $k+1=2+1=3$.} \qquad[status: $m=2$, $n=7$, $k=3$]
 \item[instr 5:] \emph{jump back to the matching \textbf{while}.}
 \item[instr 2:] is $m \geq 0$? \emph{Yes, $(2\geq0)$ is true. Do next instruction.} 
 \item[instr 3:] \emph{$m$ reset to be $m-n=2-7=-5$.} \qquad[status: $m=-5$, $n=7$, $k=2$]
 \item[instr 4:] \emph{$k$ reset to be $k+1=3+1=4$.} \qquad[status: $m=-5$, $n=7$, $k=4$]
 \item[instr 5:] \emph{jump back to the matching \textbf{while}.}
 \item[instr 2:] is $m \geq 0$? \emph{\textbf{No}, $()-5\geq0)$ is false. 
                 Jump to instr after \textbf{end while}.}  
 \item[instr 6:] \textbf{output} value of $k-1$~(i.e. $3$). 
                           \qquad[status: $m=-5$, $n=7$, $k=4$]        
 \item[\bfseries stop!] \qquad(This is what happens when there are no more instructions to execute.)                               
\end{enumerate}

\noindent
Here is a second algorithm for the same problem.
\begin{algrthm}
  \hrule\kern5pt\relax
  \begin{algorithmic}[1]
     \Require{positive integers $m$ and $n$}
     \Ensure{integer value of $\lfloor m/n \rfloor$} 
     \State \textbf{divide} $m$ by $n$ to one place beyond the decimal, call the result $r$.
      \State \textbf{output} the digits of $r$ preceding the decimal point.
  \end{algorithmic}
  \hrule\kern5pt\relax
  \caption{Calculate $\lfloor m/n \rfloor$ (again)}
\end{algrthm}

%\begin{algrthm}
%  \hrule\kern5pt\relax
%  \begin{enumerate}[label=\bfseries step \arabic*:,leftmargin=*]
%   \setlength{\itemsep}{0cm}%
%   \setlength{\parskip}{0cm}%
%   \item[\bfseries input:] positive integers $m$ and $n$.
%   \item[\bfseries output:] value of $\lfloor{m/n}\rfloor$.
%   \item{\bfseries divide} $m$ by $n$ to one place beyond the decimal,\\ call the result $r$
%   \item{\bfseries output} the digits in $r$ preceding the decimal point, and {\bfseries stop}
%  \end{enumerate} 
%    \hrule\kern5pt\relax
%\end{algrthm}
\noindent
So, again, given input $m=23$ and $n=7$ we go through the steps.
\begin{enumerate}
 \item[instr 1:] $r = 3.2$ 
 \item[instr 2:] Output $3$
 \item[\bfseries stop!]
\end{enumerate}
\end{exmp}

\clearpage
\begin{exmp}
An algorithm to make \$n change using \$10, \$5, and \$1 bills.
\begin{algrthm}
  \hrule\kern5pt\relax
  \begin{algorithmic}[1]
    \Require{positive integers $m$ and $n$}
    \Ensure{integer value of $\lfloor m/n \rfloor$} 
      \While{$n\geq10$}
        \State \textbf{output} \$$10$
        \State $n \gets n-10$
      \EndWhile
      \While{$n\geq5$}
        \State \textbf{output} \$$5$
        \State $n \gets n-5$
      \EndWhile
      \While{$n\geq1$}
        \State \textbf{output} \$$1$
        \State $n \gets n-1$
      \EndWhile
  \end{algorithmic}
  \hrule\kern5pt\relax
  \caption{Make change}
\end{algrthm}

%\no {\bfseries input}: integer amount $n\geq 0$
%
%\no step 1: {\bfseries if} $n\geq 10$, {\bfseries then} {\bfseries output} \$10 and
% {\bfseries replace} $n$ by $n-10$, {\bfseries repeat} step 1
%
%\no step 2:   {\bfseries if} $n\geq 5$, {\bfseries then} {\bfseries output} \$5 and
% {\bfseries replace} $n$ by $n-5$, and {\bfseries repeat} step 2
%
%\no step 3: {\bfseries output} \$1 $n$ times, {\bfseries output} {\itshape done!}, and {\bfseries stop}
\noindent
For an input of $27$, the output would be \quad\$$10$, \$$10$, \$$5$, \$$1$, \$$1$.
\end{exmp}

\clearpage
\section{Exercises}

\begin{exer}
Consider the following algorithm: The input will be two 
integers, $m\geq 0$, and $n\geq 1$.
\begin{algrthm}
  \hrule\kern5pt\relax
  \begin{algorithmic}
    \Require{positive integers $m\geq0$ and $n\geq1$}
    \Ensure{(to be determined)} 
    \State $s \gets 0$
    \While{$m\not=0$}
      \State $s \gets n+s$
      \State $m \gets m-1$
    \EndWhile
    \State \textbf{output} $s$
  \end{algorithmic}
  \hrule\kern5pt\relax
\end{algrthm}
%
%\no step 1: {\bfseries set} $s=0$
%
%\no step 2: {\bfseries if} $m=0$, {\bfseries then output} $s$ and {\bfseries stop}
%
%\no step 3: {\bfseries replace} $s$ by $n+s$
%
%\no step 4: {\bfseries replace} $m$ by $m-1$ and {\bfseries go to} step 2

Describe in words what this algorithm does. In other words,
what problem does this algorithm solve?
\end{exer}

\begin{exer}\label{exer:17.2}
Consider the following algorithm: The input will be any integer $n$,
greater than $1$.
\begin{algrthm}
  \hrule\kern5pt\relax
  \begin{algorithmic}
    \Require{integer $n>1$}
    \Ensure{(to be determined)} 
    \State $t \gets 0$
    \While{$n$ is even}
      \State $t \gets t+1$
      \State $n \gets n/2$
    \EndWhile
    \State \textbf{output} $t$
  \end{algorithmic}
  \hrule\kern5pt\relax
\end{algrthm}

%\no step 1. {\bfseries set} $t=0$
%
%\no step 2. {\bfseries if} $n$ is odd, {\bfseries then output} $t$ and {\bfseries stop}
%
%\no step 3. {\bfseries add} $1$ to $t$
%
%\no step 4. {\bfseries replace} $n$ by ${n\over 2}$, and {\bfseries go to} step 2

\begin{enumerate}[label=(\alph*)]
 \item List the steps the algorithm follows for the input $n=12$.
 
 \item Describe in words what this algorithm does. In other words,
 what problem does this algorithm solve?
\end{enumerate}
\end{exer}

\begin{exer}\marginnote{Such an algorithm is needed quite often in computer science.}
Design an algorithm that takes any positive integer $n$ and returns half of $n$ if it is even and
half of $n+1$ if $n$ is odd.
\end{exer}

\clearpage
\begin{exer}
Consider the following algorithm. The input will be a function $f$ together
with its finite domain, $D = \{d_1,d_2,\cdots,d_n\}$.
\begin{algrthm}
  \hrule\kern5pt\relax
  \begin{algorithmic}[1]
    \Require{function $f$ with domain  $D = \{d_1,d_2,\cdots,d_n\}$}
    \Ensure{(to be determined)} 
    \State $i \gets 1$
    \While{$i<n$}
      \State $j \gets i+1$
      \While{$j<n$}
        \If{$f(d_j)=f(d_i)$}
          \State \textbf{output} \textsf{NO}
          \State \textbf{stop}
        \EndIf
        \State $j \gets j+1$
      \EndWhile
      \State $i \gets i+1$
    \EndWhile
    \State \textbf{output} \textsf{YES}
  \end{algorithmic}\label{alg:exer 17.3 words}
  \hrule\kern5pt\relax
\end{algrthm}

%\no step 1. {\bfseries set} $i=1$
%
%\no step 2. {\bfseries if} $i=n$, {\bfseries then output} YES and {\bfseries stop}
%
%\no step 3. {\bfseries set} $j=i+1$
%
%\no step 4. {\bfseries if} $j = n$, {\bfseries then go to} step 7
%
%\no step 5. {\bfseries if} $f(d_j) = f(d_i)$, {\bfseries then output} NO and {\bfseries stop}
%
%\no step 6. {\bfseries replace} $j$ by $j+1$ and {\bfseries go to} step 4
%
%\no step 7. {\bfseries replace} $i$ by $i+1$
%
%\no step 8. {\bfseries go to} step 2

\begin{enumerate}[label=(\alph*)]
 \item List the steps the algorithm follows for the input $f: \{a,b,c,d,e\}\lra\{+,*,\&,\$,\#,@\}$
 given by $f(a)=*, f(b)= \$, f(c) = +, f(d) = \$$, and $f(e)= @$.
 
 \item Describe in words what this algorithm does. In other words,
 what problem does this algorithm solve?
\end{enumerate}

\end{exer}

\begin{exer}
Design an algorithm that will convert the ordered triple $(a,b,c)$ to the
ordered triple $(b,c,a)$. For example, if the input is $(7,X, *)$, the ouput
will be $(X,*,7)$.
\end{exer}

\begin{exer}
Design an algorithm whose input is a  finite  list of positive integers and whose output
is the sum of the even integers in the list. If there are no even integers in the list, the 
output should be $0$.
\end{exer}

\begin{exer}
 A {\bfseries palindrome} is a string of letters that reads the same in each direction.
For example, \textsf{refer} and \textsf{redder} are palindromes of length five and six respectively.
Design an algorithm that will take a string as  input and output  \textsf{yes} if the string
is a palindrome, and \textsf{no} if it is not. 
\end{exer}

\clearpage

\section{Problems}
