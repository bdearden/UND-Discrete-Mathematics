\chapter{Integers in Other Bases}

The usual way of
writing integers is in terms of groups of ones (units), and groups of
tens, and groups of tens of tens (hundreds), and so on. Thus $237$
stands for $7$ units plus $3$ tens and $2$ hundreds, or
$2(10^2)+3(10)+7$.  This is the familiar decimal notation for numbers
(deci = ten). But there is really nothing special about the number ten
here, and it could be replaced by any integer bigger than one. That is,
we could use say $7$ the way $10$ was used above to describe a number. Thus we
would specify how many units, how many $7$'s
and $7^2$'s and $7^3$'s, and so on are needed to make up the
number. When a number is expressed in this fashion with $b$ in place
of the $10$, the result is called the {\bfseries base-$\mathbf{b}$} expansion
(or {\bfseries radix-$\mathbf{b}$} expansion) of the integer.

For example, the decimal integer $132$,   is made up
of two $7^2$'s, four $7$'s and finally six units. Thus we express the
base ten number $132$ as $246$ in base $7$, or as $246_7$, the little
$_7$ indicating the base. For small numbers, with a couple of minutes
practice, conversion from base $10$ (decimal) to other
bases, and back again can be carried out mentally.  For larger numbers, mental
arithmetic 
will prove a little awkward. Luckily there is a handy algorithm
to do the conversion automatically.  

\section{Converting to and from base-$10$}
For base $10$ integers, we use the decimal digits: $0,1,2,3,4,5,6,7,8,9$.
In general, for base $b$, the digits will be $0,1,2,3\cdots,b-1$. So, for example,
a base $7$ numbers use digits $0,1,2,3,4,5,6$.


Conversion from the base $b$ expansion of a number to its decimal version
is a snap: For example, the meaning of $2302_5$
is 
$$
2302_5= 2\cdot5^3+3\cdot5^2+0\cdot5+2 = 2(125)+3(25) +0(5) + 2 = 327
$$

That sort of computation is so easy because we have been practicing base $10$
arithmetic for so many years. If we were as good at arithmetic in some base $b$,
then conversion from base $10$ to base $b$ would be just as simple. But, lacking
that comfort with base $b$ arithmetic, we need to describe the conversion 
algorithm from decimal to base $b$ a little more formally. Here's the idea.

Suppose we have a decimal number $n$ that we want to convert to some base $b$.
Let's say the base $b$ expansion is $d_kd_{k-1}\cdots d_2d_1d_0$, with the
base $b$ digits between $0$ and $b-1$.
That means

$$
n =  d_k\cdot b^k+d_{k-1}\cdot b^{k-1}+\cdots+ d_2\cdot b^2+d_1\cdot b+ d_0
$$

Now, if we divide $n$ by $b$, we can see the equation above tells us

$$
n =  (d_k\cdot b^{k-1}+d_{k-1}\cdot b^{k-2}+\cdots+ d_2\cdot b+d_1) b+ d_0
$$

So the quotient is $q=d_k\cdot b^{k-1}+d_{k-1}\cdot b^{k-2}+\cdots+ d_2\cdot b+d_1$, and
and the remainder is the base $b$ digit $d_0$ of $n$. So we have found the units digit
in the base $b$ expansion of $n$. If we repeat that process on the quotient $q$, the result
is 

$$
q =  (d_k\cdot b^{k-2}+d_{k-1}\cdot b^{k-3}+\cdots+ d_2) b+ d_1
$$
so the next base $b$ digit, $d_1$ appears as the remainder. Continuing in this fashion,
the base $b$ expansion is produced one digit at a time.

Briefly, to convert a positive
decimal integer $n$ to its base $b$ representation, divide $n$ by $b$,
to find the quotient and the remainder.  That remainder will be needed units
digit.  Then divide the quotient by $b$ again, to get a new
quotient and a new remainder. That remainder gives the next base $b$
digit. Then divide the new quotient by $b$ again,  and so on. In this way 
producing the
base $b$ digits  one after the other. 
\ms

\begin{exmp}
To convert $14567$ from decimal to base $5$, the steps are:
\begin{align*}
 14567 & =2913\cdot 5 + 2 \\
 2913 &= 582\cdot 5 + 3 \\
 582 & = 116\cdot 5 + 2  \\
 116 & = 23\cdot 5 + 1 \\
 23 & = 4\cdot 5 + 3 \\
 4 & = 0\cdot 5 + 4.
\end{align*}
So, we see that $14567=431232_5$.
\end{exmp}

\section{Converting between non-decimal bases}
\begin{exmp}
Convert $n=3355_7$ to base $5$.

The least confusing way to do such a problem would be to convert $n$ from base $7$ to base $10$,
and then convert the base $10$ expression for $n$ to base $5$. This method allows us to do
all our work in base $10$ where we are comfortable. The computations start with:
\[
n = 3\cdot7^3+3\cdot7^2+5\cdot7+5 = 1216.
\]
Then, we calculate: 
\begin{align*}
 1216 & =243\cdot 5 + 1 \\
 243 &= 48\cdot 5 + 3 \\
 48  & = 9\cdot 5 + 3  \\
 9 &= 1\cdot 5 +4 \\
 1 & = 0\cdot 5 + 1.
\end{align*}
So, we have $3355_7=14331_5$.
\end{exmp}

An alternative method, not for the faint of heart, is convert directly from base $7$ to
base $5$ skipping the middle man, base $10$. In this method, we simply divide
$n$ by $5$, take the remainder, getting the units digit, then divide the quotient by
$5$ to get the next digit, and so on, just as described above. The rub is that the arithmetic
must all be done in base $7$, and we don't know the base $7$ times table very well.
For example, in base $7$, $3\cdot 5 = 21$ is correct since three $5$'s add up to two $7$'s
plus one more.

The computation would now look like (all the $_7$'s indicating base $7$ are suppressed
for readability):
\begin{align*}
 3355 &= 465\cdot5+ 1 \quad\text{ (yes, that's really correct!),} \\
 465  &= 66\cdot 5 + 3 \\
 66  & =12\cdot 5 + 3  \\
 12 &= 1\cdot 5 +4 \\
 1 & = 0\cdot 5 + 1.
\end{align*}
Hence, once again, we have  $3355_7=14331_5$.



\section{Computer science bases: $2$, $8$, and $16$}
Particularly important in computer science applications of
discrete mathematics are the bases $2$ (called binary), $8$ (called octal)
and $16$ (called hexadecimal, or simply hex). Thus the decimal number
$75$ would be $1001011_2$ (binary), $113_8$ (octal) and $4B_{16}$ in
hex. Note that for hex numbers, symbols will be needed to represent hex
digits for $10$, $11$, $12$, $13$, $14$ and $15$.  The letters
$A,B,C,D,E$ and $F$ are traditionally used for these digits.

\clearpage

\section{Exercises}

\begin{exer}
Convert to decimal: $21_3$, $321_4$,  $4321_5$, and $FED_{16}$.
\end{exer}

\begin{exer}
Convert the decimal integer $11714$ to bases $2$, $6$, and $16$.
Remember to use $A,B, \cdots, F$ to represent base $16$ digits from $10$
to $15$, if needed.
\end{exer}

\begin{exer}\label{exer:base 7 mult tbl}
Complete the following \emph{base $7$} multiplication table. 
\begin{table}
\centering
\renewcommand{\arraystretch}{1.75}
\begin{tabular}%
{|>{\raggedleft\arraybackslash}p{0.5cm}||*{6}{>{\raggedleft\arraybackslash}p{0.5cm}|}}
\hline
$\times$ & $1$ & $2$ & $3$ & $4$ & $5$ & $6$ \\ \hline\hline
     $1$ & $1$ & $2$ & $3$ & $4$ & $5$ & $6$ \\ \hline
     $2$ & $2$ & $4$ &     &     &     &     \\ \hline
     $3$ &     &     & $12$ &     &     &     \\ \hline
     $4$ &     &     &     & $22$ &     &     \\ \hline
     $5$ &     &     &     &     &     &     \\ \hline
     $6$ &     &     &     &     &     &     \\ \hline
\end{tabular}
\end{table}
\end{exer}

\begin{exer}
Make \emph{base $6$} addition and multiplication tables similar to the base $7$
multiplication table of exercise~\ref{exer:base 7 mult tbl}.
\end{exer}

\begin{exer}
(\textbf{For those with a sweet tooth for punishment!}) 
Use the Euclidean algorithm
to compute $gcd(5122_7, 1312_7)$ without converting the numbers to base $10$.
\end{exer}

\section{Problems}

\begin{prob}
Convert to decimal: $12_3$, $123_4$,  $1234_5$, and $DDD_{16}$.
\end{prob}

\begin{prob}
Convert the decimal integer $3177$ to bases $2$, $6$, and $16$.
Remember to use $A,B, \cdots, F$ to represent base $16$ digits from $10$
to $15$, if needed.
\end{prob}

\begin{prob}\label{prob:base 8 mult tbl}
Make \emph{base $8$} addition and multiplication tables similar to the base $7$
multiplication table of exercise~\ref{exer:base 7 mult tbl}.
\end{prob}

\begin{prob}
Using the table in  problem~\ref{prob:base 8 mult tbl}, add  $126_8 + 457_8$.
\end{prob}

\begin{prob}
Using the table in  problem~\ref{prob:base 8 mult tbl}, multiply $(126_8)(457_8)$.
\end{prob}

\begin{prob}
Determine all choices of base $b\geq 2$ such that when the decimal number $100$ is written in
base $b$, the units digit is $0$.
\end{prob}

\begin{prob}
Determine all choices of base $b\geq 2$ such that when the decimal number $100$ is written in
base $b$, the units digit is $1$.
\end{prob}

