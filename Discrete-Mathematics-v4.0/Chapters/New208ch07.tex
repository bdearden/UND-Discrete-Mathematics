\chapter{Styles of Proof}



\newthought{
Earlier, we practiced proving} the validity of logical arguments, both with and without quantifiers.
The technique introduced there is one of the main tools for constructing proofs in a more
general setting.  In this chapter, various common styles of proof in mathematics are described. 
Recognizing these styles of proof will make both  reading  and constructing proofs a little less 
onerous.\marginnote{The example proofs in this chapter will use some familiar facts about integers,
 which we will prove in a later chapter.}


\section{Direct proof}
As mentioned  before, the typical form of the statement of a theorem is: 
{\itshape if $a$ and $b$ and $c$ and $\cdots$,
then $d$}. The propositions $a$,$b$,$c$, $\cdots$ are called the hypotheses, and the
 proposition $d$ is called the
conclusion. The goal of the proof is to show that $(a\land b\land c\land \cdots)\to d$
is a true proposition. In the case of propositional logic, the only thing that matters is the 
{\itshape form} of a logical
argument, not the particular propositions that are involved. That means the proof can always
 be given in 
the form of a truth table.
In areas outside of propositional logic that is no longer possible. Now the content of the propositions
must be considered. In other words, what the words mean, and not merely how they are strung
 together, becomes
important.

Suppose we want to prove an implication {\bfseries Theorem:} {\itshape If $p$, then $q$}. 
In other words,
we want to show $p\to q$ is true.
There are two possibilities: Either $p$ is false, in which case $p\to q$ is automatically
true, or $p$ is true. In this second case, we need to show that $q$ is true as well to 
conclude $p\to q$ is true.  In other words, to show $p\to q$ is true, we can
begin by assuming $p$ is true, and then give an {\itshape argument} that
$q$ must be true as well.
The outline of such a proof will look like:

\medskip
\framebox[\linewidth][l]{ %
\vbox{\vspace*{0.25cm}{\bfseries Proof:}
\begin{center}
\begin{tabular}{l l}
Step 1) &$\qquad$ Reason 1 \\
Step 2) &$\qquad$ Reason 2 \\
$\quad\vdots$  & $\qquad\quad \vdots$ \\
Step $l$) &\qquad Reason $l$
\end{tabular}
\end{center}
$\clubsuit$
}}
\medskip



Every step in the proof must be a true proposition, and since the goal is to conclude $q$ is true,
 the proposition $q$ will be the last step in the proof.
{\bfseries There are only  four acceptable reasons} that can be invoked to
 justify a step in a proof.
Each step can be: (1) a {\itshape hypothesis} (and so assumed to be true), (2) an application of
a {\itshape definition}, (3) a {\itshape known fact} proved previously, and so known to be true,
 or (4) a consequence
of applying a {\itshape rule of inference  or a logical equivalence} to 
earlier steps in the proof.
The only difference between these sorts of formal proofs and the proofs of logical
arguments we practiced earlier is the inclusion of definitions as a justification
of a step.

Before giving a few examples, there is one more point to consider. Most theorems in
 mathematics involve variables in some
way, along with either universal or existential quantifiers. But, in the case of universal quantifiers,
 tradition dictates
that the mention of the quantifier is often  suppressed, and left for the reader to fill in.
For example consider: {\bfseries Theorem:} {\itshape If $n$ is an even integer, then $n^2$ is an even integer}.
The statement is really shorthand for 
{\bfseries Theorem:} {\itshape For every $n\in \Z$, if $n$ is even, then $n^2$ is even}.
If we let $E(n)$ be the predicate {\itshape $n$ is even} with universe of discourse $\Z$, the
 theorem becomes  
{\bfseries Theorem:} {\itshape $\forall{n} (E(n)\to E(n^2))$.} The truth of such a universally quantified 
statement can be
accomplished with an application of the rule of universal generalization. In other words, we 
prove that for an arbitrary $n\in\Z$, the proposition $E(n)\to E(n^2)$ is true.
The result is stated and proved in the next theorem.

\clearpage
\begin{thm}
If $n$ is an even integer, then $n^2$ is an even integer.
\begin{proof}
 \begin{table*}
 \begin{tabular}{p{5cm} l}
1) $n$ is an even integer    & hypothesis \\
2) $n=2k$ for an integer $k$ & definition of even \\
3) $n^2 = 4k^2$              & algebra fact \\
4) $n^2 = 2(2k^2)$           & algebra fact \\
5) $n^2$ is even             & definition of even
 \end{tabular}
\end{table*}
\end{proof}
\end{thm}


Usually proofs are not presented in the dry  stepwise style of the last example. Instead, a
more narrative style is used. So the above proof could go as follows:

\begin{proof}
Suppose $n$ is an even integer. That means $n=2k$ for some integer $k$. 
Squaring both sides gives $n^2=(2k)^2 = 4k^2 = 2(2k^2)$ which shows $n^2$ is even.
\end{proof}
%%%%%%%%%%%%%%%%%%%%%%%%%%%%%%%%


All the ingredients of the stepwise proof are present in the narrative form, but this second form
 is a little more reader friendly. For example, we can include a few 
comments, such as {\itshape squaring both sides gives} to help the reader figure
out what is happening. 

The method of proof given above is called {\bfseries direct proof}. The characteristic
feature of a direct proof is that in the course of the proof, the hypotheses appear
as steps, and the last step in the proof is the conclusion of the theorem.  

It is traditional to put a marker (such as \qed, to indicate the theorem
has been {\bfseries clubbed}!) at the end of a narrative form of 
a proof to let the reader know the proof is complete.

Here is one more example of a direct proof.
\begin{thm}
If $n$ and $m$ are odd integers, then $n+m$ is even.
\end{thm} 
\begin{proof}
Suppose $m$ and $n$ are odd integers. That means $m= 2j+1$ for
some integer $j$, and $n=2k+1$ for some integer $k$. Adding gives
$m+n = (2j+1)+(2k+1) = 2j+2k+2 = 2(j+k+1)$, and so we see $m+n$ is even.
\end{proof}


\section{Indirect proof}
There are cases where a direct proof is not very convenient for one reason or 
another. There are several other styles of proof, each based on some logical
equivalence.

For example, since $p\to q \equiv \neg q\to \neg p$, we can prove
the
\begin{thm}
$p\to q$ 
\end{thm}
\noindent by instead giving a proof  of
\begin{thm}
 $\neg q\to \neg p$.
\end{thm}

In other words, we replace the requested
implication with its contrapositive, and prove that instead. This method of
proof is called {\bfseries indirect proof}. Here's an example.
\begin{thm}
If $m^2$ is an even integer, then $m$ is an even integer. 
\end{thm}
\begin{proof}
 Suppose $m$ is not even. Then $m$ is odd. So
 $m=2k+1$ for some integer $k$. Squaring both sides of that equation gives
 $m^2 = (2k+1)^2 = 4k^2+4k+1 = 2(2k^2+2k) + 1$, which shows $m^2$ is not even.
\end{proof}

Notice that we gave a {\itshape direct proof} of the equivalent theorem:
{\itshape  If $m$ is not an even integer, then $m^2$ is not an 
even integer}.

\section{Proof by contradiction}
Another alternative to a direct proof is {\bfseries proof by contradiction}. 
In this method the plan is to replace the requested {\bfseries Theorem:} $r$ (where
$r$ can be any simple or compound proposition) with 
{\bfseries Theorem:} $\neg r \to \F$, where $\F$ is any proposition known to be false.
The reason proof by contradiction is a valid form of proof is that
$\neg r \to \F\equiv r$, so that showing  $\neg r \to \F$ is true is identical
to showing $r$ is true. Proofs by contradiction can be a bit more difficult to
discover than direct or indirect proofs. The reason is that in those two types of
proof, we know exactly what the last line of our proof will be. We know where we
want to get to. But in a proof by contradiction, we only know that we want to
end up with some (any) proposition known to be false. Typically, when writing a
proof by contradiction, we experiment, trying various logical arguments, hoping
to stumble across some false proposition, and so conclude the proof. For example,
consider the following.
\begin{thm}
 $\sqrt{2}$ is irrational.
\end{thm}
\noindent The plan is to replace the requested theorem with 
\begin{thm}
 If $\sqrt{2}$
 is rational, then $\F$ (some fact known to be false).
\end{thm}
\noindent And now, we may give a direct proof of this replacement theorem:

\begin{proof}
Suppose that $\sqrt{2}$ is rational. Then there exist integers 
$m$ and $n$
with $n\neq 0$,  so that $\dl{\sqrt{2}={m\over n}}$, with $\dl{m\over n}$ in 
lowest terms. Squaring both sides
gives $\dl{2={m^2\over n^2}}$. Thus
$m^2=2n^2$ and so $m^2$ is even. Therefore $m$ is even.
So $m=2k$ for some integer $k$. Substituting $2k$ for $m$ in $m^2=2n^2$
shows  $(2k)^2=4k^2=2n^2$. Which means that $n^2=2k^2$.
Therefore $n^2$ is even, which means  $n$ is even. 
Now since both $m$ and $n$ are even, they
have $2$ as a common factor. Therefore $\dl{m\over n}$ is in lowest terms and
it is not in lowest terms. $\ctrdct$.
\end{proof}

The symbol $\ctrdct$ (two arrows crashing into each other head on) 
denotes that we have reached a \emph{fallacy} ($\F$), a statement 
known to be false. It usually marks the end of a proof by contradiction.

In the next example, we will prove a proposition of the form $p\to q$ by
contradiction. The theorem is about real numbers $x$ and $y$.
\begin{thm}
 If $0<x<y$, then $\sqrt{x}<\sqrt{y}$.
\end{thm}
\noindent Think of the statement of the theorem  in the form  $p\to q$. The plan is to replace the
 requested theorem with
\begin{thm}
 $\neg(p\to q)\to \F$.
\end{thm}
\noindent But $\neg(p\to q)\equiv \neg(\neg p\lor q)
\equiv p\land \neg q$. So we will actually prove $(p\land \neg q)\to \F$.
In other words, we will prove (directly)
\begin{thm}
 If  $0<x<y$ and  $\sqrt{x}\geq\sqrt{y}$, then (some fallacy).
\end{thm}
\begin{proof}
 Suppose  $0<x<y$ and  $\sqrt{x}\geq\sqrt{y}$. Since
 $\sqrt{x}>0$, $\sqrt{x}\sqrt{x}\geq\sqrt{x}\sqrt{y}$, which is the same as
 $x\geq \sqrt{xy}$. Also, since $\sqrt{y}>0$, $\sqrt{y}\sqrt{x}\geq \sqrt{y}\sqrt{y}$,
 which is the same as $\sqrt{xy}\geq y$. Putting $x\geq \sqrt{xy}$ and $\sqrt{xy}\geq y$
 together,
 we conclude $x\geq y$. Thus $x<y$ and $x\geq y$. $\ctrdct$
\end{proof}

\section{Proof by cases}
The only other common style of proof is {\bfseries proof by cases}. 
Let's first look at the justification for this proof technique.
Suppose we are asked to prove

\begin{thm}[Theorem X]
$p\to q$.
\end{thm}
We {\itshape dream up} some propositions, $r$ and $s$, and replace the requested
theorem with  three theorems:

\begin{thm}[Theorem XS]
\begin{enumerate*}
\item $p\lra(r\lor s)$,
\item $r\to q$, and
\item $s\to q$.
\end{enumerate*}
\end{thm}\label{thm: cases thm XS}
The propositions $r,s$ we dream up are called the {\itshape cases}. There can
be any number of cases. If we dream up three cases, then we would have
four theorems to prove, and so on. The hope is that the proofs of these
replacement theorems will be much easier than a proof of the original theorem.
\sidenote{This is the {\itshape divide and conquer} approach to a proof.}

The reason proof by cases is a valid proof technique is that
\[[(p\lra(r\lor s))\land(r\to q)\land(s\to q)]\to (p\to q)\]
is a tautology\sidenote{Prove this!}. Proof by cases, as for proof by contradiction, is generally
a little trickier than direct and indirect proofs. In a proof by contradiction,
we are not sure exactly what we are shooting for. We just hope some contradiction
will pop up. For a proof by cases, we have to dream up the cases to use, and it
can be difficult at times to dream up good cases.

\begin{thm}
  For any integer $n$, $|n|\geq n$.
\end{thm}
\begin{proof}
 Suppose $n$ is an integer. There are two cases:\marginnote{This has the form $p\lra(r\lor s)$ of
  $(1)$ in Theorem~\ref{thm: cases thm XS}.}%
 Either  (1): $n>0$, or (2): $n\leq 0$.
\begin{enumerate}
  \item[Case 1:] We need to show {\itshape If $n>0$, then $|n|\geq n$}.
   (We will do this with a direct proof.) Suppose $n>0$. Then $|n|=n$. Thus
   $|n|\geq n$ is true.
   
   \item[Case 2:] We need to show  {\itshape If $n\leq 0$, then $|n|\geq n$}.
   (We will again use a direct proof.) Suppose $n\leq 0$. Now $0\leq |n|$. Thus,
   $n\leq |n|$.
\end{enumerate}
 
 So, in any case, $n\leq |n|$ is true, and that proves the theorem.
\end{proof}

\section{Existence proof}
A proof of a statement of the form $\exists x P(x)$ is called an {\bfseries existence proof}.
The proof may be {\bfseries constructive}, meaning that the proof provides a specific example of,
or at least an explicit
recipe  for finding,  an $x$ so that $P(x)$ is true; or the proof may be 
{\bfseries  non-constructive},
meaning  that it establishes the existence of $x$ without giving a method of actually producing
an example of an $x$ for which $P(x)$ is true. 

To give examples of each type of existence proof, let's use a familiar fact (which will be proved
a little later in the course): There are infinitely many primes.  Recall that a prime is an integer
greater than $1$ whose only positive divisors are $1$ and itself.
The next two theorems are 
contrived, but they demonstrate the ideas of constructive and nonconstructive proofs.

\begin{thm}
 There is a prime with more than two digits.
\end{thm}
\begin{proof}
 Checking shows that $101$ has no positive divisors besides $1$ and itself. Also,
 $101$ has more than two digits. So we have produced an example of a prime with
 more than two digits.
\end{proof}

That is a constructive proof of the theorem. Now, here is a non-constructive proof
of a similar theorem.

\begin{thm}
 There is a prime with more than one billion digits.
\end{thm}
\begin{proof}
 Since there are infinitely many primes, they cannot all have one billion
 or fewer digits.
 So there must some primes with more than one billion digits.
\end{proof}


\section{Using a counterexample to disprove a statement}
Finally, suppose we are asked to prove a theorem of the form $\forall x\ P(x)$, and 
for one reason or another we come to believe the proposition is not true. The proposition
can be shown to be false by exhibiting a specific element from the domain of $x$ for
which $P(x)$ is false. Such an example is called a {\bfseries counterexample} to
the theorem. Let's look at a specific instance of the counterexample technique.

\clearpage
\begin{thm}[\bfseries not really!]
 For all positive integers 
$n$, $n^2-n+41$ is prime
\end{thm}
\begin{cntrxmpl}
To disprove the theorem,  we explicitly specify 
a positive integer $n$ such that $n^2-n+41$ is not prime. In fact, 
when $n=41$, the expression is not a prime since clearly $41^2-41+41 = 41^2$
is divisible by $41$. So, $n=41$ is a counterexample to the proposition. 
\end{cntrxmpl}


An interesting fact about this example is that $n=41$ is the smallest counterexample.
For $n = 1,2,\cdots 40$, it turns out that  $n^2-n+41$ is a prime! This examples
shows the danger of checking a theorem of the form $\forall x\, P(x)$ for a few
(or a few billion!) values of $x$, finding $P(x)$ true for those cases, and concluding it
is true for every possible value of $x$.

\clearpage



For the purpose of these exercises and problems, feel free to use familiar facts and definitions  about integers.
For example: Recall, an integer $n$ is even if $n=2k$ for some integer $k$. And, an integer $n$ is odd
if $n=2k+1$ for some integer $k$.

\section{Exercises}

\begin{exer}
Give a direct proof that the sum  of two even integers is even. 
\end{exer}

\begin{exer}
Give an indirect proof that if the square of the integer $n$ is odd, then $n$ is odd.

\end{exer}

\begin{exer}
Give a proof by contradiction that the sum of a rational number and an irrational number is irrational. 
\end{exer}

\begin{exer}
Give a proof by contradiction that if $5n-1$ is odd, then $n$ is even.
\end{exer}



\begin{exer}
In Chapter \ref{ch:logic prop}, exercise \ref{ex:1.last}, you concluded that
{\itshape  If $x=2$, then $x^2-2x+1=0$} is not a proposition.
Using the convention given in this chapter, what would you say now, and
why?

\end{exer}

\begin{exer}
Give a counterexample to the proposition {\itshape Every positive
integer that ends with a $7$ is a prime.}

\end{exer}

\clearpage
\section{Problems}

\begin{prob}
Give a direct proof that the sum of  an even integer and an odd integer is odd. 

Hint: Start by letting $m$ be an even integer and letting $n$ be an odd integer. That means $m = 2k$ for some integer $k$ and $n = 2j+1$ for some integer $j$. You are interested in $m+ n$, so add them up and see what you get.  Why is the thing you get an odd integer (think about the definition of {\it odd})?
\end{prob}

\begin{prob}
Give a direct proof that the sum of two odd integers is even.
\end{prob} 

\begin{prob}
 Give an  indirect proof that if $n^3$ is even, then $n$ is even. Hint: Study the solution of a similar statement in the sample exercises for this lesson.
\end{prob}


\begin{prob}
 Give a proof by contradiction that if $3n+2$ is odd, then $n$ is odd.

Hint:  This is the problem in this set that gives the most grief.  Study the section
in  the  notes  where  the  mechanics  of  proving  a  statement  of  the  form
If P, then Q by contradiction is discussed.  Be sure you understand why the first line of the proof should be something like {\it Suppose $3n+ 2$ is odd {\bf and} $n$ is even.}
\end{prob}



\begin{prob}
Give an example of a predicate $P(n)$ about positive integers $n$, such that
$P(n)$ is true for every positive integer from $1$ to one billion, but which is 
never-the-less not true for all positive integers. (Hint: there is a really simple choice
possible for the predicate $P(n)$.) 
\end{prob}

\begin{prob}
The {\bfseries maximum} of two numbers, $a$ and $b$ is $a$ provided $a\geq b$. Notation: $\max(a,b) = a$. The {\bfseries minimum}
of $a$ and $b$ is $a$ provided $a\leq b$. Notation: $\min(a,b) = a$.  Examples: $\max(2,3) = 3$, $\max(5,0) = 5$, $\min(2,3) = 2$,
$\min(5,0) = 0$, $\max(4,4) = \min(4,4) = 4$.\\
Give a proof by cases that for any numbers $s,t$,
\[\min(s,t)+\max(s,t) = s+t.\]
\end{prob}

\begin{prob}
Give a proof by cases that for integers $m,n$, we have $|mn|=|m| |n|$.
Hint: Consider four cases: (1) $m\geq 0$ and $n\geq 0$, (2) $m\geq 0$ and $n<0$,
(3) $m<0$ and $n\geq 0$, and (4) $m<0$ and $n<0$.
\end{prob}

