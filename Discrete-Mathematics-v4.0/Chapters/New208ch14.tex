\chapter{Recursively Defined Sequences}

\newthought{
Besides specifying the terms} of  a sequence with a formula, such as $a_n = n^2$, 
an alternative is to give an initial term, usually something like $b_1$, (or the first few terms, $b_1$, $b_2$,$b_3$\dots) of a sequence, and
then give a rule for building new terms from old ones. In this case, we say the sequence
has been defined {\bfseries recursively}. 

\begin{exmp}
 For example, suppose $b_1=1$, and for $n>1$, $b_n = 2b_{n-1}$. Then the $1^{st}$ 
term of the sequence will be $b_1=1$ of course. To determine $b_2$, we apply the
rule $b_2 = 2b_{2-1} = 2b_1 = 2\cdot 1 = 2$. Next, applying the rule again, 
 $b_3= 2b_{3-1} = 2b_2 = 2\cdot2=4$. Next $b_4 = 2b_3 = 8$. Continuing in this fashion,
we can form as many terms of the sequence as we wish: $1,2,4,8,16,32,\cdots$.
In this case, it is easy to guess a formula for the terms of the sequence: 
$b_n=2^{n-1}$.
\end{exmp}

In general, to define a sequence recursively, (1) we first give one or more initial terms
(this information is called the {\bfseries initial condition(s)} for the sequence), and
then (2) we give a rule for forming new terms from previous terms (this rule
is called the {\bfseries recursive formula}).

\begin{exmp}\label{exmp:recursive seq}
 Consider the sequence defined recursively by $a_1 = 0$, and, for
 $n\geq 2$, $a_n = 2a_{n-1} +1$. The five terms of this sequence are
 \[
 0,\quad 2\cdot0+1=1,\quad 2\cdot1+1=3,\quad 2\cdot3+1 = 7,
   \quad 2\cdot7+1=15\quad \cdots
 \]


 In words, 
we can describe this sequence by saying the initial term is $0$ and each
new term is one more than twice the previous term.
Again, it is easy to guess a formula that produces the terms of this sequence:
$a_n = 2^{n-1}-1$.  
\end{exmp}
Such a formula for the terms of a sequence is called a {\bf
closed form formula} to distinguish it from a recursive formula.

\section{Closed form formulas}
There is one big advantage to knowing a closed form formula for a sequence.
In  example \ref{exmp:recursive seq} above, the closed form formula for the sequence tells us
immediately that $a_{101} = 2^{100}-1$, but using the recursive formula
to calculate $a_{101}$  means we have to calculate in turn $a_1,a_2,\cdots a_{100}$,
making $100$ computations. The closed form formula allows us to jump directly
to the term we are interested in.  The recursive formula forces us to compute
$99$ additional terms we don't care about in order to get to the one we want. 
With such a
major drawback why even introduce recursively defined sequences at all? The answer
is that there are many naturally occurring sequences that have simple recursive 
definitions but have no reasonable closed form formula, or even no closed form formula 
at all in terms of familiar operations. In such cases, a recursive definition is better than nothing.

\subsection{Pattern recognition}
There are methods for determining closed form formulas for some special 
types of recursively defined sequences. Such techniques are  studied 
later in chapter~\ref{chpt:solns to recur rels}.
For now we are only interested in  understanding recursive definitions, and 
determining some closed form formulas by the method of 
\emph{pattern recognition} (aka \emph{guessing}).

\subsection{The Fibonacci Sequence}
The most famous recursively defined sequence is due to Fibonacci. There
are two initial conditions: $f_0 = 0$ and $f_1=1$. %
\marginnote{The index starts at \textbf{zero}, by tradition.} The recursive rule is,
for $n\geq 2$, $f_n = f_{n-1}+f_{n-2}$. In words, each new term is the sum
of the two terms that precede it. So, the {\bfseries Fibonacci sequence} begins 
\[
0,\ 1,\ 1,\ 2,\ 3,\ 5,\ 8,\ 13,\ 21,\ 34,\ 55,\ 89,\ 144,\ 233,\ \cdots
\]
There is a closed form formula for the Fibonacci Sequence, 
but it is not at all easy to guess:
\[
f_n = {1\over{\sqrt5}}\left({1+\sqrt{5}}\over 2\right)^n 
-{1\over{\sqrt5}}\left({1-\sqrt{5}}\over 2\right)^n
\]
 
 \subsection{The Sequence of Factorials}
For a positive integer $n$,  the symbol  $n!$ is 
read $\mathbf{n}$ \textbf{factorial} and it is defined to be the product of
all the positive integers from $1$ to $n$.\marginnote{For example, 
$5! = 1\cdot2\cdot3\cdot4\cdot5 = 120$.} In order to make many
formulas work out nicely, the value of $0!$ is defined to be $1$.

A  recursive formula can be given for $n!$. The initial term is
$0!=1$, and the recursive rule is, for $n\geq 1$, $n! = n[ (n-1)!]$.
Hence, the first few factorial values are:
\begin{align*}
 1! &= 1[0!] = 1\cdot 1  = 1, \\
 2! &= 2[1!] = 2\cdot 1 = 2, \\ 
 3! &= 3[2!] = 3\cdot 2 = 6,  \\
 4! &= 4[3!] = 4\cdot 6 = 24, \\
    &\phantom{~}\vdots
\end{align*}

We sometimes write a {\itshape general} formula for the factorial as %
\marginnote{Why is this \textbf{not} a closed form formula?}
\[
n!=1\cdot2\cdot3\cdot4\cdots n, \text{ for $n>0$.}
\]

The sequence of factorial grows very quickly. Here are the first few terms:
\[
1,\ 2,\ 6,\ 24,\ 120,\ 720,\ 5040,\ 40320,\ 362880,\ 3628800,\ 39916800,\ 479001600,\
6227020800,\ \cdots
\]

\section{Arithmetic sequences by recursion}
Consider the terms of an arithmetic sequence with initial term $a$ and common difference
$d$:
\[
a,\  (a+d),\  (a+2d),\cdots,\ (a+(n-1)d),\cdots.
\]
These terms may clearly  be found by adding $d$ to the current term to get the next. That is,
the arithmetic sequence may be defined recursively as (1) $a_1=a$, and (2) for $n\geq 2$,
$a_n = a_{n-1}+d$.

\clearpage

\section{Exercises}

\begin{exer}
List the first five terms of the sequence defined recursively by
$a_{1} = 3$, and, for $n\geq 2$, $a_{n} = a_{n-1}(2+a_{n-1})$.
\end{exer}

\begin{exer}
List the first seven terms of the sequence defined recursively by
$a_{0} = 1$, $a_{1}= 1$, and, for $n\geq 2$, $a_{n} = 1+a_{n-1}a_{n-2}$.
\end{exer}

\begin{exer}
List the first ten terms of the sequence defined recursively by
$a_{0} = 1$, and, for $n\geq 1$, $a_{n} = 1+a_{\lfloor{\frac{n}{2}}\rfloor}$.
\end{exer}

\begin{exer}
List  the first ten terms of the sequence defined recursively by
$a_{0} = 1$, and for $n\geq 1$, $a_n = 2n-a_{n-1}-1$, and guess
a closed form formula for $a_n$.
\end{exer}

\begin{exer}
The first few terms of a sequence are 
\[
1,\, 11,\,21,\,1211,\,111221,\,312211,\,13112221,\,1113213211.
\]
There is an easy recursive rule for building  the terms of this sequence. Guess the next term.
\end{exer}

\begin{exer}
Let $d$ be a fixed real number.
 For a positive integer $n$,
the symbol $nd$ means the sum of $n$ $d$'s.  Give  a recursive definition of $nd$
analogous to the definition of $n!$ given in this chapter.
\end{exer}

\clearpage

\section{Problems}

\begin{prob}

 List the first five terms of the sequence defined recursively by
 $a_1 = 2$, and, for $n\geq 2$, $a_n = a_{n-1}^2 -1$.
 
\end{prob}

\begin{prob}
 
List the first five terms of the sequence defined recursively by
$a_1 =2$, and, for $n\geq 2$, $a_n = 3a_{n-1}+2$. Guess a closed
form formula %
\marginnote{Hint: This is a lot like  example \ref{exmp:recursive seq}.}
for the sequence.
  
\end{prob}

\begin{prob}
 
 List the first five terms of the sequence with initial terms
$u_0=2$ and $u_1=5$, and, for $n\geq 2$, $u_n = 5u_{n-1}-6u_{n-2}$.
Guess a closed %
\marginnote{Hint: The terms
are simple combinations of powers of $2$ and powers of $3$.}
 form formula for the sequence. 
  
\end{prob}

\begin{prob} 

 Let $r$ be a fixed real number different from $0$.
 For a positive integer $n$,
the symbol $r^n$ means the product of $n$ $r$'s. For convenience,
$r^0$ is defined to be $1$. Give a recursive definition of $r^n$
analogous to the definition of $n!$ given in this chapter.
  
\end{prob}

\begin{prob}
 
 Give a recursive definition of the geometric sequence
with initial term $3$ and common ratio $2$.
  
\end{prob}

\begin{prob}
 
 Generalize problem 5: give a recursive definition of  the geometric sequence
with initial term $a$ and common ratio $r$.
  
\end{prob}
 

