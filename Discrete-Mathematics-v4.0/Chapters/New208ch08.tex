\chapter{Relations}\label{ch:Relations}


\newthought{Two-place predicates,} such as {\itshape $B(x,y) :$ $x$ is the brother of $y$}, play a 
central role in mathematics. Such predicates can be used to describe many basic concepts. 
As examples, consider the predicates given verbally:
\begin{enumerate}
 \item {\itshape $G(x,y) :$ $x$ is greater than or equal to $y$}
 which compares the  magnitudes of two values.
 
 \item {\itshape $P(x,y) :$ $x$ has the same parity as $y$} which
 compares the parity of two integers.
 
 \item {\itshape $S(x,y) :$ $x$ has square equal to $y$} which relates
 a value to its square.
\end{enumerate}

\section{Relations}
Two-place predicates are called {\bfseries relations}, probably because of examples
such as the {\itshape brother of} given above. To be a little more complete about it,
if $P(x,y)$ is a two-place predicate, and the domain of discourse for $x$ is the set $A$,
and the domain of discourse for $y$ is the set $B$, then $P$ is called a {\bfseries relation
from $A$ to $B$}.  When working with relations, some new vocabulary is used.
The set $A$ (the domain of discourse for the first variable) is called the {\bfseries domain}
of the relation, and the set $B$ (the domain of discourse for the second variable) is
called the {\bfseries codomain} of the relation.

\section{Specifying a relation}
There are several different ways to specify a relation. One way is to give a verbal
description as in the examples above. As one more example of a verbal description
of a relation, consider 

{\itshape $E(x,y) :$ The word $x$ ends with the letter $y$}. Here the domain
will be words in English, and the codomain will the the twenty-six letters of the alphabet.
We say the ordered pair $(cat,t)$ {\bfseries satisfies} the relation $E$, but that
$(dog,w)$ does not.

\subsection{By ordered pairs}
When dealing with abstract relations, a verbal description is not always convenient. An
alternate method is to tell what the domain and codomain are to be, and then simply list 
the ordered pairs which will satisfy the relation.  For example, 
if $A=\{1,2,3,4\}$ and $B=\{a,b,c,d\}$, then one of many possible  relations from $A$ to $B$
would be $\{(1,b),(2,c),(4,c)\}$. If we name this relation $R$, we will write
$R=\{(1,b),(2,c),(4,c)\}$. It would be tough to think of a natural verbal description
of $R$. 

When thinking of a relation, $R$, as a set of ordered pairs, it is common to write
 $aRb$ in place of $(a,b)\in R$.
For example, using the relation $G$ defined above, we can convey the fact that
the pair $(3,2)$ satisfies the relation by writing any one of the following: (1) $G(3,2)$ is true,
(2) $(3,2)\in G$, or (3) $3G2$. The third choice is the preferred one when discussing
relations abstractly.  


Sometimes the  ordered pair representation of a relation can be a bit cumbersome compared to the
verbal description. Think about the ordered pair form of the relation $E$ given above:
$E=\{\,(cat,t), (dog,g), (antidisestablishmentarianism, m), \cdots\,\}$. 

\subsection{By graph}
Another way
represent a relation is with a {\bfseries graph}\sidenote{Here graph does \textbf{not} mean the sorts of
graphs of  lines, curves and such discussed in an algebra course.}. Here, a  graph is a diagram made
up of dots, called {\bfseries vertices}, some of which are joined by lines, called 
{\bfseries edges}.  To draw a graph of a relation $R$ from $A$ to $B$, make
a column of dots, one for each element of $A$, and label the dots with the names of those elements. Then, to the right of $A$'s column  make a column of dots
for the elements of $B$. 
Then connect the vertex labelled $a\in A$ to a vertex
$b\in B$ with an edge provided $(a,b)\in R$. 
The diagram  is called the {\bfseries bipartite graph 
representation} of $R$.

\begin{exmp}\label{ex:bipgraph 8.1}
 Let $A=\{1,2,3,4\,\}$ and $B=\{a,b,c,d\,\}$, and let 
$R=\{(1,a), (2,b), (3,c), (3,d), (4,d)\}$.
Then the bipartite graph which represents $R$ is given in figure \ref{fig:bipgraph 8.1}.

\begin{marginfigure}
\definecolor{myblue}{RGB}{80,80,160}
\definecolor{mygreen}{RGB}{80,160,80}

\begin{tikzpicture}[thick,
  every node/.style={draw,circle},
  fsnode/.style={fill=myblue},
  ssnode/.style={fill=mygreen},
  every fit/.style={ellipse,draw,outer sep=1pt,inner sep=1pt,text width=2cm},
  ->,shorten >= 3pt,shorten <= 3pt
]

% the vertices of A
\begin{scope}[start chain=going below,node distance=5mm]
\foreach \i in {1,2,3,4}  %{1,2,...,4}
  \node[fsnode,on chain] (f\i) [label=left: \i] {};
\end{scope}

% the vertices of V
\begin{scope}[xshift=3cm,start chain=going below,node distance=5mm]
\foreach \i in {a,b,...,d}  %{a,b,c,d}
  \node[ssnode,on chain] (s\i) [label=right: \i] {};
\end{scope}

% the set A
\node [myblue,fit=(f1) (f4),label=above:$A$] {};
% the set B
\node [mygreen,fit=(sa) (sd),label=above:$B$] {};

% the edges
\draw [-] (f1) -- (sa);
\draw [-] (f2) -- (sb);
\draw [-] (f3) -- (sc);
\draw [-] (f3) -- (sd);
\draw [-] (f4) -- (sd);
\end{tikzpicture}
\caption{Example bipartite graph}\label{fig:bipgraph 8.1}
\end{marginfigure}
\end{exmp}


The choices made about the  ordering and the placement of the vertices for the elements of 
$A$ and $B$ may make a difference in 
the appearance of the graph, but all such graphs are considered equivalent. Also, edges can be 
curved lines. All that matters is that such diagrams  
convey graphically the same information as $R$ given as a set of ordered pairs.

\subsection{By digraph: domain$=$codomain}
It is common to have the domain and the codomain of a relation be the same set. If $R$ is a 
relation from $A$ to $A$, then we will say $R$ is a  {\bfseries relation on $A$}.
In this case there is a shorthand way of representing the relation by using 
a {\bfseries digraph}. The word digraph is shorthand for {\itshape directed graph} 
meaning the edges have a direction indicated by an arrowhead. Each element of $A$ is used to label a single point.
An arrow connects the vertex labelled $s$ to the one labelled $t$ provided
$(s,t)\in R$. An edge of the form $(s,s)$ is called a {\bfseries loop}.

\begin{exmp}
 Let $A=\{1,2,3,4,5\}$ and \\
 $R=\{(1,1),(1,2),(2,2),(2,3),(3,3),(3,4),(4,4)\}$.
Then a digraph for $R$ is shown in figure \ref{fig:bipgraph 8.2}
\begin{marginfigure}
\definecolor{myblue}{RGB}{80,80,160}
\definecolor{mygreen}{RGB}{80,160,80}

\begin{tikzpicture}[->,>=stealth',node distance=2cm,
                    thick,main node/.style={circle,fill=blue!20,draw,outer sep=5pt}
                   ]

  \node[main node] (1) {1};
  \node[main node] (5) [below right of=1] {5};
  \node[main node] (2) [above right of=5] {2};
  \node[main node] (3) [below right of=5] {3};
  \node[main node] (4) [below left  of=5] {4};

  \path%
    (1) edge node [right] {} (2)  %{} is where the edge value would go
        edge [loop left] node {} (1)
    (2) edge [loop right] node {} (2)
        edge [right] node {} (3)
    (3) edge [right] node {} (4)
        edge [loop right] node {} (3)
    (4) edge [loop left] node {} (4);
\end{tikzpicture}
\caption{Example digraph}\label{fig:bipgraph 8.2}
\end{marginfigure}
\end{exmp}

Again it is true that a different placement of the vertices may yield a different-looking, but equivalent, digraph.


\subsection{By $0$-$1$ matrix}
The last method for representing a relation is by using a $0${-}$1$ matrix. This method is particularly
handy for encoding a relation in computer memory.
An {\bfseries $m\times n$ matrix} is a rectangular array with $m$ rows and $n$ columns.
Matrices are usually denoted by capital English letters. The entries of a matrix, usually
denoted by lowercase English letters, are indexed by row and column. Either $a_{i,j}$ or $a_{ij}$
stands for the entry in a matrix in the $i$th row and $j$th column. 
A {\bfseries $0$-$1$ matrix} is one all of whose entries are $0$ or $1$. Given two 
finite sets $A$ and $B$ with 
$m$ and $n$ elements respectively, we may use the elements of $A$ (in some fixed order) to index the rows of an
$m\times n$ $0$-$1$ matrix, and use the elements of $B$ to index the columns. So for a relation $R$
from $A$ to $B$, there is a matrix of $R$, $M_{_R}$ with respect to the orderings of $A$ and $B$ which
represents $R$. The entry of $M_{_R}$ in the row labelled by $a$ and column labelled by $b$
is $1$ if $aRb$ and $0$ otherwise. This is exactly like using characteristic vectors to represent
subsets of $A\times B$, except that the  vectors are cut into $n$ chunks of size
$m$.

\begin{exmp}
 Let $A=\{1,2,3,4\}$ and $B=\{a,b,c,d\}$ as before, and consider the 
 relation $R=\{(1,a),(1,b),(2,c),(4,c),(4,a)\}$.  Then a $0$-$1$ matrix which represents 
 $R$ using the natural orderings of $A$ and $B$ is%
 \marginnote{Note: This matrix may change appearance if  $A$ or $B$ is listed in a different order.}%
\[ M_{_R}= %
 \left[
   \begin{matrix}
    1&1&0&0\\ 0&0&1&0\\ 0&0&0&0\\ 1&0&1&0 
   \end{matrix}
 \right] 
\]
\end{exmp}

\section{Set operations with relations}
Since relations can be thought of as sets of ordered pairs, it makes sense to
ask if one relation is a subset of another. Also, set operations such as union 
and intersection can be carried out with relations. 

\subsection{Subset relation using matrices}These notions can be
expressed in terms of the matrices that represent the relations.
Bit-wise operations on $0$-$1$ matrices are defined in the obvious way.
Then  $M_{_{R\cup S}}=M_{_R}\vee M_{_S}$, and   
$M_{_{R\cap S}}=M_{_R}\land M_{_S}$.
Also, for two $0$-$1$ matrices of the
same size $M\leq N$ means that wherever $N$ has a $0$ entry, the corresponding
entry in $M$ is also $0$. Then
$R\subseteq S$ means the same as  $M_{_R}\leq M_{_S}$.

\section{Special relation operations}
There are two new operations possible with relations.

\subsection{Inverse of a relation}
First, if $R$ is a relation from $A$ to $B$, then by reversing all the
ordered pairs in $R$, we get a new relation, denoted $R^{-1}$, called 
the {\bfseries inverse} of $R$.
In other words, $R^{-1}$ is the  relation from $B$ to $A$ given
by $R^{-1}=\{(b,a)|(a,b)\in R\}$. 
A bipartite graph for $R^{-1}$ can be obtained from a bipartite graph for $R$ 
simply by interchanging 
the two columns of vertices with their attached edges (or, by rotating the diagram $180^\circ$). 

If the matrix for $R$ is $M_{_R}$, then the matrix for $R^{-1}$ is produced
by taking the columns of $M_{_{R^{-1}}}$ to be the rows of $M_{_R}$. A matrix
obtained by changing the rows of $M$ into columns is called the 
{\bfseries transpose} of $M$, and written as $M^T$. So, in symbols, 
if $M$ is a matrix for $R$, then $M^T$ is a matrix for $R^{-1}$.

\subsection{Composition of relations}
The second operation with relations concerns the situation when 
$S$ is a relation from $A$ to $B$ and $R$ is a relation from $B$ to $C$.
In such a case, we can form the {\bfseries composition of $S$ by $R$} 
which is denoted
$R\circ S$. The composition is defined as
$$R\circ S=\{(a,c)|a\in A, c\in C \text{ and } \exists b\in B, \text{ such that } 
(a,b)\in S \text{ and } (b,c)\in R\}.$$

\begin{exmp}
 Let $A=\{1,2,3,4\}, B=\{\alpha, \beta\}$ and $C=\{a,b,c\}$.
 Further let $S=\{(1,\alpha), (1,\beta),$
  $(2,\alpha),(3,\beta), (4, \alpha)\}$ and $R=\{(\alpha, a), 
 (\alpha, c), (\beta, b)\}$. 
 
 Since $(1,\alpha)\in S$ and $(\alpha,a)\in R$, it
 follows that $(1,a)\in R\circ S$. Likewise, since $(2,\alpha)\in S$ and 
 $(\alpha,c)\in R$, it
 follows that $(2,c)\in R\circ S$.
 Continuing in that fashion shows that 
 $$R\circ S=\{(1,a),(1,b),(1,c), (2,a),(2,c),(3,b),(4,a),(4,c)\}.$$

\clearpage
The composition can also be determined by looking at the bipartite graphs.
Make a column of vertices for $A$ labelled $1,2,3,4$, then to the right a 
column of points for $B$
labelled $\alpha,\beta$, then again to the right a column of points for 
$C$ labelled $a,b,c$. Draw in the edges as usual for $R$ and $S$. Then 
a pair $(x,y)$ will be in $R\circ S$ provided there is a two edge path from
$x$ to $y$. (See figure \ref{fig:comp rels 8.3} at right.)

\begin{marginfigure}

\begin{tikzpicture}[thick,
  every node/.style={circle},
  Asnode/.style={fill=black!80,inner sep=0pt, minimum size=1.5mm},
  Bsnode/.style={fill=black!80,inner sep=0pt, minimum size=1.5mm},
 Csnode/.style={fill=black!80,inner sep=0pt, minimum size=1.5mm},
  ->,shorten >= 3pt,shorten <= 3pt
]

% the vertices of A
\begin{scope}[start chain=going below,node distance=5mm]
\foreach \i in {1,2,3,4}  %{1,2,...,4}
  \node[Asnode,on chain] (A\i) [label=left: \i] {};
\end{scope}

% the vertices of B
\begin{scope}[xshift=2cm,yshift=-7.75mm,start chain=going below,node distance=5mm]
%\foreach \i in {1,2}
  \node[Bsnode,on chain] (B1) [label=above: $\alpha$] {};
  \node[Bsnode,on chain] (B2) [label=below: $\beta$] {};
\end{scope}

% the vertices of C
\begin{scope}[xshift=4cm,start chain=going below,node distance=5mm]
\foreach \i in {a,b,c} 
  \node[Csnode,on chain] (C\i) [label=right: \i] {};
\end{scope}

% the edges
\draw [blue] (A1) -- (B1);
\draw [blue] (A1) -- (B2);
\draw [blue] (A2) -- (B1);
\draw [blue] (A3) -- (B2);
\draw [blue] (A4) -- (B1);
\draw [red] (B1) -- (Ca);
\draw [red] (B1) -- (Cc);
\draw [red] (B2) -- (Cb);

%label relations
\node[text=blue] at (10mm,2mm) {$S$};
\node[text=red] at (30mm,2mm) {$R$};

\end{tikzpicture}

\begin{tikzpicture}[thick,
  every node/.style={circle},
  Asnode/.style={fill=black!80,inner sep=0pt, minimum size=1.5mm},
  Bsnode/.style={fill=black!80,inner sep=0pt, minimum size=1.5mm},
 Csnode/.style={fill=black!80,inner sep=0pt, minimum size=1.5mm},
  ->,shorten >= 3pt,shorten <= 3pt
]

% the vertices of A
\begin{scope}[start chain=going below,node distance=5mm]
\foreach \i in {1,2,3,4}  %{1,2,...,4}
  \node[Asnode,on chain] (A\i) [label=left: \i] {};
\end{scope}


% the vertices of C
\begin{scope}[xshift=4cm,start chain=going below,node distance=5mm]
\foreach \i in {a,b,c} 
  \node[Csnode,on chain] (C\i) [label=right: \i] {};
\end{scope}

% the edges
\draw [purple] (A1) -- (Ca);
\draw [purple] (A1) -- (Cb);
\draw [purple] (A1) -- (Cc);
\draw [purple] (A2) -- (Ca);
\draw [purple] (A2) -- (Cc);
\draw [purple] (A3) -- (Cb);
\draw [purple] (A4) -- (Ca);
\draw [purple] (A4) -- (Cc);

\node[text=purple] at (20mm,-20mm) {$R \circ S$};

\end{tikzpicture}
\caption{Composing relations: $R \circ S$}\label{fig:comp rels 8.3}
\end{marginfigure}


From the picture it is instantly clear that, for example, $(1,c)\in R\circ S$.

In terms of $0$-$1$ matrices if $M_{_S}$ is the $m\times k$ matrix of $S$ with
 respect to the given orderings
of $A$ and $B$, and if $M_{_R}$ is the $k\times n$ matrix of $R$ with respect 
to the given orderings of $B$
and $C$, then whenever the $i,l$ entry of $S$ and $l,j$ entry of $R$ are both $1$, 
then $(a_i,c_j)\in R\circ S$.
\end{exmp}

\subsection{Composition with matrices: Boolean product}
This example motivates the definition of the {\bfseries Boolean product} of 
$M_{_S}$ and $M_{_R}$ as the corresponding
matrix $M_{_{R\circ S}}$ of the composition. More rigorously when $M$ 
is an $m\times k$ $0$-$1$ matrix
and $N$ is an $k\times n$ $0$-$1$ matrix, $M\odot N$ is the 
$m\times n$ $0$-$1$ matrix whose
$i,j$ entry is $(m_{i,1}\wedge n_{1,j})% 
\vee (m_{i,2}\wedge n_{2,j}) \vee ,..., (m_{i,k}\wedge n_{k,j})$.
This looks worse than it is. It achieves the desired result\sidenote{The boolean product is computed the same way as the ordinary matrix product where 
multiplication and addition have been replaced with \textsf{and} and \textsf{or}, respectively.}.

For the relations in the example above example
\[
M_{_{R\circ S}}=
  \left[ \begin{matrix} 1&1&1\\ 1&0&1\\ 0&1&0\\ 1&0&1 \end{matrix} \right]
   =
  \left[\begin{matrix} 1&1\\ 1&0\\ 0&1\\ 1&0 \end{matrix}\right]\odot 
    \left[\begin{matrix} 1&0&1\\ 0&1&0 \end{matrix}\right]=M_{_S}\odot M_{_R} %
\]

\clearpage

\section{Exercises}

\begin{exer}\label{exer:08.1}
 Let $A=\{a,b,c,d\}$ and \newline
 $R=\{(a,a),(a,c),(b,b),(b,d),(c,a),(c,c),(d,b),(d,d)\}$ be a relation on $A$.
 Draw a digraph which represents $R$. Find the matrix which represents $R$ 
{\bfseries {with respect to the ordering $(d,c,a,b)$.}}
\end{exer}

\begin{exer}\label{exer:08.2}
The matrix of a relation $S$ from $\{1,2,3,4,5\}$ to $\{a,b,c,d\}$ with respect to the given
orderings is displayed below. Represent $S$ as a bipartite graph, and as a set of ordered pairs.
\[
 M_{_S}=\left[\begin{matrix} 1&1&0&0\\ 0&0&1&1\\ 1&0&0&1\\ 1&0&1&0\\ 0&1&1&0 \end{matrix}\right]
\]
\end{exer}

\begin{exer} 
Find the composition of $S$ by $R$ (as given in exercises~\ref{exer:08.1} 
and \ref{exer:08.2}) as a set of ordered pairs. 
Use the Boolean product to find $M_{_{R\circ S}}$ with respect to the natural orderings.
\sidenote[][-0.25cm]{The natural ordering for R is \textbf{not} the ordering used in exercise \ref{exer:08.1}.}
\end{exer}

\begin{exer} 
 Let $B=\{1,2,3,4,5,6\}$ and let 
 \begin{align*}
   R_1=\{&(1,2),(1,3),(1,5),(2,1),(2,2),(2,4),(3,3),(3,4),\\
         &(4,1),(4,5),(5,5),(6,6)\}\text{ and }\\
   R_2=\{&(1,2),(1,6),(2,1),(2,2), (2,3),(2,5),(3,1),(3,3),(3,6),\\
         &(4,2),(4,3),(4,4),(5,1),(5,5),(5,6),(6,2),(6,3),(6,6)\}.
 \end{align*}
\begin{enumerate}[label=(\alph*)]
\item Find $R_1\cup R_2$, $R_1\cap R_2$, and $R_1\oplus R_2$. 
\item With respect to the given ordering of $B$ find the matrix of each relation in part (a)
\end{enumerate}
\end{exer}

\clearpage
\section{Problems}

\begin{prob}
 Let $A=\{a,b,c,d\}$ and $R=\{(a,a),(a,c),(b,d),(c,a),(c,c),(d,b)\}$ be a relation
on $A$. Draw a digraph which represents $R$. Draw the bipartite graph which
represents $R$.
\end{prob}

\begin{prob}
 Let $A=\{a,b,c,d\}$ and $R=\{(a,a),(a,c),(b,d),(c,a),(c,c),(d,b)\}$ be a relation
on $A$. What is the inverse of $R$?
\end{prob}

\begin{prob}
Find the composition, $R\circ S$,  where $S = \{\,(1,a), (4,a),
(5,b), (2,c),  (5,c), (3,d)\,\}$ with $R  =\{(a,x),(a,y),(b,x),(c,z),(d,z)\}$ as a set of ordered pairs.
\end{prob}

\begin{prob}
Let $R_1=\{(1,2),(1,3),(1,5),(2,1),(6,6)\}$ and\\
$R_2=\{(1,2),(1,6),(3,6), (4,2),(5,6),(6,2),(6,3)\}$. Find $R_1\cup R_2$ and $R_1\cap R_2$. 
\end{prob}

\begin{prob}
Let $L$ be the relation {\itshape less than } on the set of integers. Examples $3\,L\,7$ and $-8\,L\, 0$ are
true, but $5\,L\,2$ and $6\,L\,6$ are false. How would describe the relation $L^{-1}$?
\end{prob}

\begin{prob}
True or False: For any relation $R$, $(R^{-1})^{-1} = R$. Explain your answer.
\end{prob}

\begin{prob}
Are there relations $R$ for which $R=R^{-1}$? If not, explain why it is not possible. If so, give an example
of such a relation.
\end{prob}

\begin{prob}
Let $R$ be a relation on a set $A$, and let $R^{-1}$ be its inverse. Prove that if $(a,b)\in R\circ R^{-1}$,
then $(b,a)\in R\circ R^{-1}$.
\end{prob}

\begin{prob}
Let $A$ and $B$ be two sets. Explain why the empty set, $\emptyset$, is a relation from $A$ to $B$.
\end{prob}

\begin{prob}
Let $S$ be a relation from $A$ to $B$, and let $R$ be a relation from $B$ to $C$. Prove
$(R\circ S)^{-1} = S^{-1}\circ R^{-1}$.
\end{prob}