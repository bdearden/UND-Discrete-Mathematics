\chapter{The Method of Characteristic Roots}\label{chpt:char roots}

\newthought{There is no method} that will solve all recurrence relations. However, for one
particular type, there is a standard technique. The type is called a 
{\bf linear recurrence relation with constant coefficients}. In 
such a recurrence relation, the recurrence formula has the form
$$
a_n=c_1a_{n-1}+c_2a_{n-2}+...+c_ka_{n-k}+ f(n)
$$
where $c_1,\cdots,c_k$ are constants with $c_k\not=0$, and $f(n)$ is any function
of $n$.

The {\bfseries degree} of the recurrence is $k$, the number of terms we need to go back
in the sequence to compute each new term. If $f(n)=0$, then the recurrence relation
is called homogeneous. Otherwise it is called {\bfseries nonhomogeneous}.


In chapter~\ref{chpt:solns to recur rels}, we noted that some simple non-homogeneous 
linear recurrence relations with constant coefficients can be solved by unfolding.
This method is not powerful enough for more general problems. In this chapter we introduce a basic 
method that, in principle at least, can  be used to  solve any homogeneous linear recurrence relation 
with constant coefficients. 

\section{Homogeneous, constant coefficient recursions}
We begin by considering the degree $2$ case. That is, we have a recurrence relation of the form 
$a_n=c_1a_{n-1}+c_2a_{n-2}$, for $n\geq 2$, where $c_1$ and $c_2$ are real constants.
We must also have two initial conditions $a_0$ and $a_1$. That is, we are given 
$a_0$ and $a_1$ and the formula $a_n=c_1a_{n-1}+c_2a_{n-2}$, for $n\geq 2$. 
Notice that $c_2\neq 0$ or else we have a linear recurrence relation with constant coefficients 
and degree $1$. What we seek is a {\bf closed form formula} for $a_n$, which is a function
of $n$ alone, and which is therefore independent of the previous terms of the sequence.


\subsection{Basic example of the method}
Here's the technique in a specific example:

The problem we will solve is to find a formula for the terms of
the sequence
\begin{align*}
 a_0 &= 4 \text{ and } a_1 = 8, \text{ with} \\
 a_n &= 4a_{n-1}+12a_{n-2}, \text{for $n\geq2$.}
\end{align*}
The first thing to do is to ignore the initial conditions, and
concentrate on the recurrence relation. And the way to solve the
recurrence relation is to guess the solution. Well, actually, it is to
guess the {\bfseries form} of the solution\sidenote{An \emph{educated} guess!}. 
For such a recurrence you should
guess that the solution looks like $a_n=r^n$, for some constant $r$.
In other words, guess the solution is simply the powers of some fixed
number. The good news is that this guess will always be correct! You
will always find some solutions of this form. When this guess is plugged
into the recurrence relation and the equation is simplified, the result
is an equation that can be solved for $r$. That equation is called the
{\bfseries characteristic equation} for the recurrence.  In our example, when
$a_n=r^n$ for each $n$, the result is $r^n=4r^{n-1}+12r^{n-2}$, and
canceling $r^{n-2}$ from each term, and rearranging the equation, we get
$r^2-4r-12=0$. That's the characteristic equation. The left side can
be factored, and the equation then looks like $(r-6)(r+2)=0$, and we see
the solutions for $r$ are $r=6$ and $r=-2$. And, sure enough, if you
check it out, you will see that $a_n=6^n$ and $a_n=(-2)^n$ both satisfy
the given recurrence relation.  In other words, we find that
\begin{align*}
    6^n &=4\cdot6^{n-1}+12\cdot6^{n-2}, \text{ for all $n\geq2$, and} \\ 
 (-2)^n &=4\cdot(-2)^{n-1}+12\cdot(-2)^{n-2}, \text{ for all $n\geq2$.}
\end{align*}

Using the characteristic equation, we have a method of finding
some solutions to a recurrence relation.  This method will not find all
possible solutions however.  \emph{BUT...} if we find {\bfseries all} the solutions
to the characteristic equation, then they can be combined in a certain
way to produce all possible solutions to the recurrence relation. The
fact to remember is that if $r=a,b$ are the two solutions
to the characteristic equation (for a recurrence of order two), then
every possible solution to the linear homogenous recurrence relation
must look like \sidenote{Actually, that is not quite true. There
is a slight catch to be mentioned later (see section~\ref{sect:repeated roots}).}
\[
\alpha a^n+\beta b^n,
\]
for some constants $\alpha,\beta$. In the
example we have been working on, every possible solution looks like\sidenote{This expression is called the \textbf{general solution}
of the recurrence relation.}
\[
  a_n=\alpha (6)^n+\beta (-2)^n.
\]

Once we have figured out the general solution to the recurrence
relation, it is time to think about the initial conditions.  In our
case, the initial conditions are $a_0=4$  and $a_1=8$.  The idea is to
select the constants $\alpha$ and $\beta$ of the general solution
$a_n=\alpha 6^n+\beta (-2)^n$ so it will produce the correct two initial values.
For $n=0$ we see we need $4=a_0=\alpha 6^0+\beta (-2)^0 = \alpha+\beta$, and for
$n=1$, we need $8=a_1=\alpha 6^1+\beta (-2)^1 = 6\alpha-2\beta$. Now, we solve the 
following pair of equations for $\alpha$ and $\beta$:
\begin{align*}
  \alpha +  \beta &=4, \\
 6\alpha - 2\beta &=8.
\end{align*}
Performing a bit of algebra, we learn that $\alpha=2$ and
$\beta=2$.  Thus the solution to the recurrence is
\[
a_n=2\cdot6^n+2\cdot(-2)^n.
\]

\subsection{Initial steps: the characteristic equation and its roots}
The steps in solving a recurrence problem are:
\begin{enumerate}
 \item Determine the characteristic equation.
 \item Find the solutions to the characteristic equation.
 \item Write down the general solution to the recurrence relation.
 \item Select the constants in the general solution to produce the correct
 initial conditions.
\end{enumerate}


\section{Repeated characteristic roots.}\label{sect:repeated roots}
And now, about the little lie mentioned above: One catch with the method of
characteristic equation occurs when the equation has repeated roots. Suppose, for
example, that when the characteristic equation is factored the result is
$(r-2)(r-2)(r-3)(r+5)=0$.  The characteristic roots are $2,2,3$ and
$-5$. Here $2$ is a repeated root.  If we follow the instructions 
given above, then the general solution we would write down is
\begin{equation}\label{eqn:rep roots incorrect}
 a_n= \alpha 2^n+\beta 2^n+\gamma 3^n+\delta (-5)^n.
\end{equation}
However, this expression will {\bfseries not} include all possible solutions
to the recurrence relation. Happily, the problem is not too hard to
repair: each time a root of the characteristic equation is repeated,
multiply it by an additional factor of $n$ in the general solution, and
then proceed with step 4 as described earlier.

For our example, we modify one of the $2^n$ terms in equation~\ref{eqn:rep roots incorrect}. The correct general solution looks like\marginnote{Notice the extra factor of $n$ in the second term.}
\[
 a_n= \alpha 2^n+\beta n\cdot 2^n+\gamma 3^n+\delta (-5)^n.
\]


If $(r-2)$ had been a four fold factor of the characteristic
equation\sidenote{in other words, if $2$ had been a characteristic root four
times}, then the part of the general solution involving the $2$'s would
look like \marginnote{Each new occurrence of a $2$ is multiplied by one more factor of $n$.}
\[
 \alpha 2^n+\beta n\cdot2^n+\gamma n^2\cdot2^n+\delta n^3\cdot2^n.
\]  



\section{The method of characteristic roots more formally}
Let's describe the method of {\bfseries characteristic equation} a little more
formally.  First, the characteristic equation is  denoted  by $\chi(x)=0$ % 
\sidenote{For the general degree $2$ case above we have $\chi(x)=x^2-c_1x-c_2$.}. Notice
that the degree of $\chi(x)$ coincides with the degree of the recurrence relation.
Notice also that the non-leading coefficients of $\chi(x)$ are simply the negatives of the
coefficients of the recurrence relation. In general, 
the characteristic equation of $a_n=c_1a_{n-1}+...+c_ka_{n-k}$ is 
\[
 \chi(x)=x^k-c_1x^{k-1}-...-c_{k-1}x-c_{k}=0.
\]
A number $r$ (possibly complex) is a {\bfseries characteristic root} if $\chi(r)=0$. From basic algebra we know 
that $r$ is a root of a polynomial if and only if $(x-r)$ is a factor of the polynomial. 
When $\chi(x)$ is a degree $2$ polynomial, by the quadratic formula, either $\chi(x)=(x-r_1)(x-r_2)$, 
where $r_1\neq r_2$, or $\chi(x)=(x-r)^2$, for some $r$.
\begin{thm}
 Let $c_1$ and $c_2$ be real numbers. Suppose that the polynomial $\chi(x)=x^2-c_1x-c_2$
 has two distinct roots $r_1$ and $r_2$. Then a sequence $a:\N\to \R$ is a solution of the
 recurrence relation $a_n=c_1a_{n-1}+c_2a_{n-2}$, for $n\geq 2$ if and only if
  $a_m=\alpha r_1^m+\beta r_2^m$, 
 for all $m\in \N$, and for some constants $\alpha$ and $\beta$.\marginnote{The constants are determined by the initial conditions (see equation~\ref{eqn:alpha beta}).}
 \ms
\end{thm}
\begin{proof}
 If $a_m=\alpha r_1^m+\beta r_2^m$ for all $m\in \N$, where $\alpha$ and $\beta$ are some
 constants, then since $r_i^2-c_1r_i-c_2=0$, we have $r_i^2=c_1r_i+c_2$, for $i=1$
 and $n=2$.
 Hence, for $n\geq 2$, we have
 \begin{align*}
  c_1a_{n-1}+c_2a_{n-2}&= c_1(\alpha r_1^{n-1}+\beta r_2^{n-1}) + c_2(\alpha r_1^{n-2}+\beta r_2^{n-2})\\
  &= \alpha r_1^{n-2}(c_1r_1 + c_2) + \beta r_2^{n-2}(c_1r_2+c_2) \text{ distributing and combining,} \\
  &= \alpha r_1^{n-2}\cdot r_1^2 + \beta r_2^{n-2}\cdot r_2^2 \text{ by the remark above,} \\
  &= \alpha r_1^n+\beta r_2^n = a_n.
 \end{align*}
 Conversely, if $a$ is a solution of the recurrence relation and has initial terms $a_0$ and $a_1$, then
 one checks that the sequence $a_m=\alpha r_1^m+\beta r_2^m$ with 
 \begin{equation}\label{eqn:alpha beta}
  \alpha = \frac{a_1-a_0\cdot r_2}{r_1-r_2}, \text{ and } \beta =\frac{a_0r_1-a_1}{r_1-r_2}
 \end{equation}
 also satisfies the relation and has the same initial conditions. The equations for $\alpha$ and $\beta$
 come from solving the system of linear equations
 \begin{align*}
  a_0 & =\alpha (r_1)^0 + \beta (r_2)^0 = \alpha + \beta \\
  a_1 & = \alpha (r_1)^1 + \beta (r_2)^1 = \alpha r_1 + \beta r_2.
 \end{align*}
 This system is solved using techniques from a prerequisite course.
\end{proof} 

\begin{exmp}
 Solve the recurrence relation $a_0=2, a_1=3$ and $a_n=a_{n-2}$, for $n\geq 2$.
\end{exmp}
\begin{soln}
 The recurrence relation is a linear homogeneous recurrence relation of degree 2 with
 constant coefficients $c_1=0$ and $c_2=1$. The characteristic polynomial is 
 $$\chi(x)=x^2-0\cdot x-1=x^2-1.$$
 The characteristic polynomial has two distinct roots since
 \[
  x^2-1=(x-1)(x+1).
 \] 
 Let's say $r_1=1$ and $r_2=-1$.
 Then, we find the system of equations:
 \begin{align*}
  2&=a_0=\alpha 1^0 + \beta (-1)^0 = \alpha + \beta \\
  3 &= a_1=\alpha 1^1 +\beta (-1)^1 = \alpha + \beta (-1) = \alpha -\beta.
 \end{align*}
 Adding the two equations eliminates $\beta$ and gives $5=2\alpha$, so $\alpha=5/2$.
 Substituting this into the first equation, $2=5/2 + \beta$, we see that $\beta = -1/2$.
 Thus, our solution is
 \[
  a_n=\frac{5}{2}\cdot 1^n + \frac{-1}{2} (-1)^n = \frac{5}{2} - \frac{1}{2}\cdot (-1)^n.
 \]
\end{soln} 


\begin{exmp}
 Solve the recurrence relation $a_1=3$, $a_2=5$, and, 
 \[
  a_n=5a_{n-1}-6a_{n-2}, \text{ for $n\geq 3$.}
 \]
\end{exmp}
\begin{soln}
 Here the characteristic polynomial is 
 \[
  \chi(x)=x^2-5x+6=(x-2)(x-3),
 \]
 with roots $r_1=2$ and $r_2=3$. Now, we suppose that
\[
 a_m=\alpha 2^m + \beta 3^m, \text{ for all $m\geq 1$.}
\] 
 The initial conditions give rise to the system of equations
 \begin{align*}
 3&=a_1=\alpha 2^1 + \beta 3^1 = 2\alpha + 3\beta \\
 5& =a_2 = \alpha 2^2 +\beta 3^2 = 4\alpha + 9\beta.
 \end{align*}
 If we multiply the top equation through by $2$, we obtain
 \begin{align*}
  6& =4\alpha + 6\beta \\
  5& = 4\alpha + 9\beta.
 \end{align*}
 Subtracting the second equation from the first eliminates $\alpha$ and yields $1=-3\beta$. 
 So, we have found that $\beta= -1/3$.
 Substitution into the first equation yields $3=2\alpha + 3\cdot (-1/3)$, so $\alpha = 2$.
 Thus 
\[
 a_m=2\cdot 2^m - {1\over 3}\cdot 3^m = 2^{m+1} - 3^{m-1}, \text{ for all $m\geq 1$.}
\]
\end{soln} 

\section{The method for repeated roots}
The other case we mentioned had a characteristic polynomial of degree two with one repeated root.
Since the proof is similar we simply state the theorem.

\begin{thm}
 Let $c_1$ and $c_2$ be real numbers with $c_2\neq 0$ and suppose that
 the polynomial $x^2-c_1x-c_2$ has a root $r$ with multiplicity $2$, so that $x^2-c_1x-c_2=(x-r)^2$.
 Then, a sequence $a:\N\to \R$ is a solution of the
 recurrence relation $a_n=c_1a_{n-1}+c_2a_{n-2}$, for $n\geq 2$ if and only if 
\[
  a_m=(\alpha +\beta m) r^m,
\]
 for all $m\in \N$, and for some constants $\alpha$ and $\beta$.
\end{thm}

\begin{exmp}
 Solve the recurrence relation $a_0=-1, a_1=4$ and $a_n=4a_{n-1}-4a_{n-2}$, for $n\geq 2$.
\end{exmp}
\begin{soln}
 In this case we have $\chi(x)=x^2-4x+4=(x-2)^2$. So, we may suppose that 
\[
  a_m=(\alpha + \beta m)2^m, \text{ for all $m\in \N$.}
\]
 The initial conditions give rise to the system of equations
 \begin{align*}
  -1&=a_0=(\alpha +\beta \cdot 0)2^0 = (\alpha )\cdot 1= \alpha \\
  4 & = a_1 = (\alpha + \beta \cdot 1)2^1 = 2(\alpha + \beta)\cdot 2.
 \end{align*}
 Substituting $\alpha =-1$ into the second equation gives $4=2(\beta - 1)$, so $2=\beta - 1$ and $\beta = 3$.
 Therefore $a_m=(3m-1)2^m$, for all $m\in \N$.
\end{soln}

\section{The general case}
Finally, we state\sidenote{without proof} the general method of characteristic roots.
 
\begin{thm}
 Let $c_1, c_2, ..., c_k\in \R$ with $c_k\neq 0$. Suppose that the characteristic polynomial
 factors as
\begin{align*}
 \chi(x) &=x^k-c_1x^{k-1}-c_2x^{k-2}-...-c_{k-1}x-c_k \\
         &=(x-r_1)^{j_1}(x-r_2)^{j_2} \cdots (x-r_s)^{j_s}
\end{align*}
where $r_1$, $r_2$, $\ldots$, $r_s$ are distinct roots of $\chi(x)$, and $j_1$, $j_2$, $\ldots$, $j_s$ are positive integers such that 
\[
  j_1+j_2+j_3+...+j_s=k.
\]
 Then a sequence $a:\N\to\R$ is a solution of the recurrence relation
\begin{align*}
 a_n&=c_1a_{n-1}+c_2a_{n-2}+...+c_ka_{n-k}\quad \text{ for $n\geq k$}  \\
\intertext{if and only if}
 a_m &= p_1(m)r_1^m+p_2(m)r_2^m+...+p_s(m)r_s^m\quad \text{ for all $m\in \N$,} \\
\intertext{where} 
 p_i(m) &= \alpha_{0,i} + \alpha_{1,i}m + \alpha_{2,i} m^2 + ...+\alpha_{j_i-1,i} m^{j_i-1} 
 \quad 1\leq i\leq s
\end{align*}
 and the $\alpha_{l,i}$'s are constants.
\end{thm}

There is a problem with the general case. It is true that given the recurrence relation we 
can simply write down the characteristic polynomial. However it can be quite a challenge
 to factor it as required by the theorem.
Even if we succeed in factoring it we are faced with the tedious task of setting up and solving 
a system of $k$ linear equations in $k$ unknowns (the $\alpha_{l,i}$'s). While in theory such a system can be solved using the methods of
 elimination or  substitution  covered in a college algebra course, in practice, the amount
of labor involved can become overwhelming. For this reason, computer algebra systems are often
used in practice to help solve systems of equations, or even the original recurrence relation.

\clearpage
\section{Exercises}

\begin{exer}
For each of the following sequences find a recurrence relation satisfied by the sequence. Include a sufficient
number of initial conditions to completely specify the sequence.
\begin{enumerate}[label=(\alph*)]
 \item $a_n=2n+3, n\geq 0$
 \item $a_n=3\cdot 2^n, n\geq 1$
 \item $a_n=n^2, n\geq 1$
 \item $a_n=n+(-1)^n, n\geq 0$
\end{enumerate}
\end{exer}

\noindent\textit{Solve each of the following recurrence relations:}
\begin{exer}
$a_0=3, a_1=6,$ and $a_n=a_{n-1}+6a_{n-2},$ for $n\geq 2$.
\end{exer}

\begin{exer}
$a_0=4,a_1=7,$ and $a_n=5a_{n-1}-6a_{n-2},$ for $n\geq 2$.
\end{exer}

\begin{exer}
$a_2=5, a_3=13,$ and $a_n=7a_{n-1}-10a_{n-2},$ for $n\geq 4$.
\end{exer}

\begin{exer}
$a_1=3, a_2=5,$ and $a_n=4a_{n-1}-4a_{n-2},$ for $n\geq 3$.
\end{exer}

\begin{exer}
$a_0=1, a_1=6,$ and $a_n=6a_{n-1}-9a_{n-2},$ for $n\geq 2$.
\end{exer}

\begin{exer}
$a_1=2, a_2=8$, and $a_n=a_{n-2}$, for $n\geq 3$.
\end{exer}

\begin{exer}
$a_0=2, a_1=5, a_2 = 15$, and $a_n=6a_{n-1}-11a_{n-2}+6a_{n-3}$, for $n\geq 3$.
\end{exer}

\begin{exer}
Find a closed form formula for the terms of the Fibonacci sequence: $f_0=0$, $f_1=1$, and
for $n\geq 2$, $f_n=f_{n-1}+f_{n-2}$.
\end{exer}

\section{Problems}

\begin{prob}
Solve $a_0=1$, and $a_n=2a_{n-1},$ for $n\geq 1$ using the characteristic equation method.
\end{prob}

\begin{prob}
Solve $a_0=2, a_1=5$ and $a_n=a_{n-1}+6a_{n-2},$ for $n\geq 2$.
\end{prob}


\begin{prob}
$a_0=3,a_1=7,$ and $a_n=6a_{n-1}-5a_{n-2},$ for $n\geq 2$.

\end{prob}


\begin{prob}
$a_2=5, a_3=13,$ and $a_n=3a_{n-1}+10a_{n-2},$ for $n\geq 4$.
\end{prob}


\begin{prob}
$a_1=3, a_2=5,$ and $a_n=8a_{n-1}-16a_{n-2},$ for $n\geq 3$.
\end{prob}


\begin{prob}
$a_1=2, a_2=8$, and $a_n=4a_{n-2}$, for $n\geq 3$.
\end{prob}


\begin{prob}
$a_0=0, a_1=1, a_2 = 2$, and $a_n=-a_{n-1}+4a_{n-2}+4a_{n-3}$, for $n\geq 3$.
\end{prob}


\begin{prob}
$a_0=0$, $a_1=1$, and
for $n\geq 2$, $a_n=2a_{n-1}+a_{n-2}$.

\end{prob}



