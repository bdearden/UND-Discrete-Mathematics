\chapter{Algorithm Efficiency}

\newthought{There are many different algorithms} for solving any particular class 
of problems. In the last chapter, we considered two algorithms for 
solving the problem of looking up a phone number given a person's name.

\noindent\textbf{Algorithm L:} Look at the first entry in the book. 
If it's the number you're looking for you're
done. Else move to the next entry. 
It's either the number you're looking for or you move to the
next entry, and so on. (The {\bfseries linear search algorithm}) 

\noindent\textbf{Algorithm B:} The second algorithm took advantage of the
arrangement of a phone book in alphabetical order.
We open the phone book to the middle entry. If it's the number we're 
looking for, we're done.
Otherwise we know the number is listed in the first half of the book, 
or the last half of the book. We then pick the middle entry
of the appropriate half, and repeat the process. After a number of
repetitions, we will either be at the name we want, or learn the
name isn't in the book. (The {\bfseries {binary search algorithm}}) 

The question arises, which algorithm is {\it better}?
The question is pretty vague. Let's assume that {\it better}
means {\it uses fewer steps}. Now if there are only one or two
names in the phone book, it doesn't matter which algorithm we use,
the look-up always takes one or two steps. But what if the phone
book contains $10000$ names? In this case, it is hard to say
which algorithm is better: looking up Adam Aaronson will likely 
only take one step by the linear search algorithm, but binary
search will take $14$ steps or so. But for   Zebulon Zyzniewski,
the linear search will take $10000$ steps, while the binary search
will again take about $14$ steps.

\section{Comparing algorithms}
There are two lessons to be learned from those last examples:
\begin{enumerate}
 \item Small cases of the problem can be misleading when
 judging the quality of an algorithm, and 
 \item It's unlikely that
 one algorithm will always be more efficient than another.
\end{enumerate}


The common  approach to compare the efficiency of two algorithms takes
those two lessons into account by agreeing to the following protocol:
\begin{enumerate}
 \item only compare the algorithms
 when the size, $n$,  of the problem it is applied to is huge. 
 In the phone book example, don't worry about phone books of $100$ names
 or even $10000$ names. Worry instead about phone books with $n$ names
 where $n$ gets arbitrarily large.
 
 \item to compare two algorithms, first, for each algorithm,
  find the maximum number of steps
 ever needed when applied to a problem of size $n$. For
 a phone book  of size $n$ the linear search algorithm will require
 $n$ steps in the worst possible case of the name not being in the
 book. On the other hand, the halving process of the binary search
 algorithm means that it will never take more than about $\log_2 n$ steps
 to locate a name (or discover the name is missing) in the phone book.
 This information is expressed compactly by saying the linear search
 algorithm has {\bfseries worst case scenario} efficiency $w_{_L}(n)=n$ while
 the binary search algorithm has worst case scenario efficiency
 $w_{_B}(n) = \log_2 n$.
 
 \item we declare that algorithm \#1 is {\bfseries more efficient} than 
 algorithm \#2 provided, for all problems of huge sizes $n$, 
 $w_1(n)<w_2(n)$, where $w_1$ and $w_2$ are the worst case scenario
 efficiencies for each algorithm.
\end{enumerate}


Notice that for huge $n$, $w_{_B}(n)<w_{_L}(n)$. In fact, there is no real
contest. For example, when $n=1048576 = 2^{20}$, we get
$w_{_B}(n)=20$ while $w_{_L}(n)=1048576$, and things only get better for 
$w_{_B}$
as $n$ gets larger.

In summary, to compare two algorithms designed to solve the same
class of problems we: 
\begin{enumerate}
 \item Determine a number $n$ that indicates the size of the a problem.
 For example, if the algorithm manipulates a list of numbers, $n$
 could be the length of the list.  If the algorithm is designed to
 raise a number to a power, the size could be the power $n$.
 
 \item Decide what will be called a {\itshape step} when applying the
 algorithms. In the phone book example, we took a step to mean a 
 comparison. When raising a number to a power, a step might
 consist of performing a multiplication. A step is usually taken
 to be the most time consuming action in the algorithm, and
 other actions are ignored. Also, when determining the function,
 $w$ don't get hung up worrying about miniscule details. Don't
 spend time trying to determine if $w(n) = 2n+7$ or $w(n) = 2n + 67$.
 For huge values of $n$, the $+7$ and $+67$ become unimportant.
 In such a case, $w(n) =2n$ has all the interesting information.
 Don't sweat the small stuff.
 
 \item Determine the worst case scenario functions for the two algorithms,
 and compare them. The smaller of the two (assuming they are not
 essentially the same) is declared the more efficient algorithm.
\end{enumerate}

\begin{exmp}
Let's do a worst case scenario computation
for the following algorithm designed to determine the largest
number in a list of $n$ numbers.
\begin{algrthm}
  \hrule\kern5pt\relax
  \begin{algorithmic}[1]
     \Require{a list of $n$ numbers $a_1,a_2,\cdots,a_n$}
     \Ensure{$\mathrm{maximum}(a_1,a_2,\dots,a_n)$} 
     \State $max \gets a_1$
     \State $ k \gets 2$
     \While{$k\leq n$}
      \If{$max < a_k$}
       \State $max \gets a_k$
       \State $k \gets k+1$
      \EndIf
     \EndWhile
     \State \textbf{output} $max$
  \end{algorithmic}
  \hrule\kern5pt\relax
  \caption{Maximum list value}\label{alg:max list val}
\end{algrthm}
%\no {\bfseries input}: a list of $n$ numbers $a_1,a_2,\cdots,a_n$
%
%\no step 1: {\bfseries set} $max = a_1$
%
%\no step 2: {\bfseries set} $k = 2$
%
%\no step 3: {\bfseries if} $k>n$, {\bfseries then} {\bfseries output} $max$ and {\bfseries stop}
%
%\no step 4: {\bfseries if} $max<a_k$, then {\bfseries set} $max = a_k$
%
%\no step 5: {\bfseries replace} $k$ by $k+1$  and  {\bfseries go to} step 3
%\ms

It would be natural to use the number of items in the list, $n$, to represent
the size of a problem. And let's use the comparisons as steps. We are
going to make two comparisons each for each of the items in the list
in every case (every case is a worst case for this algorithm!).
So we give this algorithm an efficiency $w(n) =2n$.
Notice that we actually only need comparisons for the last $n-1$ items
in the list, and the exact number of times the comparisons in instructions
(3) and (4) are carried out might take a few minutes to figure out.
But it's clear that both are carried out {\itshape about} $n$ times, and
since we are only interested in huge $n$'s, being off by a few (or a few
billion) isn't really going to matter at all.
\end{exmp}

\clearpage

\section{Exercises}

\begin{exer}\label{exer:18.1}
For the algorithm presented in exercise~\ref{exer:17.2} from the last chapter:
\begin{enumerate}[label=(\alph*)]
 \item Select a value to represent the {\itshape size} of an instance of the 
 problem the algorithm is designed to solve. 
 
 \item Decide what will constitute a {\itshape step} in the algorithm.
 
 \item Determine the worst case scenario function $w(n)$.
\end{enumerate}
\end{exer}

\begin{exer}
 Repeat exercise~\ref{exer:18.1} for the following algorithm:
\begin{algrthm}
  \hrule\kern5pt\relax
  \begin{algorithmic}[1]
    \Require{Sets of reals: $\{x_1,x_2,\dots,x_n\}$ and $\{y_1,y_2,\dots,y_n\}$ of size $n$}
    \Ensure{(to be determined)} 
    \State $S \gets 0$
    \State $i \gets 1$
    \While{$i \leq n$}
      \State $S \gets S+x_i\cdot y_i$
      \State $i \gets i+1$
    \EndWhile
    \State \textbf{output} $S$
  \end{algorithmic}
  \hrule\kern5pt\relax
\end{algrthm}

%\no {\bfseries {Input:}} Two strings of real numbers $x_1,x_2,...,x_n$, and $y_1,y_2,...,y_n$
%
%\no Step 1: {\bfseries {set}} $S=0$
%
%\no Step 2: {\bfseries {set}} $i=1$
%
%\no Step 3: {\bfseries {If}} $i>n$, {\bfseries {output}} $S$ and {\bfseries {stop}}
%
%\no Step 4: {\bfseries {replace}} $S$ by $S+x_i*y_i$
%
%\no Step 5: {\bfseries {replace}} $i$ by $i+1$
%
%\no Step 6: {\bfseries {go to}} Step 3

%\begin{enumerate}[label=(\alph*)]
% \item Select a value to represent the {\itshape size} of an instance of the 
% problem the algorithm is designed to solve. 
% 
% \item Decide what will constitute a {\itshape step} in the algorithm.
% 
% \item Determine the worst case scenario function $w(n)$.
%\end{enumerate}

\end{exer}

\clearpage

\section{Problems}

