\chapter{Rules of Inference}

\newthought{%
The heart of mathematics} is proof. In this chapter, we give a careful
description of what exactly constitutes a proof in the realm of propositional 
logic. Throughout the course 
various methods of proof will be demonstrated, including the
particularly important style of proof called {\itshape induction}. It's important to 
keep in mind
that all proofs, no matter
what the subject matter might be, are based on the notion of a valid argument as 
described
in this chapter, so the ideas presented here are fundamental to all of mathematics.


Imagine
trying carefully to define what a proof is, and it quickly becomes clear just how
difficult a task that is.  So it shouldn't come as a surprise that the
description takes on a somewhat technical looking aspect.  But don't let all
the symbols and abstract{-}looking notation be misleading.  All these rules
really boil down to plain old common sense when looked at
correctly.

The usual form of a theorem in mathematics is: If $a$ is true and $b$ is
true and $c$ is true, etc., then $s$ is true. The $a$, $b$, $c$, $\cdots$ are called
the {\bfseries hypotheses}, and the statement $s$ is called the {\bfseries conclusion}. 
For example,
a mathematical theorem might be: if $m$ is an even integer and $n$ is an
odd integer, then $mn$ is an even integer.  Here the hypotheses are {\itshape $m$ is
an even integer} and {\itshape $n$ is and odd integer}, and the conclusion is 
{\itshape $mn$ is an even integer}.

\section{Valid propositional arguments}
In this section we are going to be concerned with proofs from the
realm of propositional logic rather than the sort of theorem from mathematics 
mentioned above.
We will be interested in arguments in which the {\bfseries form} of the argument is the
item of interest rather than the {\bfseries content} of the statements in the argument.

For example, consider the simple argument: 
(1) {\itshape My car is either red or blue}
and (2) {\itshape My car is not red}, and so 
(3) {\itshape My car is blue}.  Here the hypotheses
are (1) and (2), and the conclusion is (3). It should be clear that this is a 
{\bfseries valid argument}. That
means that if you agree that (1) and (2) are true, then you {\itshape must} accept that 
(3) is 
true as well.

\begin{defn}  An argument is called {\bfseries valid} provided that if you agree 
that all the 
hypotheses are true,
then you must accept the truth of the conclusion.  
\end{defn}

Now the content of that argument 
(in other words,
the stuff about {\itshape my} and {\itshape cars} and {\itshape colors}) really have nothing to 
do with the validity
of the argument. It is the {\bfseries form} of the argument that makes it valid. 
The form of this argument is
(1) $p\lor q$ and (2) $\lnot p$, therefore (3) $q$. Any argument that has this form 
is valid, whether it
talks about cars and colors or any other notions. For example, here is another 
argument of the very same 
form: (1) {\itshape I either read the book or just looked at the pictures} 
and (2) {\itshape I didn't read the book}, 
therefore (3) {\itshape I just looked at the pictures}.


Some arguments involve quantifiers. 
For instance, consider the classic example of a logical argument: 
(1) {\itshape All men are
mortal} and 
(2) {\itshape Socrates is a man}, and so (3) {\itshape Socrates is mortal}.
Here the hypotheses are the statements (1) and (2), 
and the conclusion is statement (3).
If we let $M(x)$ be {\itshape $x$ is a man} and $D(x)$ be {\itshape $x$ is mortal} (with
domain for $x$ being everything!), then this argument could be symbolized as shown.

\begin{margintable}
\begin{tabular}{l}
$\forall{x}(M(x)\to D(x))$ \\
$M$(Socrates)\\
\hline
\thus $D$(Socrates)
\end{tabular}
\end{margintable}


The general form of a proof that a logical argument is valid consists in assuming all 
the hypotheses
have truth value $T$, and showing, by applying valid rules of logic, 
that the conclusion
must also have truth value $T$.

\clearpage
Just what are the valid rules of logic that can be used in the course
of the proof? They are called the Rules of Inference, and there are  seven
of them listed in the table below. Each rule of inference arises
from a tautology, and actually there is no end to the rules of inference,
since each new tautology can be used to provide a new rule of inference.
But, in real life, people rely on only a few basic rules of
inference, and the list provided in the table is plenty
for all normal purposes.

\begin{table}
\centering
\begin{tabular}{ l  l l }
\toprule
\textbf{Name} & \span{\bfseries Rule of Inference} \\
\midrule
Modus Ponens & $p$ and $p\rightarrow q$ & $\thus q$ \\
\addlinespace
Modus Tollens & $\neg q$ and  $p\rightarrow q$ & $\thus \neg p$ \\
\addlinespace
Hypothetical Syllogism\hspace*{0.25cm} & $p\rightarrow q$ and $q\rightarrow r$ & $\thus p\rightarrow r$ \\
\addlinespace
Addition & $p$  & $\thus p\vee q$ \\
\addlinespace
Simplification & $p\wedge q$ & $\thus p$ \\
\addlinespace
Conjunction & $p$ and $q$ &  $\thus p\wedge q$\\
\addlinespace
Disjunctive Syllogism & $p\vee q$ and $\neg p$ &  $\thus q$ \\
\bottomrule
\end{tabular}
\caption{Basic rules of inference}
\label{tbl:rules infer}
\end{table}

It is important not to merely look on these rules as marks on the
page, but rather to understand what each one says in words. For example,
Modus Ponens corresponds to the common sense rule: if we are told {\itshape $p$ is true},
and also {\itshape If $p$ is true, then so is $q$}, then 
we would leap to the reasonable conclusion that {\itshape $q$ is true}. That is all
Modus Ponens says. Similarly, for the rule of proof of Disjunctive Syllogism:
knowing 
 {\itshape Either $p$ or $q$ is true}, and {\itshape $p$ is not true}, 
we would immediately
conclude {\itshape $q$ is true}. 
That's the rule we applied in the {\itshape car} example above. 
Translate the remaining six rules of inference into
such common 
sense statements. Some may sound a little awkward, but they ought to all
elicit an {\itshape of course that's right} feeling once understood. Without 
such an understanding, the rules seem like
a jumble of mystical symbols,  and building logical arguments will be pretty
difficult. 

What exactly goes into a logical argument?  Suppose we want to prove
(or show valid) an argument of the form 
{\itshape If $a$ and $b$ and $c$ are true, then
so 
is $s$}. One way that will always do the trick is to construct a truth table
as in examples earlier in the course.  We check the rows in the table where
all the hypotheses are true, and make sure the conclusion is also true in those rows. 
That would complete the proof. In fact that is exactly the method used to 
justify the seven
rules of inference given in the table. But building truth tables is
certainly tedious business, and it certainly doesn't seem too much like the
way we learned to do proofs in geometry, for example. An alternative is the
construction of a logical argument which begins by assuming the hypotheses
are all true and  applies the basic rules of inferences from the table  until the
desired conclusion is shown to be true.

Here is an example of such a proof. 
Let's show that the argument displayed in figure~\ref{fig:a logic arg}
is valid.

\begin{marginfigure}
  \begin{tabular}{l}
  $p$\\
  $p\to q$\\
  $s\lor r$\\
  $r\to\neg{q}$\\
  \midrule
  \thus $s\lor t$
  \end{tabular}
  \caption{A logical argument}\label{fig:a logic arg}
\end{marginfigure}

 Each step in the argument will be  justified
in some way, either
\begin{enumerate*}
 \item as a hypothesis (and hence assumed to have truth value $T$), or
 \item as a consequence of previous steps and some rule of inference from the table, or
 \item as a statement logically equivalent to a previous statement in the proof.  
\end{enumerate*} 
 Finally  the last statement
in the proof will be the desired conclusion. Of course, we could  prove the
argument valid by constructing 
a $32$ row truth table instead! Well, actually we wouldn't need all $32$ rows, but
it would be pretty tedious in any case.

Such 
proofs can be viewed as games in which the hypotheses serve as the starting
position in a game, the goal is to reach the conclusion as the final position
in the game, and the rules
of inference (and logical equivalences) specify the legal moves. Following this
outline, we can be sure every step in the proof is a true statement, and, in 
particular, the desired conclusion is true, as we hoped to show.
\begin{table}
 \textbf{Argument:}
   \begin{tabular}[t]{l}
   $p$\\
   $p\to q$\\
   $s\lor r$\\
   $r\to\neg{q}$\\
   \midrule
   \thus $s\lor t$
   \end{tabular}
 \quad\textbf{Proof:}
 \begin{tabular}[t]{r l l}
 (1)& $p$                   & hypothesis\\
 (2)& $p\rightarrow q$      & hypothesis\\
 (3)& $q$                   & Modus Ponens (1) and (2)\\
 (4)& $r\rightarrow \neg q$ & hypothesis\\
 (5)& $q\rightarrow \neg r$ & logical equivalent of (4)\\
 (6)& $\neg r$              & Modus Ponens (3) and (5)\\
 (7)& $s\lor r$             & hypothesis\\
 (8)& $r\lor s$             & logical equivalence of (7)\\
 (9)& $s$                   & Disjunctive Syllogism (6) and (8)\\
 (10)& $s\lor t$            & Addition
 \end{tabular}
 \caption{Proof of the validity of an argument}\label{tbl:proof of arg}
\end{table}

One step more complicated than the last example are arguments that are presented
in words rather than symbols. In such a case, it is necessary to first convert
from a verbal argument to a symbolic argument, and then check the argument to see
if it is valid. For example, consider the argument: {\itshape Tom is a cat. If Tom
is a cat, then Tom likes fish. Either Tweety is a bird or Fido is a dog.
If Fido is a dog, then Tom does not like fish. So, either Tweety is a bird or
I'm a monkey's uncle.} Just reading this argument, it is difficult to decide
if it is valid or not. It's just a little too confusing to process. But it is valid,
and in fact it is the very same argument as given above. Let
$p$ be {\itshape Tom is a cat}, let $q$ be {\itshape Tom likes fish}, let $s$ be
{\itshape Tweety is a bird}, let $r$ be  {\itshape Fido is a dog}, and let
$t$ be {\itshape I'm a monkey's uncle}. Expressing the statements in the 
argument in terms of $p,q,r,s,t$ produces exactly the symbolic argument proved above.   

\section{Fallacies}
Some logical arguments have a convincing ring to them but are nevertheless
invalid. The classic example is an argument of the form {\itshape If it is snowing,
then it is winter. It is winter. So it must be snowing.} A moment's 
thought
is all that is needed to be convinced  the conclusion does not follow from the two
hypotheses. Indeed, there are many winter days when it does not snow. The error
being made is called the {\bfseries fallacy of affirming the conclusion}. In 
symbols,
the argument is claiming that $[(p\to{q}) \land{q}]\to p$ is a tautology, but in
fact, checking a truth table shows that it is not a tautology. Fallacies arise
when statements that are not tautologies are treated as if they were
tautologies.


\section{Arguments with quantifiers}
Logical arguments involving propositions using quantifiers require a few
more rules of inference. As before, these rules really amount to no more than a
formal way to express common sense. For instance, if the proposition
$\forall\,x\,P(x)$ is true, then certainly for every object $c$ in the
universe of discourse, $P(c)$ is true. After all, if the statement $P(x)$ is true
for every possible choice of $x$, then, in particular, it is true when $x=c$. The
other three rules of inference for quantified statements are just as obvious.  
All four quantification rules appear in table~\ref{tbl:quant rules}.


\begin{table}
\resizebox{\textwidth}{!}{%
 \begin{tabular}{ll}
 \toprule
 \textbf{Name} & \textbf{Instantiation Rules}  \\
 \midrule
 Universal Instantiation &$\forall x P(x)$  $\thus P(c)$ if $c$ is in the domain of $x$ \\
 \addlinespace
 Existential Instantiation &$\exists x P(x)$  $\thus P(c)$ for some $c$ in the domain of $x$ \\
 \bottomrule
 \addlinespace
 \toprule
 \textbf{Name} & \textbf{Generalization Rules}  \\
 \midrule
 Universal Generalization 
 & $P(c)$ for \textbf{arbitrary} $c$ in the domain of $x$  $\thus \forall x\,P(x)$ \\
 Existential Generalization 
 &  $P(c)$ for some $c$ in the domain of $x$ \qquad $\thus \exists x\,P(x)$  \\
 \bottomrule
 \end{tabular}
 } %end resizebox
 \caption{Quantification rules}\label{tbl:quant rules}
\end{table}


\begin{exmp}
 Let's analyze the following (fictitious, but obviously valid) argument 
to see how these rules of inference are used. {\itshape All books written by
Sartre are hard to understand. Sartre wrote a book about kites. So, there
is a book about kites that is hard to understand.} Let's use to following
predicates to symbolize the argument:
\begin{enumerate}
\item $S(x) :  x $ was written by Sartre.
\item $H(x) : x $ is hard to understand.
\item $K(x) : x $ is about kites.
\end{enumerate}
The domain for $x$ in each case is {\itshape all books}.
In symbolic form, the argument and a proof are

%\vfill
%\eject


\textbf{Argument:}
\begin{tabular}[t]{l}
$\forall{x}(S(x)\rightarrow H(x))$ \\
$\exists{x}(S(x) \land K(x))$ \\
\hline
$\thus \exists{x}(K(x)\land H(x))$
\end{tabular}

\medskip

\textbf{Proof:}
\begin{tabular}[t]{l l}
1) $\exists{x}(S(x) \land K(x))$& hypothesis \\
2) $S(c)\land K(c) \hbox{ for some } c $& Existential Instantiation (1) \\ 
3) $S(c)$& Simplification (2) \\ 
4) $\forall{x}(S(x) \to H(x))$& hypothesis \\ 
5) $S(c)\to H(c)$& Universal Instantiation (4) \\ 
6) $H(c)$& Modus Ponens (3) and (5) \\ 
7) $K(c) \land S(c)$& logical equivalence (2) \\ 
8) $K(c)$& Simplification (7) \\ 
9) $K(c)\land H(c)$& Conjunction (8) and (6) \\ 
10) $\exists{x}(K(x)\land H(x))$& Existential Generalization (9)
\end{tabular}
\end{exmp}

\clearpage




\section{Exercises}

\begin{exer}
  Show {\itshape $p\lor q$ and $\neg p \lor r, \quad \thus q\lor r$} is a valid rule
of inference.  It is called {\bfseries Resolution}.
\end{exer}

\begin{exer}
Prove the following argument is valid. {\itshape All Porsche owners are 
speeders. No owners of sedans buy premium fuel.  Car owners that do not buy
premium fuel never speed. So Porsche owners do not own sedans.}
Use {\itshape all car owners} as the domain of discourse.
\end{exer}

\begin{exer}
Prove the following symbolic argument is valid.
\begin{center}
  \begin{tabular}{l}
  $\lnot p\to (r \land \lnot s)$  \\ 
  $ t\to s$   \\ 
  $u\to \lnot p$   \\ 
  $\lnot w$ \\
  $u \lor w$   \\ 
  \hline
  $\thus  \lnot t \lor w$
  \end{tabular}
\end{center}
\end{exer}



\clearpage
\section{Problems}


\begin{prob}
  Show that {\itshape $p\to q$ and $\neg p,\quad \thus \neg q$} is not a valid rule
of inference. It is called the \textbf{Fallacy of denying the hypothesis}. 

\end{prob}

\begin{prob}
Prove the following symbolic argument is valid.
\begin{center}
  \begin{tabular}{l}
  $\lnot p\land  q$  \\ 
  $ r\to p$   \\ 
  $\neg r\to s$   \\ 
  $s\to t$   \\ 
  \hline
  $\thus  t$
  \end{tabular}
\end{center}
\end{prob}

\begin{prob}
Prove the following symbolic argument is valid.
\begin{center}
  \begin{tabular}{l}
  $p\lor  q$  \\ 
  $ q\to r$   \\ 
  $(p \land s)\to t$   \\ 
  $\lnot r$   \\ 
  $\lnot q \to (p \land s)$   \\ 
  \hline
  $\thus  t$
  \end{tabular}
\end{center}
\end{prob}

\begin{prob}
Prove the following symbolic argument is valid.
\begin{center}
  \begin{tabular}{l}
  $(\lnot p\lor  q) \to r$  \\   
  $ s \lor \lnot q$   \\ 
  $\lnot t$\\
  $p \to t$   \\ 
  $(\lnot p \land r) \to \lnot s$   \\ 
  \hline
  $\thus  \lnot q$
  \end{tabular}
\end{center}
\end{prob}

\begin{prob}\label{pr:valid Ralph}
Express the following argument is symbolic form and prove the argument is valid. 
{\itshape If Ralph doesn't do his homework 
or he doesn't feel sick, then he will go to the party and  he will stay up late. If he goes to the
party, he will eat too much.  He didn't eat too much. So Ralph did his homework.}
\end{prob}

\begin{prob}
In problem \ref{pr:valid Ralph}, show that you can logically deduce that Ralph felt sick.
\end{prob}

\begin{prob}
In prob \ref{pr:valid Ralph}, can you logically deduce that Ralph stayed up late?
\end{prob}

\begin{prob}
Prove the following symbolic argument.

\begin{tabular}[t]{l}
$\exists{x}(A(x)\land\neg B(x))$ \\
$\forall{x}(A(x) \to C(x))$ \\
\hline
$\thus \exists{x}(C(x)\land \neg B(x))$
\end{tabular}

\end{prob}



