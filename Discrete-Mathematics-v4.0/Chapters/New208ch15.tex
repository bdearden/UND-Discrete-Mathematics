\chapter{Recursively Defined Sets}


\newthought{Two different ways} of defining a set have been discussed. We can describe a set
by the roster method, listing all the elements that are to be members of the set, or 
we can describe a set using set-builder notation by  giving a predicate that the elements
of the set are to satisfy. Here we consider defining sets in another natural way: recursion.

\section{Recursive definitions of sets}
 Recursive definitions can also be used to build sets of objects. The spirit
is the same as for recursively defined sequences: give some initial conditions and
a rule for building new objects from ones already known.

\begin{exmp}\label{exmp:recur set Evens}
  For instance, here is a
 way to recursively define the set of positive even integers, $E$. First the
 initial condition: $2\in E$. Next the recursive portion of the definition: If
 $x\in E$, then $x+2\in E$. Here is what we can deduce using these two rules. First
 of course, we see $2\in E$ since that is the given initial condition. Next, since
 we know $2\in E$, the recursive portion of the definition, with $x$ being played
 by $2$,  says $2+2\in E$, so that now we know $4\in E$. Since $4\in E$, the
 recursive portion of the definition, with $x$ now being played 
 by $4$,  says $4+2\in E$, so that now we know $6\in E$. Continuing in this way, it
 gets easy to believe that $E$ really is the set of positive even integers.
\end{exmp}

\clearpage
 Actually, there is a little more to do with example~\ref{exmp:recur set Evens}. The claim is that
$E$ consists of exactly all the positive even integers. In other words, we also
need to make sure that no other things appear in $E$ besides the positive even
integers. Could $312211$ somehow have slithered into the set $E$? To verify that
such a thing does not happen, we need one more fact about recursively defined
sets. The only elements that appear in a set defined recursively are those that
make it on the basis of either the initial condition or the recursive portion of
the definition. No elements of the set appear, as if by magic, from nowhere.

In this case, it is easy to see that no odd integers sneak into the set. For if so,
there would be a smallest odd integer in the set and the only way it could be elected to
the set is if the integer two less than it were in the set. But that would mean a 
yet smaller odd integer would be in the set, a contradiction. We won't go into that 
sort of  detail
for the following examples in general. We'll just consider the topic at the
intuitive level only. 


\begin{exmp} Give a recursive definition of the set, $S$,  of all nonnegative integer
powers of $2$.

Initial condition: $1\in S$.  Recursive rule: If $x\in S$, then $2x\in S$. Applying the initial
condition and then the recursive rule repeatedly gives the elements:
$$
1\qquad 2\cdot1=2\qquad 2\cdot 2 = 4\qquad 2\cdot 4 = 8\qquad 2\cdot 8 = 16
$$
and so on, and that looks like the set of nonnegative powers of $2$.
\end{exmp}

\begin{exmp} A set, $S$, is defined recursively by 

(1) (initial conditions) $1\in S$
and $2\in S$, and 

(2) (recursive rule) If $x\in S$, then $x+3\in S$. Describe the 
integers in $S$.

The plan is to use the initial conditions and the recursive rule
to build elements of $S$ until we can guess a description of the
integers in $S$.
From the initial conditions we know $1\in S$ and $2\in S$. Applying the
recursive rule to each of those we get $4,5\in S$, and using the 
recursive rule on those gives $7, 8\in S$, and so on.

So we get $S=\{1,2,4,5,7,8,10,11,\cdots\}$ and it's apparent that
$S$ consists of of the positive integers that are not multiples of $3$.
\end{exmp}

\section{Sets of strings}
Recursively defined sets appear in certain computer science courses where 
they are used to
describe sets of strings. To form a string, we begin with an 
{\bf alphabet} which
is a set of symbols, traditionally denoted by $\Sigma$.
 For example $\Sigma = \{ a,b,c\}$ is an alphabet of three symbols,
and $\Sigma= \{!, @, \#, \$, \%, \&, X, 5\}$ is an alphabet of eight symbols. 
A {\bf string} over the alphabet $\Sigma$ is any finite sequence of symbols from the alphabet.
For example $aaba$ is a string of {\bf length} four over the alphabet 
$\Sigma = \{ a,b,c\}$, and $!!5X\$\$5@@$ is a length nine string over
$\Sigma= \{!, @, \#, \$, \%, \&, X, 5\}$. There is a special string over any 
alphabet denoted by $\lambda$ called the {\bf empty string}. It
contains no symbols, and has length $0$.

\begin{exmp} A set, $S$,  of strings over the alphabet 
$\Sigma = \{ a,b\}$
is given recursively by (1) $\lambda\in S$, and (2) If $x\in S$, then
$axb\in S$. Describe the strings in $S$.

The notation $axb$ means write down the string $a$ followed by the
string $x$ followed by the string $b$. So if $x= aaba$ then $axb=
aaabab$. Let's experiment with the recursive rule a bit, and then guess
a description for the strings in $S$. Starting with the initial condition
we see $\lambda\in S$. Applying the recursive rule to $\lambda$ gives
$a\lambda b = ab\in S$. Applying the recursive rule to $ab$ gives
$aabb\in S$, and applying the recursive rule to $aabb$ shows 
$aaabbb\in S$. It's easy to guess the nature of the strings in $S$:
Any finite string of $a$'s followed by the same number of $b$'s.
\end{exmp}
 
\begin{exmp} Give a recursive definition of the set  $S$ of strings 
over $\Sigma = \{ a,b,c\}$ which do not contain adjacent $a$'s. For example
$ccabbbabba$ is acceptable, but $abcbaabaca$ is not.

For the initial conditions we will use (1) $\lambda\in S$, and $a\in S$.
If we have a string with no adjacent $a$'s, we can extend it by adding
$b$ or $c$ to either end. But we'll  need to be careful when adding 
more $a$'s. For the recursive rule we will use
(2) if $x\in S$, then $bx,xb,cx,xc\in S$ and $abx,xba,acx,xca\in S$.

Notice how the string $a$ had to be put into $S$ in the initial conditions
since the recursive rule won't allow us to form that string from 
$\lambda$. 
\end{exmp}

Here is different answer to the same question. It's a little harder to dream up, 
but the rules are much cleaner. The idea is that if we take two strings
with no adjacent $a$'s, we can put them together and be sure to get a new string with
no adjacent $a$'s  provided we stick either $b$ or $c$ between them.
So, we can define the set recursively by (1) $\lambda\in S$ and $a\in S$, 
and (2) if $x,y\in S$, then $xby, xcy\in S$.

\begin{exmp} 
Give a recursive definition of the set  $S$ of strings 
over $\Sigma = \{ a,b\}$ which contain more $a$'s than $b$'s.

The idea is that we can build longer strings from smaller ones by
(1) sticking two such strings together, or (2) sticking two such
strings together along with a $b$ before the first one, between the
two strings, or after the last one. That leads to the following
recursive definition:
(1) $a\in S$ and (2) if $x,y\in S$ then $xy, bxy, xby,xyb\in S$.
That looks a little weird since in the recursive rule we added
$b$, but since $x$ and $y$ each have more $a$'s than $b$'s, the
two together will have a least two more $a$'s than $b$'s, so it's
safe to add $b$ in the recursive rule. 

Starting with the initial condition, and then applying the recursive rule
repeatedly, we form the following elements of $S$:
$$
a, aa, baa, aba, aab, aaa, baaa, abaa, aaba, baaa,\cdots
$$
\end{exmp}


\begin{exmp}
A set, $S$,  of strings over the alphabet 
$\Sigma = \{ a,b\}$ is defined recursively by the rules
(1) $a\in S$, and (2) if $x\in S$, then $xbx\in S$. Describe
the strings in $S$.

Experimenting we find the following elements of $S$:
$$
a, aba, abababa, abababababababa, \cdots
$$

It looks like $S$ is the set of strings beginning with $a$ followed by
a certain  number of $ba$'s. If we look at the number of $ba$'s
in each string, we can see a pattern: $0,1,3,7,15,31,\cdots$,
which we recognize as being the numbers that are one less than
the positive integer powers of $2$ ($1,2,4,8,16,32,\cdots)$.
So it appears $S$ is the set of strings which consisting of $a$ followed
by $2^n-1$ pairs $ab$ for some integer $n\geq 0$.
\end{exmp}


\clearpage
\section{Exercises}

\begin{exer}
The set $S$ is described recursively by (1) $1\in S$, and (2) if $n\in S$, then $n+1\in S$.\\
To what familiar set is $S$ equal? 
\end{exer}

\begin{exer}
 Give a recursive definition of the set of positive integers
that end with the digits $17$.
\end{exer}

\begin{exer} 
Give a recursive definition of the set of positive integers
that are not multiples of $4$.
\end{exer}

\begin{exer} 
 Describe the strings in the  set $S$ of strings over the alphabet 
$\Sigma = \{a,b,c\}$
defined recursively by (1) $\lambda \in S$ and (2) if $x\in S$, then 
$axbc\in S$. 
\end{exer}


\begin{exer} 
  Describe the strings in the set $S$ of strings over the alphabet 
$\Sigma = \{a,b,c\}$ defined recursively by (1) $c \in S$
and (2) if $x\in S$ then $ax\in S$ and $bx\in S$ and $xc\in S$.
\end{exer}


\begin{exer} 
 A {\bf palindrome} is a string that reads the same in both
directions.%
\marginnote{A classic palindrome:\\ {\itshape A man, a plan, a canal: panama.} }
For example, $aabaa$ is a palindrome of length five and
$babccbab$ is a palindrome of length eight.  The empty string
is also a palindrome. Give a recursive
definition of the set of palindromes over the alphabet\\
$\Sigma =\{a,b,c\}$.   
\end{exer}

\clearpage
\section{Problems}

\begin{prob}
 A set $S$ of integers is defined recursively by the rules: \\[2pt]
 (1) $1\in S$, and (2) If $n\in S$, then $2n+1 \in S$. \\[2pt]
 
 \begin{enumerate} 
 \item Is $15\in S$?\\[2pt]
 \item Is $65\in S$?\\[2pt]
 \end{enumerate}
 Explain your answers.
 \end{prob}
 
 \begin{prob}
 A set of integers is defined recursively by the rules (1) $0 \in S$, and (2)
 if $n \in S$, then $2n+2 \in S$. Give a simple description of the integers in $S$.
 \end{prob}
 
 \begin{prob}
 Give a recursive definition of the set 
 \[
 \{ 3^n -3\,|\, n \text{ a positive integer}\} = \{0, 6, 24, 78, 240, 726, 2184, \ldots\}.
 \]
 \end{prob} 
 
 \begin{prob}
  A set, $S$, of strings over the alphabet 
$\Sigma = \{a,b,c\}$ is defined recursively by (1) $a \in S$
and (2) if $x\in S$ then $bxc\in S$. List all the strings in $S$ of length seven or less.
\end{prob}

\begin{prob}
 A set, $S$, of positive integers is defined recursively by the rule:\\[2pt]
(1) $1\in S$, and (2) If $n\in S$, then $2n-1\in S$. List all the elements in the set $S$.
\end{prob}
 
 \begin{prob}
  Give a recursive definition of the set of positive integers that end with the digit $1$.
\end{prob}

\begin{prob}
Give a recursive definition of the set of strings over the alphabet $\Sigma = \{a,b,c\}$ of the 
form $aaa\cdots abccc\cdots c$. More carefully: zero or more $a$'s followed by a single $b$ 
followed by the same number of $c$'s as $a$'s.
\end{prob} 

\begin{prob}
Describe the strings in the set $S$ of strings over the alphabet 
$\Sigma = \{a,b,c\}$ defined recursively by (1) $a \in S$
and (2) if $x\in S$ then $ax\in S$ and $xb\in S$ and $xc\in S$.\\[3pt]

Hint: Your description should be a sentence that provides an easy test to check if a given string
is in the set or not. An example of such a description is: {\it $S$ consists of all strings of $a$'s, $b$'s, and $c$'s,
with more $a$'s than $b$}'s. That isn't a correct description since $abb$ is in $S$ and doesn't have more $a$'s than
$b$'s, and also $baac$ isn't in $S$, but does have more $a$'s than $b$'s. So that attempted description is really 
terrible. One way to do this problem is to use the rules to build a bunch of strings in $S$ until a suitable
description becomes obvious.  Alternatively, just thinking about the recursive rules might be sufficient for you to see a simple description of the strings in $S$.
\end{prob}

\begin{prob}
A set $S$ of ordered pairs of integers is defined recursively
by (1) $(1,1)\in S$, and (2) if $(m,n)\in S$, then $(m+2,n)\in S$, and
$(m,n+2)\in S$, and $(m+1,n+1)\in S$. Give a simple description
of the ordered pairs in $S$.
\end{prob}






