\chapter{Permutations and Combinations}

\newthought{By a {\bfseries permutation} of a set of objects} we mean a listing of the objects of the set in
a specific order. For example, there are six possible permutations of the set
$A = \{a,b,c\}$. They are
\[
abc,\quad acb,\quad  bac,\quad bca,\quad cab,\quad cba.
\]

The product rule explains why there are six permutation of $A$: there are
$3$ choices for the first letter, once that choice has been made there are $2$
choices for the second letter, and finally that leaves $1$ choice for the last
letter. So the total number of permutations is $3\cdot2\cdot1 = 6$.

\section{Permutations}
A set with $n$ elements is called an {\bfseries $n$-set}. We have just shown that
a $3$-set has $6$ permutations. The same reasoning shows that an $n$-set
has $n\cdot(n-1)\cdot(n-2)\cdots2\cdot1 = n!$ permutations. So the total
number of different ways to arrange a deck of cards is $52!$,
a number with $68$ digits:
{\color{blue}$80658175170943878571660636856403766975289505440883277824000000000000$}.

Instead of forming a permutation of all the elements of an $n$-set, we might
consider the problem of first selecting some of the elements of the set, say
$r$ of them, and then forming a permutation of just those $r$ elements.
In that case we say we have formed an {\bfseries $r$-permutation} of
the $n$-set. All the possible $2$-permutations of the $4$-set
$A=\{a,b,c,d\}$ are
\[ab,\quad ac,\quad ad,\quad ba,\quad bc,\quad bd,\quad ca,\quad
cb,\quad cd,\quad da, \quad db, \quad dc
\]
Hence, there are twelve $2$-permutations of a $4$-set. 

\begin{notation}
 In general, $P(n,r)$ denotes the number of different
 \marginnote{The number of $2$-permutations of a $4$-set is
 $\displaystyle P(4,2)$=12.}
 $r$-permutations of an $n$-set.
\end{notation}

The product rule provides a simple formula for $P(n,r)$.
There are $n$ choices for the first element, and once that choice has been
made, there are $n-1$ choices for the second element, then $n-2$ for the
third, and so on, until finally, there are $n-(r-1) = n-r+1$ choices for the
$r^{th}$ element. So $\dl{P(n,r)=n(n-1)\cdot ...\cdot (n-r+1)}$.
That expression can be written more neatly as follows:
\marginnote{And, that is the way to remember the formula:{\color{blue} \[P(n,r)=  \frac{n!}{(n-r)!}.\]}}
\begin{align*}
 P(n,r)&=n(n-1)\cdot ...\cdot (n-r+1) \\
 &= {\frac{n(n-1)\cdot ...\cdot (n-r+1)(n-r)(n-r-1)\cdots2\cdot1}
 {(n-r)(n-r-1)\cdots2\cdot1}} = {\frac{n!}{(n-r)!}}
\end{align*}

\begin{exmp}\label{exmp:pres v.pres etc.}
As an example, the number of ways of selecting a president, vice-president,
secretary, and treasurer from  a group of $20$ people is $P(20,4)$ (assuming
no person can hold more than one office). If you want the actual numerical
value, it is $\displaystyle {\frac{20!}{(20-4)!}} = 20\cdot19\cdot18\cdot17
= 116280$, but the best way to write the answer in most cases would be
just $\displaystyle P(20,4)=\frac{20!}{16!}$, and skip the numerical computations.
\end{exmp}


\begin{exmp}
 How many one-to-one functions are there from a
 $5$-set to a $7$-set?
 
 While this question doesn't sound on the surface like a problem of permutations, 
 it really is. Suppose the $5$-set is $A=\{1,2,3,4,5\}$ and the $7$-set is
 $B=\{a,b,c,d,e,f,g\}$. One example of a one-to-one function from  $A$ to $B$ 
 would be $f(1) =a, f(2) = c, f(3)=g, f(4)=b, f(5) = d$. But, if we agree to
 think of the elements of $A$ listed in their natural order, we could specify
 that function more briefly as $acgbd$. In other words, each one-to-one function
 specifies a $5$-permutation of $B$, and, conversely, each $5$-permutation
 of $B$ specifies a one-to-one function.\marginnote{The same reasoning shows there are
  $P(n,r)$ one-to-one functions from an $r$-set to an $n$-set.}
   So the number of one-to-one functions
 from a $5$-set to a $7$-set is equal to the number of $5$-permutations of a
 $7$-set, and that is $P(7,5)=2520$. 
\end{exmp}

\begin{exmp}
 Here are a few easily seen values of $P(n,r)$:
 \begin{enumerate}
   \item $P(n,n) = n!$
   
   \item $P(n,1) = n$
   
   \item $P(n,0) = 1$\marginnote{There is only one $0$-permutation, the one with no symbols!}
   
   \item $P(n,r) = 0$ if $r>n$\marginnote{There are no permutations with length
   greater than $n$ of $n$ objects.}
 \end{enumerate}
\end{exmp}

\section{Combinations}
When forming permutations, the order in which the elements are listed is
important. But there are many cases when we are interested only in 
which elements are selected and we do not care about the order. For example,
when playing poker, a hand consists of five cards dealt from a standard
$52$-card deck. The order  in which the cards arrive in a hand does not
matter, only the final selection of the five cards is important. When order
is not important, the selection is called a combination rather than a permutation.
More carefully, an {\bfseries $r$-combination} from an $n$-set is an 
$r$-subset of the $n$-set. In other words an $r$-combination of an $n$-set is 
an unordered selection of  $r$ distinct elements from the $n$-set. 


\begin{exmp}\label{exmp:C(5,2)}
 The $2$-combinations of the $5$-set $\{a,b,c,d,e\}$ are
 \begin{align*}
   \{a,b\},\quad  &\{a,c\},\quad \{a,d\},\quad \{a,e\},\quad \{b,c\}, \\
   \{b,d\},\quad  &\{b,e\},\quad \{c,d\},\quad \{c,e\},\quad \{d,e\}.
 \end{align*}

\end{exmp}

\begin{notation}
 The number of $r$-combinations from an $n$-set is denoted
 \marginnote{Example~\ref{exmp:C(5,2)} shows that \newline
 $\displaystyle C(5,2) = {5\choose 2}=10$.}
  by $C(n,r)$ or, sometimes, $\displaystyle{{n\choose r}}$.
\end{notation}

\begin{exmp}
 Here are a few easily seen values of $C(n,r)$:
 \begin{enumerate}
  \item $C(n,n) = 1$
   
  \item $C(n,1) = n$
   
  \item $C(n,0) = 1$\marginnote{There is only one $0$-subset, the empty set.}
   
  \item $C(n,r) = 0$ if $r>n$\marginnote{There are no subsets of an $n$-set with size
   greater than $n$.}
 \end{enumerate}
 
\end{exmp}


There is a compact formula for $C(n,r)$ which can be derived using the
product rule in a sort of back-handed way. An $r$-permutation of an
$n$-set can be built using a sequence of two tasks. First, select
$r$ elements of the $n$-set. There are $C(n,r)$ ways to do that task. Next, 
arrange those $r$ elements in some specific order. There are $r!$ ways
to do that task. So, according to the product rule, the number
of $r$-permutations of an $n$-set will be $C(n,r)r!$. However, we
know that the number of $r$-permutations of an $n$-set are $P(n,r)$.
So we may conclude that $P(n,r) = C(n,r)r!$, or, rearranging that, we see
\[
C(n,r) = {n\choose r}= {{P(n,r)}\over{r!}} = {{n!}\over{{r!(n-r)!}}}
\]

\begin{exmp}
Suppose we have a club with $20$ members. 
If we want to select a committee
of $4$ members, then there are 
\[C(20,4)= {{20!}\over{4!(20-4)!}}
= {{20\cdot19\cdot18\cdot17}\over{4\cdot3\cdot2\cdot1}}= 4845
\]
ways to do this since the order 
of people on the committee
doesn't matter.  
\end{exmp}

Compare this answer with example~\ref{exmp:pres v.pres etc.}  where we counted the number
of possible selections\sidenote{$P(5,4)=20!/16!=116280$ ways} for president, vice-president, secretary, and treasurer
from the group of $20$. 
The difference between the two
cases is that the earlier example is a question about permutations 
(order matters), whereas this example is
a question about combinations (order does not matter). 


\clearpage
\section{Exercises}

\begin{exer}
In how many ways can the $26$ volumes (labeled $A$ through $Z$) 
of the Encyclopedia of PseudoScience be placed on a shelf?
\end{exer}

\begin{exer}
In how many ways can those same $26$ volumes be placed on a shelf
 if superstitions demand the volumes labeled with vowels must be adjacent?
 In how many ways can they be placed on the shelf obeying the conflicting superstition
 that volumes labeled with vowels cannot touch each other?
\end{exer}

\begin{exer}
For those same $26$ volumes, how many ways can they be placed
in a two shelf bookcase if volumes $A$-$M$ go on the top shelf and $N$-$Z$
go on the bottom shelf?
\end{exer}

\begin{exer}
In how many ways can  seven men and four women sit in a row if the
men must sit together?
\end{exer}

\begin{exer}
$20$ players are to be divided into two $10$-man teams. In how many
ways can that be done?
\end{exer}

\begin{exer}
A lottery ticket consists of five different integers selected from $1$ to $99$.
How many different lottery tickets are possible? How many tickets would you need
to buy to have a one-in-a-million chance of winning by matching all five randomly
selected numbers?
\end{exer}

\begin{exer}
A committee of size six is selected from a group of nine clowns and
thirteen  lion tamers. 
\begin{enumerate}[label=(\alph*)]
 \item How many different committees are possible?
 
 \item How many committees are possible if there must be exactly two clowns
 on the committee?
 
 \item How many committees are possible if lion tamers must outnumber clowns
 on the committee?
\end{enumerate}
\end{exer}

\section{Problems}

\begin{prob}
In how many ways can the ten digits be written in a row?
\end{prob}


\begin{prob}
In how many ways can the ten digits be written in a row if the odd digits have to be adjacent?
\end{prob}


\begin{prob}
In how many ways can the ten digits be written in a row if the even and odd digits have to alternate?
\end{prob}


\begin{prob}
How many bit strings of length ten have exactly four $0$'s?
\end{prob}


\begin{prob}
How many bit strings of length ten have at most four $0$'s?
\end{prob}


\begin{prob}
How many length twenty strings of $a$'s, $b$'s, and $c$'s have ten $a$'s, six $b$'s, and four $c$'s? 
\end{prob}


\begin{prob}
How many bit strings of length ten have more $0$'s than $1$'s?
\end{prob}


\begin{prob}
In how many ways can a subset of two numbers from $1$ to $100$ (inclusive)  be selected if the selected numbers cannot be consecutive? 
\end{prob}


