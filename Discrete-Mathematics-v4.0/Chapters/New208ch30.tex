\chapter{The Binomial Theorem and Pascal's Triangle}

\newthought{The quantity $C(n,k)$ is also written} as $\displaystyle \binom{n}{k}$, and called a {\it binomial 
coefficient} . It gives the number of $k$-subsets of an $n$-set, or, equivalently, it gives
the number of ways of selecting $k$ items from $n$ items. 

\section{Combinatorial proof}
Facts involving the binomial coefficients can  be proved algebraically, using the
formula $\displaystyle \binom{n}{k} = {\frac{n!}{k!(n-k)!}}$. But often the same facts can be
proved much more neatly by recognizing   $\displaystyle \binom{n}{k}$ gives the number
of $k$-subsets of an $n$-set. This second sort of proof is called a combinatorial
proof. Here is an example of each type of proof.

\begin{thm}[Pascal's Identity]\label{thrm:Pascal's Id}
 Let $n$ and $k$ be non-negative integers, then
\[
\binom{n+1}{k}=\binom{n}{k-1}+\binom{n}{k}.
\]
\end{thm}
\begin{proof}(an algebraic proof) 
\begin{align*}
  {n\choose k-1}+{n\choose k}&= {\frac{n!}{(k-1)!(n-(k-1))!}}+ {\frac{n!}{k!(n-k)!}} \\
 &=  {\frac{n!}{(k-1)!(n-k+1)!}}+ {\frac{n!}{k!(n-k)!}} \\
 &= {\frac{n!}{(k-1)!(n-k)!}}\left(\frac{1}{n-k+1}+ \frac{1}{k}\right) \\
 &=  {\frac{n!}{(k-1)!(n-k)!}}\left(\frac{k}{k(n-k+1)}+ {\frac{n-k+1}{k(n-k+1)}}\right) \\
 &=  {\frac{n!}{(k-1)!(n-k)!}}{\frac{k+(n-k+1)}{k(n-k+1)}} \\
 &= {\frac{n!}{(k-1)!(n-k)!}}{\frac{n+1}{k(n-k+1)}} \\
 &= {\frac{n!(n+1)}{(k-1)!k(n-k)!(n-k+1)}} \\
 &= {\frac{(n+1)!}{k!(n+1-k)!}} \\
 &= {{n+1}\choose k}
\end{align*}
\end{proof}

\begin{proof} (a combinatorial proof) Let $S$ be a set with $n+1$ elements. 
Select one particular element $a\in S$. There are two ways to produce a subset of $S$
of size $k$. We can include $a$ in the subset, and toss in $k-1$ of the remaining
$n$ elements of $S$. There are $\displaystyle {n\choose{k-1}}$ to do that. Or, we can avoid $a$,
and choose all $k$ elements from the other $n$ elements of $S$. There are $\displaystyle{n\choose k}$
ways to do that. So, according to the sum rule, there is a total of
$\displaystyle {n\choose{k-1}}+{n\choose k}$ subsets of size $k$ of $S$. But we know there are
 $\displaystyle{{n+1\choose k}}$ subsets of size $k$ of $S$. So it must be that
 $\displaystyle {n+1\choose k}={n\choose k-1}+{n\choose k}$.
 \end{proof}

\subsection{Constructing combinatorial proofs}
The idea of a combinatorial proof is to ask a counting problem that  can be  answered in
two different ways, and then conclude the two answers must be equal. In the proof above,
we asked how many $k$-subsets there are of an $(n+1)$-set. We provided an argument to
show two answers were correct: $\displaystyle {{n+1}\choose k}$ and $\displaystyle {n\choose{k-1}}+
{n\choose k}$, and so we could conclude the two answers must be equal:
\[
{n+1\choose k}={n\choose k-1}+{n\choose k}.
\]

As in the combinatorial proof of Pascal's Identity~\ref{thrm:Pascal's Id}, 
such arguments can be much less work, far less tedious,
and much more illuminating, than algebraic proofs. Unfortunately, they can also be
much more difficult to discover since it is necessary to dream up a good counting problem that
will have as answers the two expressions we are trying to show are equal, and there
is no algorithm for coming up with such a suitable counting problem.

\begin{exmp}
 Give a combinatorial%
 \sidenote{An algebraic proof of this identity is absolutely trivial:
 \begin{align*}
     {n\choose k}  &= {\frac{n!}{k!(n-k)!}} \\
    \intertext{and} 
  {n\choose {n-k}} &= {\frac{n!}{(n-k)!(n-(n-k))!}} \\
                   &= {\frac{n!}{k!(n-k)}}.
 \end{align*}
 Thus, we see  that the two expressions are indeed equal.}
  proof of   $\displaystyle {n\choose k} = {n\choose {n-k}}$.
\begin{soln}
To provide a combinatorial proof, we ask {\it how many ways are there to
grab $k$ elements of an $n$-set?}. One answer of course is $\displaystyle {n\choose k}$. But here
is a second way to view the problem. We can select $k$ elements of an $n$-set by deciding
on $n-k$ elements {\bfseries not} to pick. Since there are $\displaystyle {n\choose{n-k}}$ ways to select
the $n-k$ not to pick, there must be $\displaystyle {n\choose{n-k}}$ ways to select $k$ elements
of an $n$-set. Since the two answers must be equal we conclude that
$\displaystyle {n\choose k} = {n\choose {n-k}}$.\;\qed
\end{soln}
\end{exmp}

\begin{exmp}
Give a combinatorial proof of {\bfseries Vandermonde's Identity}: 
\[
{{n+m}\choose k} = {n\choose 0}{m\choose k}+{n\choose 1}{m\choose {k-1}}+
{n\choose 2}{m\choose {k-2}}+\cdots+{n\choose k}{m\choose 0}.
\]
\begin{soln}
 Consider the set $\{a_1,a_2,\cdots,a_n,b_1,b_2,\cdots b_m\}$ of $n+m$ elements.
 We ask, {\itshape how may $k$-subsets does the set have?}.
 
  One answer of course  is  $\displaystyle{{n+m}\choose k}$. 
  
  But here's another way to answer the question: 
 
 we can select $0$ of the $a$'s and $k$ of the $b$'s. 
 There are $ {n\choose 0}{m\choose k}$ ways  to do that. 
 
 Or we select $1$ of the $a$'s and $k-1$ of the $b$'s. 
 There are $ {n\choose 1}{m\choose k-1}$ ways to do that. 
 
 Or we select $2$ of the $a$'s and $k-2$ of the $b$'s. 
 There are $ {n\choose 2}{m\choose k-2}$ ways to do that.
 
 And so on, until we reach the option of  selecting $k$ of the $a$'s and $0$ of the $b$'s. 
 There are $ {n\choose k}{m\choose 0}$ ways to do that.
 
 By the sum rule, it follows that another way to count the number of $k$-subsets is
 $$ {n\choose 0}{m\choose k}+{n\choose 1}{m\choose {k-1}}+
 {n\choose 2}{m\choose {k-2}}+\cdots +{n\choose k}{m\choose 0}
 $$
 and that proves Vandermonde's Identity.\;\qed
\end{soln}
\end{exmp}
 
\section{Pascal's Triangle}
The binomial coefficients are so named because they appear when a binomial $x+y$ is raised
to an integer power $\geq 0$. 
To appreciate the connection, let's look at a table of the binomial coefficients $\displaystyle {n\choose k}$.
The table is arranged in rows starting with row $n=0$, and within each row, the entries
are arranged from left to right for $k=0,1,2,\cdots,n$.
The result, called Pascal's Triangle,  is shown in figure~\ref{fig:Pascal's Triangle}.
\begin{marginfigure}
\centering
\resizebox{\textwidth}{!}{ %
%\includegraphics{./Graphics/PascalsTriangle-0.png}
\def\N{5}
\tikz[x=0.75cm,y=0.5cm, 
  pascal node/.style={font=\footnotesize}, 
  row node/.style={font=\footnotesize, anchor=west, shift=(180:1)}]
  \path  
    \foreach \n in {0,...,\N} { 
       (-\N/2-1, -\n) node  [row node/.try]{Row \n:}
        \foreach \k in {0,...,\n}{
          (-\n/2+\k,-\n) node [pascal node/.try] {%
%            \pgfkeys{/pgf/fpu}%
%            \pgfmathparse{round(\n!/(\k!*(\n-\k)!))}%
%            \pgfmathfloattoint{\pgfmathresult}%
%            \pgfmathresult%
             $\binom{\n}{\k}$  
        }}} 
(-2.5,-5.75) node [row node/.try] {$\iddots$}
(0.75,-5.75) node [row node/.try] {$\vdots$}
(3.75,-5.75) node [row node/.try] {$\ddots$};
}\caption{Pascal's Triangle}\label{fig:Pascal's Triangle}
\end{marginfigure}

Filling in  the numerical values for the binomial coefficients gives the table shown in
figure~\ref{fig:Pascal's Triangle (numeric)}.
\begin{marginfigure}
\centering
\resizebox{\textwidth}{!}{ %
%\includegraphics{./Graphics/PascalsTriangle_numeric-0.png}
\def\N{5}
\tikz[x=0.75cm,y=0.5cm, 
  pascal node/.style={font=\footnotesize}, 
  row node/.style={font=\footnotesize, anchor=west, shift=(180:1)}]
  \path  
    \foreach \n in {0,...,\N} { 
       (-\N/2-1, -\n) node  [row node/.try]{Row \n:}
        \foreach \k in {0,...,\n}{
          (-\n/2+\k,-\n) node [pascal node/.try] {%
            \pgfkeys{/pgf/fpu}%
            \pgfmathparse{round(\n!/(\k!*(\n-\k)!))}%
            \pgfmathfloattoint{\pgfmathresult}%
            \pgfmathresult%
 %            $\binom{\n}{\k}$  
        }}} 
(-2.5,-5.75) node [row node/.try] {$\iddots$}
(0.75,-5.75) node [row node/.try] {$\vdots$}
(3.75,-5.75) node [row node/.try] {$\ddots$};
}\caption{Pascal's Triangle (numeric)}\label{fig:Pascal's Triangle (numeric)}
\end{marginfigure}


Note that we number the rows starting with $0$.
We already know quite a bit about the entries in Pascal's Triangle. Since $\displaystyle {n\choose 0} 
={n\choose n} = 1$ for all $n$, each row begins and ends with a $1$. From the symmetry
formula $\displaystyle {n\choose k} = {n\choose {n-k}}$, the rows read the same in  both directions.
By Pascal's Identity, each entry in a row is the sum of the two entries diagonally above it.
That last fact makes it easy to add new rows to Pascal's Triangle.

The numbers in the first, second, and third rows of Pascal's triangle probably seem familiar.
 In fact, we see that %
 \marginnote{%
 The coefficients in these binomial expansions are exactly the entries in the corresponding
 rows of Pascal's Triangle. This even works for the $0^{th}$ row: $(x+y)^0 = 1$.%
 }%
\begin{align*}
 (x+y)^0 = &{\color{blue}1}\\
 (x+y)^1 = x+y = {\color{blue}1}\cdot x &+{\color{blue}1}\cdot y\\
 (x+y)^2 = x^2+2xy +y^2 = {\color{blue}1}\cdot x^2 + &{\color{blue}2}\cdot xy + {\color{blue}1}\cdot y^2\\
 (x+y)^3 = x^3 + 3x^2y + 3xy^2 + y^3 = 
  {\color{blue}1}\cdot x^3 + {\color{blue}3}\cdot x^2y &+ {\color{blue}3}\cdot xy^2 + {\color{blue}1}\cdot y^3\\
\end{align*}


\section{The Binomial Theorem}
The fact that the coefficients in the expansion of the binomial $(x+y)^n$ (where $n\geq 0$ is 
an integer) can be read off from the $n^{th}$ row of Pascal's Triangle is called the
Binomial Theorem. We will give two proofs of this theorem, one by induction, and the
other a combinatorial proof. 

\begin{thm}[The Binomial Theorem] When $n$ is a non-negative integer and $x,y\in \R$
\[
(x+y)^n=\sum_{k=0}^n {n\choose k} x^k y^{n-k}.
\]
\end{thm}
\begin{proof}{\itshape (by induction on $n$)} When $n=0$ the result is clear. So suppose that for some
$n\geq 0$ we have $\displaystyle{(x+y)^n=\sum_{k=0}^n {n\choose k}x^k y^{n-k}}$, for any $x,y\in \R$. Then, we have 
\begin{align*}
 (x+y)&^{n+1} =(x+y)^n(x+y), \\
 & = \Big[\sum_{k=0}^n {n\choose k} x^k y^{n-k} \Big](x+y) \quad {\text{~by~inductive~hypothesis}}, \\
 & = \Big[\sum_{k=0}^n {n\choose k} x^{k+1} y^{n-k}\Big]+\Big[\sum_{k=0}^n {n\choose k} x^k y^{n+1-k}\Big] \\
 & = {n\choose n} x^{n+1} + \Big[\sum_{k=0}^{n-1} {n\choose k} x^{k+1} y^{n-k}\Big] + \Big[\sum_{k=1}^n {n\choose k} x^k y^{n+1-k}\Big]+
 {n\choose 0}y^{n+1} \\
 & = {n\choose n} x^{n+1} + \Big[\sum_{l=1}^{n} {n\choose l-1} x^l y^{n-(l-1)}\Big] + \Big[\sum_{k=1}^n {n\choose k} x^k y^{n+1-k}\Big]+
 {n\choose 0}y^{n+1} \\
 & = {n\choose n} x^{n+1} + \Big[\sum_{l=1}^{n} {n\choose l-1} x^l y^{n+1-l}\Big] + \Big[\sum_{k=1}^n {n\choose k} x^k y^{n+1-k}\Big]+
 {n\choose 0}y^{n+1} \\
 & = {n\choose n} x^{n+1} + \Big[\sum_{k=1}^{n} \Big[{n\choose k-1} + {n\choose k}\Big] x^k y^{n+1-k}\Big]+ {n\choose 0}y^{n+1} \\
 & = {n+1\choose n+1} x^{n+1} + \Big[\sum_{k=1}^{n} {n+1\choose k} x^k y^{n+1-k}\Big]+ {n+1\choose 0}y^{n+1} \quad
 {\text{ by~Pascal's~identity}}, \\
 &= \sum_{k=0}^{n+1} {n+1\choose k} x^k y^{n+1-k}.
 \end{align*}
\end{proof}

\clearpage
Now, let's look at a combinatorial proof of the Binomial Theorem. 
\begin{proof}
 When the binomial
 $
 (x+y)^n = (x+y)(x+y)(x+y)\cdots(x+y)
 $
 is expanded, the terms are produced by selecting either the $x$ or the $y$ from each
 of the $n$ factors $x+y$ appearing on the right side of the equation. The number of 
 ways of selecting exactly $k$ $x$'s from the $n$ available is $\displaystyle {n\choose k}$, and
 so that will be the coefficient of the term $x^ky^{n-k}$ in the expansion.\marginnote{Isn't this amazing!}
\end{proof}

\begin{exmp}
The coefficient of $x^7y^3$ in the expansion of $(x+y)^{10}$
is 
\[
 \binom{10}{7} = {\frac{10!}{7!3!}}= {\frac{10\cdot9\cdot8}{1\cdot2\cdot3}}
= 120.
\]
\end{exmp}

\begin{exmp}The coefficient of  $x^7y^3$ in the expansion of $(2x-3y)^{10}$ is 
\[
 {{10}\choose 7}2^7(-3)^3 = {\frac{10!}{7!3!}}2^7(-3)^3= -{\frac{10\cdot9\cdot8}{1\cdot2\cdot3}}
 2^73^3 
 = -120\cdot128\cdot27 = -414720.
\]
\end{exmp}

From the binomial theorem we can derive facts such as 

\begin{thm}
A finite set with $n$ elements has $2^n$ subsets.
\end{thm}
\begin{proof}
By the sum rule the number of subsets of an $n$-set is
\[
\sum_{k=0}^n {n\choose k}=\sum_{k=0}^n {n\choose k} 1^k 1^{n-k}.
\]
By the Binomial Theorem $\displaystyle{\sum_{k=0}^n {n\choose k} 1^k 1^{n-k} = (1+1)^n=2^n}.$
\end{proof}

\begin{thm}If $n\geq 1$, then 
\[
{n\choose 0}-{n\choose 1}
+{n\choose 2}-\cdots +(-1)^n {n\choose n} = 0.
\]
\end{thm}
\begin{proof}By the Binomial Theorem, 
\[{n\choose 0}-{n\choose 1}
+{n\choose 2}-\cdots +(-1)^n {n\choose n} = (1-1)^n = 0.
\]
\end{proof}

\clearpage
\section{Exercises}

\begin{exer}
Determine the sixth row of Pascal's Triangle.
\end{exer}

\begin{exer}
Determine the coefficient of  $x^3y^7$ in the expansion\\ of $(3x-2y)^{10}$.
\end{exer}

\begin{exer}
Give an algebraic proof that $\displaystyle {{2n}\choose 2} = 2{n\choose 2} + n^2$.
\end{exer}

\begin{exer}
Give an algebraic proof that $\displaystyle \binom{r}{s}\binom{s}{t} = \binom{r}{t}\binom{r-t}{s-t}$.
\end{exer}

\begin{exer}
Give a combinatorial proof that $\displaystyle \binom{r}{s}\binom{s}{t} = \binom{r}{t}\binom{r-t}{s-t}$.
\end{exer}

\begin{exer}
Using the same reasoning as in the combinatorial proof of the Binomial Theorem,
determine the coefficient of $x^4y^5z^6$ in the expansion of $(x+y+z)^{15}$.
\end{exer}

\begin{exer}
Show that if $p$ is a prime and $0< k<p$, then $p$ divides $\displaystyle {p\choose k}$.
Hint: When $\displaystyle {p\choose k}$ is written out, how many times does $p$ occur as a factor
of the numerator and how many times as a factor of the denominator?
\end{exer}

\section{Problems}

\begin{prob}
Determine the seventh row of Pascal's Triangle.
\end{prob}

\begin{prob}
Determine the coefficient of  $x^5y^2$ in the expansion\\ of $(2x-5y)^{7}$.
\end{prob}

\begin{prob}
Give a combinatorial  proof that $\displaystyle {{2n}\choose 2} = 2{n\choose 2} + n^2$.\\
Hint: How many ways are there to select a 2-subset of $\{a_1,a_2,\cdots,a_n,b_1,b_2,\cdots b_n\}$?
\end{prob}

\begin{prob}
Give an algebraic proof that $\displaystyle \binom{n}{2}\binom{n-2}{m-2} = \binom{n}{m}\binom{m}{2}$.
\end{prob}

\begin{prob}
Give a combinatorial proof that $\displaystyle \binom{n}{2}\binom{n-2}{m-2} = \binom{n}{m}\binom{m}{2}$.
\end{prob}

\begin{prob}
Give an algebraic proof that $\displaystyle m\binom{n}{m} = n\binom{n-1}{m-1}$.
\end{prob}

\begin{prob}
Give a combinatorial proof that $\displaystyle m\binom{n}{m} = n\binom{n-1}{m-1}$.
\end{prob}


\begin{prob}
Give a combinatorial proof that $\displaystyle{{3m\choose 3}={m\choose 3}+2m{m\choose 2}+m{2m\choose 2} +{2m\choose 3}}$.
\end{prob}


\begin{prob}
Determine the coefficient of $x^3y^2z^3$ in the expansion\\ of $(x+2y-3z)^8$.
\end{prob}