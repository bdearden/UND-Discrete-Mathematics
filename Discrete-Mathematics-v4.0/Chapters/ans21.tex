    \section*{Chapter 21}

\begin{Solution}{21.1}
$107653 = (4)(22869) + 16177$. So the quotient is $4$, and the remainder  is $16177$.
\end{Solution}

\begin{Solution}{21.2}

The square root of $1297$ is $36.01...$, so if $1297$ is not a prime, it must have a prime divisor no more than $36$. Testing $2, 3, 5, 7, 11, 13, 17, 19, 23, 29,$ and $31$, (use a calculator if you want), we find none of those ten integers divides $1297$. That means $1297$ is a prime.

\end{Solution}

\begin{Solution}{21.3}
The divides relation is reflexive: For every integer $a$, $a|a$ is true since $a\cdot 1 = a$.
\end{Solution}

\begin{Solution}{21.4}
The divides relation is not symmetric. For example, $2|4$ is true, but $4|2$ is false.
\end{Solution}

\begin{Solution}{21.5}
The divides relation is transitive. Suppose $a|b$ and $b|c$ are both true. That means there
are integers $d$ and $e$ such that $ad = b$ and $be = c$. Multiplying each side of the first 
of those two equations by $e$ gives $ade = be$, so $a(de)= c$. Since $de$ is an integer, that
equation shows $a|c$ is true. $\clubsuit$
\end{Solution}

\begin{Solution}{21.6}
The expression $4|12$ is a proposition, not a number. Correct is $4|12$ is true. 
Note that $\frac{12}{4}=3$ is correct notation.
\end{Solution}

\begin{Solution}{21.7}
The $1000$ consecutive numbers are

\[
1001! + 2,\,1001!+3,\, 1001!+4,\, 1001!+5,\,\;dots, 1001!+k,\,\ldots, 1001!+1001
\]

The first number in the list is not a prime since $2$ is a factor of each term, and so
$2$ is a factor of $1001!+2$. Likewise, $3$ is a proper factor of the second number, $4$ is
a proper factor on the third number, and so on, until $1001$ is a proper factor of the 
$1000^{th}$ number in the list. So none of the integers can be a prime. $clubsuit$
\end{Solution}

\begin{Solution}{21.8}
\underbar{Proof}:\\
Suppose $a|b$. That means $ac = b$ for some integer $c$.
Then $(-a)(-c) = ac = b$, so $-a|b$. $\clubsuit$
\end{Solution}

