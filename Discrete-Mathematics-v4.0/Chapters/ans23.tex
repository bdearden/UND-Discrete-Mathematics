   \section*{Chapter 23}
    
\begin{Solution}{23.1}
Part 1 (back-substitution method):

\begin{align*}
 13447 &= 1\cdot7667+5780 \\
 7667 &= 1\cdot5780+ 1887\\
  5780 &= 3\cdot1887+119 \\
  1887 &= 15\cdot119+102 \\
  119  &= 1\cdot102+17 \\
  102  &= 6\cdot17+0
\end{align*}

$\gcd(13447,7667) = 17$.

\begin{align*}
17 &= (1)(119)+ (-1)(102)\\[3pt]
     &= (1)(119) + (-1)(1887 +(-15)(119)) = (-1)(1887) + (16)(119)\\[3pt]
     &= (-1)(1887) + (16)(5780 +(-3)(1887)) = (16)(5780) + (-49)(1887)\\[3pt]
     &= (16)(5780) +(-49)(7667 + (-1)(5780 ) = (-49)(7667) + (65)(5780)\\[3pt]
     &= (-49)(7667) +(65)(13447 + (-1)(7667)) = (65)(13447) + (-114)(7667)
\end{align*}

$(65)(13447) + (-114)(7667) = 17 = \gcd(13447,7667)$

\vskip 10pt

Part 2 (Extended Euclidean Algorithm Table):

\begin{table}
\renewcommand{\arraystretch}{1.25}
\begin{tabular}{|*{8}{>{\raggedleft\arraybackslash}p{0.996cm}|}}
 \hline
 13447&7667&5780&1887&119&102&17&0 \\
  \hline
  &&1&1&3&15&1&6 \\
 \hline
 0&1&-1&2&-7&107&-114&791 \\
 \hline
 1&0&1&-1&4&-61&65&-451 \\
 \hline
\end{tabular}
\end{table}

Thus, we may conclude that 
\[
\gcd(13447,7667) = 17 = (65)(13447) + (-114)(7667).
\]

You'll likely agree that the Extended Euclidean Algorithm table is a lot neater and much less prone to error.



\end{Solution}


\begin{Solution}{23.2}
Since the only values that can be written as a linear combination of $a$ and $b$ are multiples of $\gcd(a,b)$, it must be that $1$ is a multiple of $\gcd(a,b)$. Or, saying the same thing  another way, $gcd(a,b)$ must be a positive divisor of $1$. The only choice is $\gcd(a,b) = 1$.

\end{Solution}

\begin{Solution}{23.3}
Since the only values that can be written as a linear combination o $a$ and $b$ are multiples of $\gcd(a,b)$, it must be that $19$ is a multiple of $\gcd(a,b)$. Or, saying the same thing  another way, $gcd(a,b)$ must be a positive divisor of $19$. So, $\gcd(a,b) = 1$ or $19$.


\end{Solution}

\begin{Solution}{23.4}
Since the only values that can be written as a linear combination o $a$ and $b$ are multiples of $\gcd(a,b)$, it must be that $18$ is a multiple of $\gcd(a,b)$. Or, saying the same thing  another way, $gcd(a,b)$ must be a positive divisor of $18$. So, $\gcd(a,b) = 1, 2, 3, 6, 9,$ or $18$.
\end{Solution}


