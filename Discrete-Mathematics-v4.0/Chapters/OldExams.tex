\begin{enumerate}

\item Let $M(x,y)$ be the propositional function {\it x has met y}.
The domains for $x$ and $y$ are all people.
Express the proposition {\it Everyone who has met Al has also
met Bill} in symbolic form.
\medskip 

\item Give a counterexample to the proposition {\it The square of an
even integer cannot end with the digit $6$}.
\medskip 
\item  Write out a direct proof of the proposition: {\it If $m$ is an even integer and $n$ is an
odd integer, then $m+n$ is an odd integer.}

\medskip 
\item Let $C(x)$ be the propositional function {\it $x$ owns a cat}, and let
$S(x)$ be the propositional function {\it $x$ has scratches }. The domain for $x$ is all
people.
\begin{enumerate}
\item Express $\forall{x}(C(x)\rightarrow S(x))$ in a smooth English sentence.

\item Express the negation of the proposition in part (a) as a smooth English sentence.
\end{enumerate}

\medskip 
\item {\bf True \ \  False}: $\forall{x}\exists{y} P(x,y) \equiv \forall{y}\exists{x}P(y,x)$ 
\medskip 
\item Let $A=\{\, a,b\,\}$ and $B=\{\,a,c\,\}$. List the elements in $A \times B$.
\medskip 
\item 
\begin{enumerate}
\item Draw the graph of the relation $D(x,y): x \text{ \it divides } y$, where
the domain and codomain for $D$ are both the set $\{2,3,4,5,6\}$.

\item For the relation $D$, state whether or not $D$ has the 
following properties: reflexive, symmetric, antisymmetric, and transitivity.
\end{enumerate}
 \medskip 
 
\item $R$ is the relation on the integers given by $R(n,m)$: $n$ and $m$ are within $2$ units of each other.
 For example,
$R(12,13)$ is true since $12$ and $13$ are only $1$ unit apart. But $55$ and $60$ are $5$ units apart,
so $R(55,60)$. Circle all the  properties the 
relation $R$ has in the list below.

\vskip 5pt
\hskip 20pt (a) Reflexive\hfill
\vskip 5pt
\hskip 20pt (b) Symmetric\hfill
\vskip 5pt
\hskip 20pt (c) Antisymmetric\hfill
\vskip 5pt
\hskip 20pt (d) Transitive\hfill
\vskip 5pt
\hskip 20pt (e) Self Referential\hfill
\medskip 
\item Let $K(x,y)$ be the propositional function \underbar{$x$ knows $y$}, where the universe of discourse is all people. Express the proposition \underbar{Everyone except Joe knows Ralph} in symbolic  form.
\medskip 
\item Let $E$ be an equivalence relation of a set $A$. For elements $u,v,w\in A$ prove that if $u\in [v]$ and
$wEu$, then $w\in [v]$. (Recall that $[v]$ is the equivalence class of $v$.)
\medskip 
\item   Use a truth table to decide if $p\rightarrow(\lnot p\land q)$ is logically equivalent
to $p\rightarrow \lnot q$.




\item  A set $S$ of strings is defined recursively over the alphabet $\Sigma = \{\,a,b,c\,\}$
by the rules (1) $\lambda\in S$, and (2) if $x\in S$, then $abxc\in S$. There aren't very
many elements of $S$ of length $9$ or less. List them all.

\medskip

\item Describe in words the strings in the set $S$ of the problem above.

\medskip

\item A sequence is defined recursively by (1) $w_0 = 2$ and $w_1 = 1$, and,
(2) for $n\geq 2$,  $w_n = w_{n-1} + w_{n-2}$. What is the value of $w_5$?

\medskip

\item Compute $\displaystyle \sum_{k=0}^{50} \left(\frac{1}{3}\right)^k$.

\medskip

\item \underbar{Use induction} to prove: For every integer $n\geq 1$,

\[
\frac{1}{1\cdot2}+\frac{1}{2\cdot3}+\frac{1}{3\cdot4}+\cdots+\frac{1}{n(n+1)}= \frac{n}{n+1}.
\]


\item {\bf T}rue or {\bf F}alse:  If $a$ and $b$ both divide $c$, then $ab$ divides $c$.

\medskip
\item \underbar{Use the Euclidean algorithm} to determine 
the greatest 
common divisor of $715$ and $297$. 

\medskip

\item Write the greatest common divisor of $715$ and $297$ as a linear combination 
of those two numbers.

 

\item {\bf True \ \  False}: If $p$ is a prime, and $p\,|\,ab$, where $a$ and $b$ are integers, 
then either $p\,|\,a$ or $p\,|\,b$.

\medskip

\item The number of positive integers that divide $320$ is
\begin{enumerate}

\item $7$\\[4pt]

\item $14$\\[4pt]

\item $32$\\[4pt]

\item $160$\\[4pt]

\item $320$\\

\end{enumerate}

\medskip

\item {\bf True \ \  False}: $-123\equiv 11 \,(mod\,31)$.

\medskip

\item  Find all solutions to the congruence equation $12x\equiv 2 \,(mod\,5)$.

\medskip

\item Determine the number of {\it full house} poker hands ($5$ cards, order not important, selected
from a $52$ card deck, with $3$ cards of one rank, and $2$ cards of a second rank.)

\medskip

\item When $(2x-3y)^{25}$ is expanded, what is the coefficient of the term $x^{10}y^{15}$?

\medskip

\item How many string of length six of the $26$ letters either begin with $xx$ or end with $ooo$?

\medskip

\item {\bf True \ \  False}: For any finite sets $A$ and $B$, $|A\cup B| = |A| + |B|$.

\medskip

\item Find a closed form formula for $a_0 = 1$, $a_1 = 1$, and for $n\geq2$, $a_n = 3a_{n-1} + 10a_{n-2}$.

\medskip

\item {\bf True \ \  False}: When solving a nonhomogeneous recursion with recursive relation
$a_n = 2a_{n-1} + n^2 + 1$, a good choice for the form of a particular solution is $a_n^{(p)} =  An^2 + 1$.

\medskip
 
 \item {\bf True \ \  False}: Every graph with a Eulerian path will have an Eulerian circuit.
 
 \medskip

\item {\bf True \ \  False}: Every tree with $10$ vertices must have $9$ edges. 

\medskip

\item {\bf True \ \  False}: The wheel, $W_{10}$, has a Hamiltonian circuit.

\end{enumerate}

\pagebreak

\begin{center}Hints and Solutions to Sample Question\end{center}
\begin{enumerate}

\medskip

\item $\forall{x}(M(x,Al)\longrightarrow M(x,Bill))$

\medskip

\item $4$ is a counterexample: $4$ is even and $4^2 = 16$ ends with a $6$.

\medskip

\item Suppose $m$ is even and $n$ is odd. By the definitions of even and odd, there are integers $j$, $k$ so that $m= 2j$ and $n = 2k+1$.  Then $m+n$ = 2j+(2k+1) = 2(j+k)+1. Since $m+n$ is $1$ more than
two times an integer, $m+n$ is odd by the definition of odd.

\medskip


\item 

\begin{enumerate}
\item Every cat owner has scratches.

\item There is a cat owner without scratches. 
\end{enumerate}

\medskip


\item True. This may be clearer if you read the propositions as: For every choice of the first variable, there is a choice of the second variable that makes $P$ a true proposition.

\medskip


\item The ordered pairs in $A\times B$ are $(a,a)$, $(a,c)$, $(b,a)$, and $(b,c)$.

\medskip


\item 
\begin{enumerate}
\item The directed graph would have arrows from $2$ to $4$, from $2$ to $6$, from $3$ to $6$, and
in addition, there is a loop at every vertex.

\item There is a loop at every vertex, so $D$ is reflexive. Since there is an arrow from $2$ to $4$, but not one from $4$ to $2$, the relation is not symmetric. The relation is antisymmetric since there are no two way
arrows. Finally, $D$ transitive by default since there are no cases where we have and arrow from 
$a$ to $b$ and an arrow from $b$ to $c$, so we never have to check for an arrow from $a$ to $c$.
(Well, except for cases where  $a=b$ or $b=c$, but in these cases, the $a$ to $c$ arrow is one of the
two already listed, so the arrow from $a$ to $c$ is in the digraph for sure.)
\end{enumerate}
\medskip


\item Let's assume that by {\it within $2$ units of each other} means {\it $2$ or less units apart}. So $R(10,12)$ 
is true.  $R$ is reflexive is every integer is within $2$ units of itself. It is symmetric since if $m$ is within $2$ unit of 
$n$, then $n$ is within $2$ unit of $m$, It is not antisymmetric since, for example, $R(10,11)$ and 
$R(11,10)$ are both true, but $10\not=11$. Finally,  $R$ is not transitive since, for example, $R(12,14)$ and 
$R(14,16)$ are both true, but $R(12,16)$ is false.

\medskip


\item  The English is ambiguous. If the intention is that we know everyone besides Joe knows Ralph, but we don't have any information about whether Joe knows Ralph or not, then a correct answer is $\forall{x}( (x\not= Joe) \longrightarrow K(x,Ralph))$. On the other hand, if the intention is that we are also really sure Joe does not know Ralph, then a correct answer is
$\forall{x}( (x\not= Joe) \longrightarrow K(x,Ralph)) \land \lnot K(Joe,Ralph)$.

\medskip


\item  Let's give a direct proof. So suppose  $E$ is an equivalence relation of a set $A$, and  that $u\in [v]$ and $wEu$. Since $u\in [v]$, we know $uEv$ is true. We are told $wEu$ is true as well. Since $E$ is transitive, from $wEu$ and $uEv$, we can conclude $wEv$. That tells us $w\in [v]$, as we needed to show.

\medskip


\item \noindent {\underbar {Example}} $\neg(p\wedge q)\equiv (\neg p \vee \neg q)$.
$$\vbox{\offinterlineskip
\halign { \strut # & # & \vrule ~~# & \vrule ~~# & \vrule ~~# & \vrule ~~# \cr
$p$ & $q$ & $\neg p\land q$ & $p\longrightarrow(\neg p\land q)$  & $\neg q$ & $p \longrightarrow\neg q$ \cr
\noalign{\hrule}
T   &  T   &  F  &  F & F & F  \cr
T   &  F   &  F  &  F & T & T  \cr
F   &  T   &  T  &  T & F & T  \cr
F   &  F   &  F  &  T & T & T  \cr
}}$$

Since the $4^{th}$ and $6^{th}$ columns do not match, the propositions are not logically equivalent.

\medskip


\item The strings are $\lambda$, $abc$, $ababcc$, and $abababccc$.

\medskip

\item The strings consist of zero or more repetitions of the pattern $ab$ followed on the
right by an equal number of $c$'s.

\medskip

\item The sequence begins $w_0=2$, $w_1=1$, $w_2 =1+2 = 3$, $w_3 = 3+1 = 4$,
$w_4 = 4+3 = 7$, $w_5 = 7+4 = 11$. 

\medskip

\item Using the geometric sum formula the total is 
$\displaystyle \frac{1- \left(\frac{1}{3}\right)^{51}}{1-\frac{1}{3}}.$

That answer can be simplified if you want (but it's not necessary) to 
$\displaystyle \frac{3^{51}-1}{2(3^{50})}$.
\medskip

\item For the basis step, $n=1$, we check to see if $\displaystyle \frac{1}{1\cdot2}= \frac{1}{2}$, and it does.

For the inductive step, assume 

\[
\frac{1}{1\cdot2}+\frac{1}{2\cdot3}+\frac{1}{3\cdot4}+\cdots+\frac{1}{n(n+1)}= \frac{n}{n+1}
\]

for some $n\geq 1$. Now we try to prove

\[
\frac{1}{1\cdot2}+\frac{1}{2\cdot3}+\frac{1}{3\cdot4}+\cdots+\frac{1}{n(n+1)} +\frac{1}{(n+1)(n+2)}
= \frac{n+1}{n+2}.
\]

Beginning with the left side of that equation, and using the assumption we get

\begin{align*}
\frac{1}{1\cdot2}+\frac{1}{2\cdot3}+\frac{1}{3\cdot4}+&\cdots+\frac{1}{n(n+1)} +\frac{1}{(n+1)(n+2)}\\[5pt]
& = \frac{n}{n+1}+ \frac{1}{(n+1)(n+2)}\\[5pt]
&=  \frac{n(n+2)}{(n+1)(n+2)}+ \frac{1}{(n+1)(n+2)}\\[5pt]
&= \frac{n^2+2n+1}{(n+1)(n+2)}\\[5pt]
&= \frac{(n+1)(n+1)}{(n+1)(n+2)}\\[5pt]
&= \frac{n+1}{n+2} \text{ as we needed to show.}  \clubsuit\\[5pt]
\end{align*}


\medskip

\item The proposition is false. A counterexample is $6|90$ and $10|90$, but $60$ does not divide $90$.

\medskip

\item

\medskip

 The table is:
\[
\vbox{\offinterlineskip
\halign{\strut \vrule \hskip .5em \hss $#$ \hss & 
\vrule \hskip .5em \hss $#$ \hss & 
\vrule \hskip .5em \hss $#$ \hss & 
\vrule \hskip .5em \hss $#$ \hss &  
\vrule \hskip .5em \hss $#$ \hss & 
\vrule \hskip .5em \hss $#$ \hss\kern 4pt\vrule
 \cr
\noalign{\hrule}
715&297&121&55&11&0\cr
\noalign{\hrule}
 &&2&2&2&5\cr
\noalign{\hrule}
0&1&2&5&12&\cr
\noalign{\hrule}
1&0&1&2&5&\cr
\noalign{\hrule}
}}
\]
 $\gcd(715,297) = 11$

\item From the table above: $715(5) + 297(-12) = 11$.

\medskip 


\item True. This is one of the basic properties we proved for primes.

\medskip

\item Since $320 = 32\cdot 10 = 64\cdot 5 = 2^6\cdot 5$, we know positive divisors will look
like $2^a\cdot5^b$ where $a$ can be any of the seven values $0,1,2,3,4,5,6$ while $b$ has two
options: $0,1$. That gives a grand total of $7\cdot2 = 14$ positive integers that divide $320$.

\medskip

\item False since $11-(-123) = 134$ is not divisible by $31$. 

\medskip

\item Since $12\equiv 2\,(mod\,5)$, the stated problem is the same as $2x\equiv 2\,(mod\,5)$,
and since obviously $x\equiv 1\,(mod\,5)$ is the only solution to that equivalence equation.

\medskip

\item Answer: $\displaystyle \binom{13}{1}\binom{4}{3}\binom{12}{1}\binom{4}{2}$. Reasoning: we first pick one
of the thirteen ranks for the three of a kind, then pick three of the four cards of that rank, then pick one of the twelve remaining ranks for the pair, then two of the four cards of that rank. Since we need to do all four tasks, we use the product rule. 


\medskip


\item The term involving $x^{10}y^{15}$ is $\displaystyle \binom{25}{10}(2x)^{10}(-3y)^{15}$, and so the
coefficient is $\displaystyle -\binom{25}{10}2^{10}3^{15}$. We'll leave the answer in this form of course!


\medskip

\item Let's use inclusion-exclusion to do the counting. Let $X$ be the set of strings that begin $xx$, 
and $O$ the set of strings that end $ooo$. We need to count $X\cup O$.

\[
|X\cup O| = |X| + |O| - |X\cap O| = 26^4 + 26^3 - 26
\]

\medskip

\item False. The stated relation is correct only if $A$ and $B$ are disjoint.

\medskip

\item The characteristic equation is $r^2 -3r-10 = 0$. The left side factors as $(r+2)(r-5)$, so the characteristic
roots are $r = -2, 5$. That means the general solution is $a_n = A(-2)^n + B(5^n)$.

The initial conditions require  $1 = A + B$  and $1 = -2A + 5B$. Multiplying the first equation by $2$ and adding
it to the second shows $3 = 7B$ so $\displaystyle B = \frac{3}{7}$. Since $A+B=1$, we must have 
$\displaystyle A= \frac{4}{7}$.
So the solution is $\displaystyle a_n = \left(\frac{4}{7}\right)(-2)^n + \left(\frac{3}{7}\right)5^n$. 

\medskip

\item False. The correct form of the guess is the most general form of the nonhomogeneous part, and that would be
$An^2+Bn+C$.

\medskip

\item False. Example: The $3$-link, $L_{3}$ has a Eulerian path, but does not have a Eulerian circuit.

\medskip 

\item True. The number of edges in a tree is always one less than the number of vertices.
 
\medskip

\item True. Start at the {\it axle} vertex, go out to the {\it rim}, to a vertex $a$, go around the rim
(say clockwise, just to be specific) until you reach the vertex {\it just one before} $a$, and then return
to the axle vertex.


\end{enumerate}
