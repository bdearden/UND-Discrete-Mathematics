\newcommand{\notdivides}{{\mathrm{/}\mkern-9.0mu{|}}} %replaces \cent in math mode
\chapter{The \textsf{divides} Relation and Primes}

\newthought{Given integers $a$ and $b$ we say} that $a$ {\bfseries divides} $b$ and write $a|b$ provided%
\sidenote{That is, $a$ divides into $b$ \textbf{evenly}.}
there is an integer $c$ with $b=ac$.
 In that case we also say that 
$a$ is a {\bfseries factor} of $b$, or that $a$ is a {\bfseries divisor}
of $b$, or 
 that $b$ is a {\bfseries multiple} of $a$.  For example $3|12$ since 
$12 = 3\cdot 4$. Keep in mind that
{\itshape divides} is  a relation. When you see $a|b$ you should think 
{\itshape is that true or false}. Don't write things like $3|12 = 4$!
If $a$ does not divide $b$,  write $a \notdivides b$.
 For example, it is true%
\sidenote{Fact: $3$ does not divide into $13$ \textbf{evenly}.}
that $3 \notdivides{13}$.
 

\section{Properties of \textsf{divides}}
Here is a list of a few simple facts about the divisibility relation. 
\begin{thm}
For $a,b,c\in \Z$ we have
\begin{enumerate}
  \item $a|0$
  
  \item $\pm 1|a$ 
  
  \item If $a|b$, then $-a|b$
  
  \item If $a|b$ and $b|c$, then $a|c$. 
  So $a|b$ is a transitive relation on $\Z$
  
  \item $a|-a$
  
  \item If $a|b$ and $b\not=0$, then $0<|a|\leq |b|$
  
  \item If $a|1$, then $a=\pm 1$
  
  \item If $a|b$ and $b|a$, then $a=\pm b$
  
  \item $a|b$ and $a|c$, then $a|(mb+nc)$ for all $m,n\in \Z$
  
  \item If $a|b$, then $a|bc$ for all $c\in \Z$
\end{enumerate}
\end{thm} 

Here are the proofs of a few of these facts.
\begin{enumerate}\itshape
 \item[(1)] \textbf{Proof.} For any integer $a$, $a0 = 0$, so $a|0$. \qed
 
 \item[(4)] \textbf{Proof.} Suppose $a|b$ and $b|c$. That means there are
 integers $s,t$ so that $as=b$ and $bt=c$. Substituting $as$ for $b$ in the
 second equation gives $(as)t = c$, which is the same as $a(st) = c$. 
 That shows $a|c$. \qed
 
 \item[(9)] \textbf{Proof.} Suppose $a|b$ and $a|c$. That means
 there are integers $s,t$ such that $as=b$ and $at=c$. Multiply
 the first equation by $m$ and the second by $n$ to get
 $a(sm)= mb$ and $a(tn) = nc$. Now add those two equations:
 $a(sm) + a(tn) = mb+nc$. Factoring out the $a$ on the left shows
 $a(sm+tn) = mb+nc$, and so we see $a|(mb+nc)$. \qed
\end{enumerate}


\section{Prime numbers}
The prime integers play a central role in number theory.
A positive integer larger than $1$ is said to be { \bfseries prime} if its only positive divisors are $1$ and
itself. The first few primes are $2,3,5,7,11,13,17,19,23,29,31,37$.


A positive integer larger than $1$ which is not prime is { \bfseries composite}. So a composite number
$n$ has a positive divisor $a$ which is neither $1$ nor $n$. By part (6) of the theorem above, $1<a<n$. 

So, to check if an integer $n$ is a prime, we can trial divide it in turn by 
$2,3,4,5,\cdots n-1$, and if we find one of these that divides $n$, we
can stop, concluding that $n$ is not a prime. On the other hand,
if we find that none of those divide $n$, then we can conclude $n$
is a prime. This algorithm for checking a number for primeness can
be made more efficient. For example, there is really no need to test to see
if $4$ divides $n$ if we have already determined that $2$ does not divide
$n$. And the same reasoning shows that to test $n$ for primeness
we need only check in to see if $n$ is divisible by any of
$2,3,5,7,11,13$ and so on up to the largest prime less than $n$.
For example, to test $15$ for primeness, we would trial divide by
the six values $2,3,5,7,11,13$. But even this improved algorithm can 
be made more efficient by the following theorem.

\begin{thm}
Every composite number $n$ has a divisor $a$, with \[2\leq a\leq \sqrt{n}.\]
\end{thm}
\begin{proof} 
Suppose $n$ is a composite integer. That means 
$n=ab$  where $1<a,b<n$.  Not both $a$ and $b$ are greater than 
$\sqrt{n}$, for if so
 $n=ab>\sqrt{n}\sqrt{n}=(\sqrt{n})^2=n$, and that is a 
contradiction.
\end{proof}

So, if we haven't found a divisor of $n$ by the time we reach
$\sqrt{n}$, then $n$ must be a prime.

We can be a little more informative, as the next theorem shows.

\begin{thm}
Every integer $n>1$ is divisible by a prime.
\end{thm}
\begin{proof}
Let $n>1$ be given. The set, $D$, of all integers greater
than $1$  that divide $n$ is nonempty since $n$ itself is certainly in that
set. Let $m$ be the smallest integer in that set. Then $m$ must be a prime
since if $k$ is an integer with $1<k<m$ and $k|m$, then $k|n$, and so
$k\in D$. That is a contradiction since $m$ is the smallest element of 
$D$. Thus $m$ is a prime divisor of $n$.
\end{proof}


Among the more important theorems in number theory is the following.
\begin{thm}
The set of prime integers is infinite.
\end{thm}
\begin{proof}
Suppose that there were only finitely many primes.
List them all: $2,3,5,7,\cdots,p$. Form the number 
$N=1+2\cdot3\cdot5\cdot7\cdots p$. According to the last theorem, there
must be a prime that divides $N$, say $q$. Certainly $q$ also
divides $2\cdot3\cdot5\cdot7\cdots p$ since that is the product of all 
the primes, so $q$ is one of its factors. Hence $q$ divides
$N-2\cdot3\cdot5\cdot7\cdots p$. But that's crazy since
$N-2\cdot3\cdot5\cdot7\cdots p=1$. We have reached a contradiction,
and so we can conclude there are infinitely many primes.
\end{proof}

\section{The division algorithm for integers}
\begin{thm}[The Division Algorithm for Integers] 
If $a,d\in \Z$, with $d>0$, there exist\marginnote{
The quantities $q$ and $r$ are called the { \bfseries quotient}
and {\bfseries remainder} when $a$ is divided by $d$.}
unique integers $q$ and $r$, with $a=qd+r$, and $0\leq r<d$.

\end{thm}
\begin{proof}
Let $S=\{a-nd|n\in \Z, {\rm {~and~}} a-nd \geq 0\}$. Then $S\neq \emptyset$, since $a-(-|a|)d\in S$ for sure.
Thus, by the Well Ordering Principle, $S$ has a least element, call it $r$. Say $r=a-qd$. Then 
we have $a=qd+r$, and $0\leq r$. If $r\geq d$, then $a=(q+1)d+(r-d)$, with $0\leq r-d$ contradicting
the minimality of $r$. 

To prove uniqueness, suppose that $a=q_1d+r_1=q_2d+r_2$, with $0\leq r_1,r_2<d$. Then 
$d(q_1-q_2)=r_2-r_1$ which implies that $r_2-r_1$ is a multiple of $d$. Since $0\leq r_1,r_2<d$,
we have $-d<r_2-r_1<d$. Thus the only multiple of $d$ which $r_2-r_1$ can possibly be is $0d=0$.
So $r_2-r_1=0$ which is the same thing as $r_1=r_2$. Thus $d(q_1-q_2)=0=d0$. Since $d\neq 0$ we
can cancel $d$ to get $q_1-q_2=0$, whence $q_1=q_2$.$\quad$ $\;$\
\end{proof}


\clearpage

\section{Exercises}

\begin{exer}
  Determine the quotient and remainder when $107653$ is
divided by $22869$.
\end{exer}

\begin{exer}
Determine if $1297$ is a prime.
\end{exer}

\begin{exer}
Prove or give a counterexample: The divides relation is reflexive.
\end{exer}

\begin{exer}
Prove or give a counterexample: The divides relation is symmetric.
\end{exer}

\begin{exer}
Prove or give a counterexample: The divides relation is transitive.
\end{exer}

\begin{exer}
What is wrong with the expression $4|12 = 3$?
\end{exer}

\begin{exer}
Show that none of the $1000$ consecutive integers $1001! + 2$ to $1001!+1001$ are primes.
\end{exer}

\begin{exer}
 Prove: For $a,b,c\in \mathbb{Z}$, if $a|b$, then $-a|b$.
\end{exer}

\section{Problems}

\begin{prob}
For positive integers, $a$ and $b$, if the quotient when $a$ is divided by $b$ is $q$,
what are the possible quotients when $a+1$ is divided by $b$?
\end{prob}

\begin{prob}
For positive integers, $a$ and $b$, if the quotient when $a$ is divided by $b$ is $q$,
what are the possible quotients when $2a$ is divided by $b$?
\end{prob}

\begin{prob}
Prove: For integers $a,b$, if $a|b$, then $-a|b$.
\end{prob}

\begin{prob}
 Prove or give a counterexample: If $p$ is a prime, then $2p+1$ is a prime.
\end{prob}

\begin{prob} 
Determine all the integers that $0$ divides. \\
(Hint: Think about the definition of the divides relation.\\
The correct answer is probably not what you expect.)
\end{prob}

\begin{prob}
Determine if $3599$ is a prime. (Hint: This is easy since $3599 = 3600 - 1$)
\end{prob}

\begin{prob}
Determine if $5129$ is a prime.
\end{prob}

\begin{prob}
Prove property 10 of Theorem 21.1: For integers $a,b,c$, if $a|b$, then $a|bc$.
\end{prob}

\begin{prob}
Suppose the remainder when $a$ is divided by $b$ is $r$. Determine the remainder when $a+2b$ is divided by $b$. More generally, if $k$ is any integer, determine the remainder when $a+kb$ is divided by $b$.
\end{prob}

\begin{prob}
Show that for any integer $n$, there are $n$ consecutive non-prime integers.
\end{prob}

